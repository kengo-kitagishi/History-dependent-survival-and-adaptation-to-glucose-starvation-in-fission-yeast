\chapter{Introduction}
\section{Introduction}
【genreral】
1,Microorganismは変化する環境の中で生きる
Microorganism constantly transition between environment with dramatically different osmolarities(あの論文の条件ほどdramaticallyでないな...)
Stress adaptationを理解することは非常に重要である。
2,さまざまな環境変化に対する内部の変化が調べられているが、Although genes and environmental condition affecting starvation survival have been identified.

fitness is determined by how well they survive when nutrients are depleted.
【Technical purpose】
Here, we study carbon-starved {Schizosaccharomyces Pombe}

We imaged individual starving {S.pombe} with live-cell time-lapse microscopy.
We thereby captured the メインの結果。

As in previous picture
\begin{figure}[htbp]
\centerline{\includegraphics[width=0.8\textwidth]{figure/UTokyo_logo.png}}
\caption{The Univ. of Tokyo\cite{nonlinear}} 
\end{figure}

\begin{figure}[H]
  \centering
  \newcommand{\subfig}[2]{%
    \subfloat[]{\includegraphics[width=0.55\textwidth]{#1}\label{#2}}%
  }
  \subfig{figure/UTokyo_logo.png}{utokyo:1} \\
  \subfig{figure/UTokyo_logo.png}{utokyo:2}

  \captionsetup{font=small}
  \caption{(a) is The Univ. of Tokyo, (b) is The Univ. of Tokyo}
  \label{utokyo}
\end{figure}

The Univ. of Tokyo logo is in Fig. \ref{utokyo:1}.

\begin{figure}[H]
  \centering
  \newcommand{\subfig}[2]{%
    \subfloat[]{\includegraphics[width=0.4\textwidth]{#1}\label{#2}}%
  }
  \subfig{figure/UTokyo_logo.png}{utokyo:3} 
\end{figure}

\begin{figure}[H]
  \centering
  \newcommand{\subfig}[2]{%
    \subfloat[]{\includegraphics[width=0.4\textwidth]{#1}\label{#2}}%
  }
  \subfig{figure/UTokyo_logo.png}{utokyo:4}

  \captionsetup{font=small}
  \caption{(a) is The Univ. of Tokyo, (b) is The Univ. of Tokyo}
  \label{utokyo:s}
\end{figure}

The Univ. of Tokyo logo is in Fig. \ref{utokyo:s}.

\begin{table}[H]
\centerline{\includegraphics[width=0.7\textwidth]{figure/UTokyo_logo.png}}
\captionsetup{font=small}
\caption{The Univ. of Tokyo}
\label{cep}
\end{table}
