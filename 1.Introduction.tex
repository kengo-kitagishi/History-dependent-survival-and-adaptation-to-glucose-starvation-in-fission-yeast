\chapter{Introduction}
\section{Introduction}
 Microorganisms live in changing environments. It is said that adaptability changes based on past experiences.
 
 Microorganisms constantly transition between environments with dramatically different environments.
 Organisms have adopted a variety of strategies to cope with situations in which energy is limited. For example, when nutrients become limiting for growing and dividing cells, they can exit the cell division cycle and enter quiescence, similar to when they start differentiating. 
 
 Such nutrient starvation-induced quiescence reverses as soon as nutrients become available again. In extreme cases of energy shortage, cells can enter a dormant state of little or no energy consumption.
  
 It occasionally enters a dormant state. The cytoplasm has been described as a complex viscoelastic fluid, a gel-like material or a colloidal liquid at the transition to a glass-like state. The maintenance of such a complex, anisotropic cellular architecture and its remodelling in response to environmental changes requires a constant input of energy.  And that structure itself has biological significance.

2,さまざまな環境変化に対する内部の変化が調べられているが、Although genes and environmental conditions affecting starvation survival have been identified.

fitness is determined by how well they survive when nutrients are depleted.

As a collective, they have been proposed to form higher-order structures that mediate a solid-like state of the cytoplasm.

Yanagita 論文によれば過去の表現型の変化が効いてくる可能性があるとされているが
low glucose-state で起こっていることがその後の生存にどのように関わってくるかがわからない。

 分子的なメカニズムはわからないので今回我々は細胞質密度に着目して追跡を行った。non-essential gene3400株で調べたが、よくわからなかったそう。
あとallen2006isolation
[To further investigate the molecular nature of CF, we used our automated quantification of LD motion to screen for gene deletions that prevent CF in cells during deep starvation, by using a library of 3400 fission yeast strains that each carry a deletion of a non-essential gene (see Materials and Methods; \href{https://pmc.ncbi.nlm.nih.gov/articles/PMC6857596/\#JCS231688C32}{Kim et al., 2010}). The strains were screened on SD8 because, due to the presence of multiple auxotrophic mutations, these strains entered deep starvation with a delay. Of the roughly 500 deletions strains that did not show CF during deep starvation, we noticed a clear accumulation of mutants affecting autophagy. Autophagy is an evolutionarily conserved mechanism used by cells to remove damaged organelles and to recycle cellular components (\href{https://pmc.ncbi.nlm.nih.gov/articles/PMC6857596/\#JCS231688C52}{Nakatogawa et al., 2009}). To investigate the role of autophagy in CF, we took a prolonged look at two strains carrying a deletion of either \textit{atg1} (\textit{atg1}Δ) or \textit{atg8} (\textit{atg8}Δ), genes that both encode essential autophagy pathway components. Cells of both mutant strains showed normal exponential growth and entered quiescence at a time similar to that of wild-type cells (\href{https://pmc.ncbi.nlm.nih.gov/articles/PMC6857596/\#JCS231688F5}{Fig. 5}A), and both mutant strains entered the CF state, although with a 2–3-day delay compared with wild type (\href{https://pmc.ncbi.nlm.nih.gov/articles/PMC6857596/\#JCS231688F5}{Fig. 5}B,C; \href{http://movie.biologists.com/video/10.1242/jcs.231688/video-8}{Movie 8}). Consistently, cell wall digestion of \textit{atg1}Δ and \textit{atg8}Δ mutant cells in a hypotonic environment produced spherical protoplasts between SD6 and SD8. However, only at SD9 did they robustly maintain a cylindrical shape similar to that in wild-type cells at SD6 (\href{https://pmc.ncbi.nlm.nih.gov/articles/PMC6857596/\#JCS231688F5}{Fig. 5}D). These results indicate that autophagy is not essential for CF but that it promotes the ability of cells to enter the CF state.]

【Technical purpose】
Here, we study carbon-starved {Schizosaccharomyces Pombe}

We imaged individual starving {S.pombe} with live-cell time-lapse microscopy.
We thereby captured the メインの結果。

We discribe fission yeast cells that have slowly run out of carbon-source in stages, these cell showed a increase in intracellular mass density in  成長をドラスティックに止めるのに十分なほどのlow glucose condition  which affect 次の完全グルコース飢餓における成長脳の喪失を防いでいる。
As in previous picture, 























\begin{figure}[htbp]
\centerline{\includegraphics[width=0.8\textwidth]{figure/UTokyo_logo.png}}
\caption{The Univ. of Tokyo\cite{nonlinear}} 
\end{figure}

\begin{figure}[H]
  \centering
  \newcommand{\subfig}[2]{%
    \subfloat[]{\includegraphics[width=0.55\textwidth]{#1}\label{#2}}%
  }
  \subfig{figure/UTokyo_logo.png}{utokyo:1} \\
  \subfig{figure/UTokyo_logo.png}{utokyo:2}

  \captionsetup{font=small}
  \caption{(a) is The Univ. of Tokyo, (b) is The Univ. of Tokyo}
  \label{utokyo}
\end{figure}

The Univ. of Tokyo logo is in Fig. \ref{utokyo:1}.

\begin{figure}[H]
  \centering
  \newcommand{\subfig}[2]{%
    \subfloat[]{\includegraphics[width=0.4\textwidth]{#1}\label{#2}}%
  }
  \subfig{figure/UTokyo_logo.png}{utokyo:3} 
\end{figure}

\begin{figure}[H]
  \centering
  \newcommand{\subfig}[2]{%
    \subfloat[]{\includegraphics[width=0.4\textwidth]{#1}\label{#2}}%
  }
  \subfig{figure/UTokyo_logo.png}{utokyo:4}

  \captionsetup{font=small}
  \caption{(a) is The Univ. of Tokyo, (b) is The Univ. of Tokyo}
  \label{utokyo:s}
\end{figure}

The Univ. of Tokyo logo is in Fig. \ref{utokyo:s}.

\begin{table}[H]
\centerline{\includegraphics[width=0.7\textwidth]{figure/UTokyo_logo.png}}
\captionsetup{font=small}
\caption{The Univ. of Tokyo}
\label{cep}
\end{table}
