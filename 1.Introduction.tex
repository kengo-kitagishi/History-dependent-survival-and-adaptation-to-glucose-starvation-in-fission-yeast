\chapter{Introduction}
タイトル案(この論文の新規claimを端的に伝えるものが良い)
Highlights(複数個あって良いが、4つ以上だと多い。その場合は2つの論文に分けるべき)
Point 1
Point 2
…
(Abstractはこの段階では書かない)
Introduction
一般的背景(Communication purpose)
Molecular crowding is crucial for biological function.
Molecular crowding is a prominent physical property of the cytoplasm. There

dormancy



個別背景(Technical purpose)
low-glucose accliminated S.pombe profoundly endure in subsequent carbon-free medium
cytoplasmic solidificationとcytoplasmic freezing
死ぬ細胞と生きる細胞における物理的な状態の違い、どう言った細胞が生きて、どういった細胞が死ぬのかの違いを見極めることがむずかしい。
具体的課題
Here, we hogehoeg
この論文は、上記具体的課題に対し、どのような解決策や新規知見をもたらしたか
Results
acute
osmotic induced endure
... 

\section{Introduction}
[Communication Purpose] 
dormancy \& physical property

(dormancy)
Microorganisms live in changing environments. It is said that adaptability changes based on past experiences. 
Microorganisms constantly transition between environments with dramatically different environments. 
Organisms have adopted a variety of strategies to cope with situations in which energy is limited. 
For example, when nutrients become limiting for growing and dividing cells, they can exit the cell division cycle and enter quiescence.
In extreme cases of energy shortage, cells can enter a dormant state of little or no energy consumption.)
Such nutrient starvation-induced quiescence reverses as soon as nutrients become available again.
As in previous research, 

(physical property)
Molecular crowding is a fundamental feature of \cite{lohka1985induction}\cite{zhou2008macromolecular}\cite{miermont2013severe}\cite{chen2024viscosity}\cite{neurohr2020relevance}
The cytoplasm has been described as a complex viscoelastic fluid, a gel-like material or a colloidal liquid at the transition to a glass-like state.\cite{nishizawa2017universal}
The maintenance of such a complex, anisotropic cellular architecture requires a constant input of energy. \cite{ebata2023activity}

Although genes and environmental conditions affecting starvation survival have been identified. what physical property cause dormant cells irreversibilty lose their vaiability and what determines their life span during staravation have been discovered
low glucose-state で起こされるさまざまなphysical propartyの変化がその後の生存にどのように関わってくるかがわからない。 


【Technical purpose】
Here, we study carbon-starved {Schizosaccharomyces Pombe}

We imaged individual starving {S.pombe} with live-cell time-lapse microscopy.
We thereby captured the メインの結果。

We discribe fission yeast cells that have slowly run out of carbon-source in stages, these cell showed a increase in intracellular mass density in  成長をドラスティックに止めるのに十分なほどのlow glucose condition which affect 次の完全グルコース飢餓における成長能の喪失を防いでいる。























\begin{figure}[htbp]
\centerline{\includegraphics[width=0.8\textwidth]{figure/UTokyo_logo.png}}
\caption{The Univ. of Tokyo\cite{nonlinear}} 
\end{figure}

\begin{figure}[H]
  \centering
  \newcommand{\subfig}[2]{%
    \subfloat[]{\includegraphics[width=0.55\textwidth]{#1}\label{#2}}%
  }
  \subfig{figure/UTokyo_logo.png}{utokyo:1} \\
  \subfig{figure/UTokyo_logo.png}{utokyo:2}

  \captionsetup{font=small}
  \caption{(a) is The Univ. of Tokyo, (b) is The Univ. of Tokyo}
  \label{utokyo}
\end{figure}

The Univ. of Tokyo logo is in Fig. \ref{utokyo:1}.

\begin{figure}[H]
  \centering
  \newcommand{\subfig}[2]{%
    \subfloat[]{\includegraphics[width=0.4\textwidth]{#1}\label{#2}}%
  }
  \subfig{figure/UTokyo_logo.png}{utokyo:3} 
\end{figure}

\begin{figure}[H]
  \centering
  \newcommand{\subfig}[2]{%
    \subfloat[]{\includegraphics[width=0.4\textwidth]{#1}\label{#2}}%
  }
  \subfig{figure/UTokyo_logo.png}{utokyo:4}

  \captionsetup{font=small}
  \caption{(a) is The Univ. of Tokyo, (b) is The Univ. of Tokyo}
  \label{utokyo:s}
\end{figure}

The Univ. of Tokyo logo is in Fig. \ref{utokyo:s}.

\begin{table}[H]
\centerline{\includegraphics[width=0.7\textwidth]{figure/UTokyo_logo.png}}
\captionsetup{font=small}
\caption{The Univ. of Tokyo}
\label{cep}
\end{table}
