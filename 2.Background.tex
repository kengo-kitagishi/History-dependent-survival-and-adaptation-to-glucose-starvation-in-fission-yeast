\chapter{Background}

\section{細胞の休眠状態における細胞質の状態}

酵母細胞は、温度変化、浸透圧、酸化ストレスなど、さまざまなストレスに直面すると、代謝的および形態的な変化を起こす。

Munderら(2023)は、エネルギー不足により細胞内pHが低下すると、酵母細胞が休眠状態への移行を促進することを示しました。細胞質が酸性pHになると、細胞は体積を縮小し、高分子集合体を形成しました。これらの細胞は、オルガネラの移動性や外因性トレーサー粒子の移動性が低下し、活性細胞に比べて測定可能なほど硬くなりました。

栄養が枯渇した環境では、単細胞生物である大腸菌や酵母、哺乳類細胞は増殖を停止し、生存のための休眠状態に入る。細胞がエネルギー枯渇状態に陥ったり細胞質pHが低下したりすると、細胞質全体がガラス状態とも形容される非常に拡散の遅い状態に転移する。


\section{分裂酵母におけるグルコース飢餓時の細胞質固体化現象}

\subsection{1. 細胞質固体化のメカニズム}
細胞質固体化とは、栄養枯渇により細胞質が流動的な状態から固体的(ガラス様)の状態に移行し、細胞内部の構造物の動きが大きく制限される現象を指す。分裂酵母 \textit{Schizosaccharomyces pombe} では、グルコース源が枯渇すると細胞質の動的性質が劇的に変化し、多くの細胞内顆粒や小器官がほとんど拡散しなくなることが観察されている。一方で、小さな分子(例:GFP程度のサイズ, 約4–5 nm)は固体化後も細胞質内を比較的自由に拡散できる。すなわち、固体化した細胞質ではサイズ依存的な可動性の差が生じ、リボソームのような大きな分子複合体の拡散は著しく抑制される一方で、小分子や低分子量タンパク質は移動可能なままとなる。この固体化状態は可逆的であり、適切な条件下で再び流動的な細胞質へ戻りうる。
この現象のメカニズムには複数の要因が関与すると考えられている。
\begin{itemize}
    \item エネルギー枯渇と細胞質pHの低下:例えば出芽酵母では、グルコース欠乏による休眠時に細胞質pHが顕著に低下し、それが引き金となって広範な高分子会合(マクロ分子の集合体形成が起こり、細胞質がより剛直な固体様状態へ転移する。このような酸性環境下でのタンパク質会合は細胞質粘度を増大させる一因であると考えられる。
    \item 細胞容積の縮小と分子混雑の増大:グルコース飢餓に際し、細胞は浸透圧や液胞機能の変化によって細胞全体の容積を縮小させることがある。出芽酵母の研究では、グルコース欠乏後に細胞体積が減少し、細胞内の高分子混雑度(macromolecular crowding)が上昇する結果、拡散制限が生じることが報告された。実際、ATP濃度低下時の酵母細胞では細胞膜の内側への陥入や細胞質濃縮が起こり、細胞質がコロイドガラスのような状態に近づくとされる。このような細胞質の高密度化は、細胞内構造物の移動を物理的に拘束し、固体化を促進する要因となる。
    \item 細胞骨格の再編成:細胞質固体化に際して、細胞骨格構造にも顕著な変化が生じる。分裂酵母では飢餓に入ると微小管は急速に脱重合し、飢餓開始4–5日目にはごく短い断片を除いてほぼ完全になくなる。一方、アクチン細胞骨格は再編成を経て、飢餓数日後には細胞周囲に沿って伸びる太く安定なFアクチン束が形成される。飢餓6日目にはこの「靴紐状」のアクチン束が細胞内アクチンの大部分を占め、動的なターンオーバーをほとんど示さなくなる。しかし興味深いことに、アクチン重合阻害剤ラトランクリンBを添加してこれらのアクチン構造の形成を阻害しても、細胞質固体化(CF)は正常に起こり得ることが示された。また、アクチン遺伝子の変異株においても固体化への移行が観察されている。これらの結果は、アクチンや微小管などの細胞骨格そのものは固体化現象の必須要因ではないことを示唆している。むしろ細胞骨格の消失や再編成は、固体化に伴う細胞質環境変化の結果または付随現象と考えられる。
    \item タンパク質の会合体・凝集体形成:栄養欠乏下では、多くの酵素やリボソーム関連タンパク質、mRNA結合タンパク質などが凝集体や高次会合体を形成することが知られている。例えば出芽酵母の休眠細胞では、解糖系酵素や翻訳因子eIF2Bがフィラメント状または顆粒状の膜の無いコンデンサ(液滴様構造)を形成する。またプロテアソームや細胞骨格要素、ストレス顆粒(Pボディ)などが飢餓に応答して可逆的に集合し、酵素活性を一時的に停止するとともに貯蔵形態として蓄えられる。分裂酵母においても長期飢餓で多数の顆粒状構造が観察され、電子顕微鏡解析から細胞質内に無秩序(アモルファス)または秩序だった高分子集合体が多数出現することが示された。こうした液滴やフィラメントの形成が細胞質全体の網目状ネットワークを構築し、細胞質を全体として固化させる一因になると考えられる。
    \item 細胞内リサイクルと代謝物の蓄積:長期の炭素飢餓に適応するため、細胞はオートファジー(自食作用)によって不要なタンパク質や細胞小器官を分解し、アミノ酸や脂肪酸などを再利用する。実際、分裂酵母でオートファジーに必須の遺伝子(\textit{atg1}や\textit{atg8})を欠損させると、固体化状態への移行が2〜3日遅れることが報告されている。これはオートファジーが固体化達成の促進因子であり、細胞内のリサイクル産物(例えばエネルギー源や構造分子)を供給することで固体化を助長していることを示唆する。また、胞子や他の休眠細胞ではトレハロースなどの高濃度の保護糖が蓄積し、細胞質のガラス化に寄与することが知られている。分裂酵母の胞子では、トレハロースが細胞質粘性を高める重要な要因となっており、発芽(休眠解除)時にはPKA経路によりトレハロース分解酵素が活性化され、この糖が分解されることで細胞質が急速に流動化することが確認されている。興味深い点として、胞子の細胞質流動化には新規タンパク質合成や細胞骨格再構築を要さず、主に蓄積糖の分解によって物理化学的性質が変化する。以上より、長期飢餓下の栄養不足細胞では細胞自身の物質分解と蓄積代謝物の利用により細胞質環境を調節し、固体化状態を成立・維持していると考えられる。
\end{itemize}
以上のように、分裂酵母における細胞質固体化は多因子的な機構によってもたらされる。【図式的に言えば、栄養エネルギー枯渇 → pH低下・ATP減少 → 分子混雑度増加・水分減少 → タンパク質会合体網の形成】といった段階が考えられ、それらによって「細胞質を流動から固体へ転移させるスイッチ」が入ると考察されている。実際、分裂酵母細胞を長期培養してグルコースを使い果たさせると、急激に細胞質が固化する「細胞質の凍結(cytoplasmic freezing, CF)」態へとスイッチすることが確認されている。この状態では細胞内容積が約20\%減少し(対数増殖期の細胞では約43\%減少)、細胞質の見かけの弾性率**(剛性)が有意に上昇することから、力学的にも硬いゲル状・ガラス状の物質に変わったとみなせる。

\subsection{2. 固体化と休眠状態(Quiescence)との関係}
グルコース欠乏による細胞質固体化は、細胞の休眠(quiescence)状態と深く関連している。休眠状態とは、細胞周期が停止し増殖を休止した状態(いわゆるG0期)であり、不利な環境下で生存を維持するための待機モードといえる。分裂酵母では、窒素源が欠乏すると接合・胞子形成に入る一方、相手がいない場合は栄養が十分でも細胞周期を停止して休眠状態に入ることが知られている\href{https://pmc.ncbi.nlm.nih.gov/articles/PMC6857596/\#:~:text=When\%20nutrients\%20become\%20growth\%20limiting,yeast\%20cells\%20that\%20have\%20slowly}{pmc.ncbi.nlm.nih.gov}。一方、炭素源(グルコース)の枯渇でも細胞は増殖を停止し、エネルギー不足に適応した休眠様の状態になる\href{https://pmc.ncbi.nlm.nih.gov/articles/PMC3123465/\#:~:text=stochastic\%2C\%20accompanied\%20by\%20a\%20curious,choline\%29\%2C\%20which\%20increased\%20or}。このように、栄養飢餓は細胞を休眠へ導く主要なシグナルであり、固体化現象はまさに休眠細胞に特有の細胞質状態として現れる。

実験的事実として、分裂酵母をグルコース枯渇培地で培養すると約2日で細胞増殖が停止して休眠に入り、その後数日にわたり細胞内構造の運動性低下が進行していく\href{https://pmc.ncbi.nlm.nih.gov/articles/PMC6857596/\#:~:text=Starved\%20cells\%20have\%20two\%20intracellular,immobilisation\%20states}{pmc.ncbi.nlm.nih.gov}\href{https://pmc.ncbi.nlm.nih.gov/articles/PMC6857596/\#:~:text=starvation,SD3\%20and\%20SD4\%2C\%20LDs\%20showed}{pmc.ncbi.nlm.nih.gov}。5~6日経過するとほぼ全ての細胞内小器官や顆粒の動きが停止し、完全な固体化(CF状態)に達する\href{https://pmc.ncbi.nlm.nih.gov/articles/PMC6857596/\#:~:text=constant\%20\%28Fig,Movie\%201\%3B\%20\%2059\%20Fig}{pmc.ncbi.nlm.nih.gov}。これは\textbf{休眠突入後さらに時間をかけて細胞質が深い固体状に変化する}ことを示唆する。興味深いことに、固体化した休眠細胞では細胞壁を酵素で分解しても内容物が流出せず、\textbf{細胞壁がなくとも元の棒状の形状を保つ}ほど細胞質が硬直している\href{https://pmc.ncbi.nlm.nih.gov/articles/PMC6857596/\#:~:text=spherical\%20shape,cells\%2C\%20the\%20majority\%20of\%20CF}{pmc.ncbi.nlm.nih.gov}\href{https://pmc.ncbi.nlm.nih.gov/articles/PMC6857596/\#:~:text=spherical\%20shape\%2C\%20CF\%20cells\%20slipped,Fig}{pmc.ncbi.nlm.nih.gov}。対照的に、対数増殖期の細胞を同様に処理すると細胞内容物が液状化して丸く膨潤するため、固体化細胞では\textbf{休眠に伴う細胞質の構造維持力}が飛躍的に高まっていることがわかる\href{https://pmc.ncbi.nlm.nih.gov/articles/PMC6857596/\#:~:text=spherical\%20shape,cells\%2C\%20the\%20majority\%20of\%20CF}{pmc.ncbi.nlm.nih.gov}。この性質は「細胞質がまるで細胞骨格・細胞壁のような役割を果たし、細胞形状を自律的に支えている」状態とも言える。

休眠状態に入った細胞では、細胞周期関連タンパク質の発現低下やシグナル経路の制御変化が起こり、代謝も大きく再編成される。例えば分裂酵母では、グルコース濃度が臨界値以下に低下すると細胞周期が停止する一方で\textbf{代謝産物の蓄積}や\textbf{ストレス応答経路の活性化}が見られる\href{https://pmc.ncbi.nlm.nih.gov/articles/PMC3123465/\#:~:text=stochastic\%2C\%20accompanied\%20by\%20a\%20curious,choline\%29\%2C\%20which\%20increased\%20or}{pmc.ncbi.nlm.nih.gov}。Yanagidaらの研究では、グルコース極限環境において即座に増殖停止した細胞は寿命が短いが、あらかじめ低グルコース環境で徐々に適応させて休眠状態に入れた細胞は長期間生存可能になることが示されている\href{https://pmc.ncbi.nlm.nih.gov/articles/PMC3123465/\#:~:text=narrow\%20range\%20of\%20concentrations\%20\%282,choline\%29\%2C\%20which\%20increased\%20or}{pmc.ncbi.nlm.nih.gov}\href{https://pmc.ncbi.nlm.nih.gov/articles/PMC3123465/\#:~:text=divided,choline\%29\%2C\%20which\%20increased\%20or}{pmc.ncbi.nlm.nih.gov}。このことは\textbf{緩やかな飢餓による休眠への移行}が、細胞に耐久性を持たせる重要なプロセスであることを意味し、その適応過程で細胞質固体化が達成されると推測される。

さらに、休眠細胞では\textbf{オートファジーや液胞機能}が顕著に活性化する。前述のとおりオートファジーは固体化を促進する役割があるが、同時にそれ自体が休眠状態維持に不可欠なプロセスである。不要になったリボソームの分解(リボファジー)や蓄積物質の動員によって、休眠細胞は基本的な代謝を賄い、長期生存に必要な最低限のエネルギーを供給する\href{https://pmc.ncbi.nlm.nih.gov/articles/PMC6857596/\#:~:text=this\%20state,mediate\%20the\%20CF\%20state\%20in}{pmc.ncbi.nlm.nih.gov}。特に液胞(動物細胞のリソソームに相当)は、休眠への移行期に内容積や数が変化し、細胞内のタンパク質や代謝物のリサイクルセンターとして機能する。分裂酵母では、飢餓が進行すると\textbf{液胞のサイズが小さくなり数が増加}することが観察されている\href{https://pmc.ncbi.nlm.nih.gov/articles/PMC6857596/\#:~:text=match\%20at\%20L559\%20ER\%20and,tagged}{pmc.ncbi.nlm.nih.gov}。これは多数のオートファジー小胞が液胞と融合し、内容物を分解している可能性を示唆する。実際、休眠細胞では液胞内に蓄積された顆粒状物質が活発に運動する様子が捉えられており、固体化した細胞質中にあっても\textbf{液胞内部は比較的流動的な区画}として機能しているようである\href{https://pmc.ncbi.nlm.nih.gov/articles/PMC6857596/\#:~:text=exception\%20were\%20small\%20particles\%20that,2\%20days\%20in\%20culture\%20as}{pmc.ncbi.nlm.nih.gov}。したがって、休眠状態では細胞全体としては静止・固化する一方、\textbf{液胞内など一部の区画では分解と物質輸送が続いている}と考えられる。この部分的な動的領域が細胞内リサイクルを担い、休眠細胞が生き延びるためのリソースを供給しているといえよう。

総じて、細胞質の固体化は休眠状態の一側面であり、細胞周期停止・代謝再編成・オートファジー誘導など休眠プログラムの諸要素と密接に連動している。【休眠=エネルギー節約モード】に入った細胞では、エネルギー浪費につながる不要な酵素活性や物質拡散は抑制される必要があるが、固体化した細胞質はまさに\textbf{細胞内部の動的プロセスを低ギアに落とす物理的状態}といえる\href{https://pmc.ncbi.nlm.nih.gov/articles/PMC4850707/\#:~:text=it\%20to\%20be\%20associated\%20with,fluid\%20that\%20can\%20reversibly\%20transition}{pmc.ncbi.nlm.nih.gov}\href{https://pmc.ncbi.nlm.nih.gov/articles/PMC4850707/\#:~:text=cytoplasm\%20to\%20a\%20solid,like\%20state}{pmc.ncbi.nlm.nih.gov}。一方、必要なシグナル伝達(例えば栄養復帰の情報)は小分子の拡散によって速やかに全細胞へ行き渡る余地を残している\href{https://pmc.ncbi.nlm.nih.gov/articles/PMC11214080/\#:~:text=spores\%20and\%20uncovered\%20signaling\%20pathways,such\%20as\%20ribosomes\%2C\%20is\%20restricted}{pmc.ncbi.nlm.nih.gov}。このような\textbf{巧みな動静分離}が、休眠細胞において成立している固体化現象の本質である。

\subsection{3. 固体化の適応的意義}

細胞質固体化は、一見すると細胞が「硬直」してしまう不都合な状態のようにも思える。しかし近年の研究から、これは生物が\textbf{ストレス環境を乗り切るための積極的な適応戦略}であることが明らかになってきた。すなわち、固体化した細胞質は休眠細胞に\textbf{多大な利点}をもたらし、将来的な生存と増殖再開の可能性を高めている。

第一に、細胞質を固体化することで\textbf{細胞内部の秩序が維持される}点が挙げられる。通常、細胞質は高度に混み合った粘弾性の流体であり、拡散やモーター輸送によって常に構成要素が動的に再配置されている\href{https://pmc.ncbi.nlm.nih.gov/articles/PMC6857596/\#:~:text=N\%C3\%B8rrelykke\%20et\%20al,to\%20environmental\%20changes\%20requires\%20a}{pmc.ncbi.nlm.nih.gov}。しかしエネルギー供給が途絶えた状況では、能動的な輸送が停止し受動的拡散が支配的になるため、細胞内部の秩序が失われやすい。固体化はこの問題に対処する方法であり、\textbf{細胞質の構造を一種の固定化状態にしておくことで重要な細胞内配置を保つ}役割を果たす\href{https://pmc.ncbi.nlm.nih.gov/articles/PMC6857596/\#:~:text=provide\%20experimental\%20evidence\%20that\%2C\%20in,cellular\%20architecture\%20in\%20dormant\%20cells}{pmc.ncbi.nlm.nih.gov}。実際、分裂酵母のCF細胞では細胞骨格が消失してなお各オルガネラが極端に偏位したり崩壊したりしないことが確認されており、細胞質自体が「構造体」として細胞内容の配置を保持している\href{https://pmc.ncbi.nlm.nih.gov/articles/PMC6857596/\#:~:text=spherical\%20shape,cells\%2C\%20the\%20majority\%20of\%20CF}{pmc.ncbi.nlm.nih.gov}\href{https://pmc.ncbi.nlm.nih.gov/articles/PMC6857596/\#:~:text=spherical\%20shape\%2C\%20CF\%20cells\%20slipped,Fig}{pmc.ncbi.nlm.nih.gov}。この構造維持は、長期休眠からの復帰時に迅速に正常機能を再開するために重要である。いわば**「細胞質の記憶維持」**とでも言うべき機能であり、固体化した細胞質は将来の環境好転に備えて細胞内配置を保存していると考えられる\href{https://pmc.ncbi.nlm.nih.gov/articles/PMC6857596/\#:~:text=solidification\%20of\%20yeast\%20cells\%20described,cellular\%20architecture\%20in\%20dormant\%20cells}{pmc.ncbi.nlm.nih.gov}。

第二に、固体化した細胞質は\textbf{外的ストレスから細胞構成要素を保護}する働きがある。ガラス様に固まった環境では、大きな構造体が動かない分、たとえば機械的衝撃や浸透圧変化に対して内部構造が耐性を示す可能性がある。またタンパク質の拡散や反応が抑制されるため、ダメージを受けた分子が不用意に他の分子と相互作用して\textbf{凝集体を形成するリスク}も減少する\href{https://www.molbiolcell.org/doi/10.1091/mbc.E20-02-0125\#:~:text=induces\%20a\%20massive\%20reorganization\%20of,a\%20\%E2\%80\%9Csolidification\%E2\%80\%9D\%20of\%20the\%20cytoplasm}{molbiolcell.org}\href{https://www.molbiolcell.org/doi/10.1091/mbc.E20-02-0125\#:~:text=All\%20these\%20cellular\%20rearrangements\%20result,favorable\%20environmental\%20conditions\%20are\%20restored}{molbiolcell.org}。Munderらの研究では、エネルギー不足で固体化した細胞質状態への転移は\textbf{飢餓下での生存に必須}であることが示されている\href{https://pmc.ncbi.nlm.nih.gov/articles/PMC4850707/\#:~:text=and\%20foreign\%20tracer\%20particles,fluid\%20that\%20can\%20reversibly\%20transition}{pmc.ncbi.nlm.nih.gov}\href{https://pmc.ncbi.nlm.nih.gov/articles/PMC4850707/\#:~:text=widespread\%20macromolecular\%20assembly\%20of\%20proteins,like\%20state}{pmc.ncbi.nlm.nih.gov}。これは、この状態に入れない細胞はストレス障害によって死滅しやすいことを意味する。実際、多くの生物で見られる休眠戦略(細菌の胞子形成、種子の乾燥耐性など)では、細胞内水分を減らしガラス状の細胞質を作ることで酵素失活や有害な化学反応を防いでいる\href{https://pmc.ncbi.nlm.nih.gov/articles/PMC6857596/\#:~:text=organisation,high\%20amounts\%20of\%20carbohydrates\%2C\%20possibly}{pmc.ncbi.nlm.nih.gov}。分裂酵母の休眠においても、固体化は\textbf{細胞を休眠傷害(長期間の停止による構造劣化)から守る防御策}と位置づけられる。

第三に、固体化によって\textbf{エネルギー消費が極限まで削減}される点も適応上有利である。細胞質が流動的なままでは、大型複合体が拡散してしまい、それを元の適切な位置に保持・戻すためにエネルギーが必要となる。また酵素基質の拡散や不要な反応が進むことで無駄な代謝が起こり得る。固体化した環境では拡散が強く制限されるため、\textbf{細胞内化学反応も停止に近い状態}になる\href{https://pmc.ncbi.nlm.nih.gov/articles/PMC4811765/\#:~:text=undertakes\%20a\%20startling\%20transition\%20upon,which\%20cells\%20globally\%20alter\%20their}{pmc.ncbi.nlm.nih.gov}。Joynerらの研究によれば、グルコース飢餓時の酵母細胞ではクロマチンやmRNP顆粒の運動が著しく制限されるものの、この制限はATP枯渇やpH低下だけでは説明できず、細胞が能動的に\textbf{拡散係数を調節}している結果である\href{https://pmc.ncbi.nlm.nih.gov/articles/PMC4811765/\#:~:text=undertakes\%20a\%20startling\%20transition\%20upon,which\%20cells\%20globally\%20alter\%20their}{pmc.ncbi.nlm.nih.gov}。言い換えれば、細胞は固体化という物理状態に自らを置くことで\textbf{低エネルギー消費のホームステasis}を確立している\href{https://pmc.ncbi.nlm.nih.gov/articles/PMC4811765/\#:~:text=undertakes\%20a\%20startling\%20transition\%20upon,which\%20cells\%20globally\%20alter\%20their}{pmc.ncbi.nlm.nih.gov}\href{https://pmc.ncbi.nlm.nih.gov/articles/PMC4811765/\#:~:text=mobility\%20is\%20induced\%20by\%20a,a\%20unique\%20homeostasis\%20during\%20starvation}{pmc.ncbi.nlm.nih.gov}。これは乾眠中の種子や休眠芽胞がほとんど代謝を行わず長期間生存できるのと同じ理屈であり、分裂酵母においても固体化によって\textbf{生存可能時間を飛躍的に延長}していると考えられる。

さらに固体化状態は、環境が好転した際には\textbf{迅速な復帰}を可能にする準備段階でもある。固体化した細胞質は上述のように内部構造を保存しているため、栄養再添加などの刺激に応答してすみやかに解凍(流動化)し、細胞小器官や酵素が再び活動を開始できる。\href{https://www.molbiolcell.org/doi/10.1091/mbc.E20-02-0125\#:~:text=All\%20these\%20cellular\%20rearrangements\%20result,favorable\%20environmental\%20conditions\%20are\%20restored}{molbiolcell.org}実際、分裂酵母のCF細胞にグルコースを再添加すると、ごく短時間でリボソームなど大きな粒子の可動性が回復し、細胞周期が再開することが観察されている\href{https://pmc.ncbi.nlm.nih.gov/articles/PMC11214080/\#:~:text=cytoplasmic\%20properties,the\%20spore\%20cytoplasm\%20impedes\%20the}{pmc.ncbi.nlm.nih.gov}。胞子の場合も、cAMP-PKA経路の活性化によりトレハロースが分解されると\textbf{瞬時に細胞質粘度が低下}し、種々の代謝が再開する\href{https://pmc.ncbi.nlm.nih.gov/articles/PMC11214080/\#:~:text=cytoplasmic\%20properties,of\%20diffusion\%20coefficients\%20with\%20tracer}{pmc.ncbi.nlm.nih.gov}。このように固体化は可逆的かつ迅速に解除可能な状態であり、\textbf{休眠からの復活をスムーズにする待機戦略}とみなせる。

以上の点から、細胞質固体化には\textbf{ストレス耐性と生存戦略上の意義}が認められる。それは単なる受動的な現象ではなく、細胞が自らの生存を守り将来の増殖機会に備えるために獲得した\textbf{適応的な休眠戦略}なのである\href{https://pmc.ncbi.nlm.nih.gov/articles/PMC4850707/\#:~:text=widespread\%20macromolecular\%20assembly\%20of\%20proteins,like\%20state}{pmc.ncbi.nlm.nih.gov}。この視点は、細胞質を「環境に応じて可逆的に相転移させうる動的システム」として捉え直す契機ともなっており、細胞質固体化現象は生物が厳しい環境を乗り越えるための巧妙な\textbf{可塑性メカニズム}と位置付けられる\href{https://pmc.ncbi.nlm.nih.gov/articles/PMC4850707/\#:~:text=cytoplasm\%20to\%20a\%20solid,like\%20state}{pmc.ncbi.nlm.nih.gov}。

\subsection{4. 他のストレス条件における固体化メカニズム:共通点と相違点}

細胞質固体化現象は、グルコース飢餓以外のストレス条件においても報告されており、基本原理に共通性が見られる場合と、ストレス特有の相違点が見られる場合がある。ここでは、窒素飢餓や酸化ストレスなど他の代表的ストレスにおける細胞質状態変化を、グルコース飢餓の場合と比較する。

\textbf{(a)窒素飢餓の場合:}

 窒素源の枯渇(例:アンモニウムやアミノ酸の欠乏)は、炭素飢餓と並ぶ一般的な飢餓ストレスである。分裂酵母では、窒素飢餓に陥るとまず接合と胞子形成の分化経路に入り、8–12時間程度で胞子が形成される。しかし相手株がいない場合や変異株では胞子形成ができず、そのまま\textbf{無分化の休眠(G0)状態}へ移行する\href{https://pmc.ncbi.nlm.nih.gov/articles/PMC6857596/\#:~:text=When\%20nutrients\%20become\%20growth\%20limiting,yeast\%20cells\%20that\%20have\%20slowly}{pmc.ncbi.nlm.nih.gov}。この窒素飢餓由来の休眠細胞にも、細胞質固体化と類似の現象が起こると考えられるが、その様態は炭素飢餓の場合といくつか異なる可能性がある。まず、炭素は存在するため細胞は呼吸や発酵を続けられ、一部エネルギー産生は維持される。このため\textbf{細胞質pHの急激な低下}や\textbf{ATP枯渇}は炭素飢餓ほど顕著ではなく、固体化の進行も緩やかであると予想される。また、窒素飢餓では炭素代謝産物(エタノールなど)の蓄積や、炭素過剰に起因する細胞内酸化ストレスも生じうる。Yanagidaらの研究によれば、窒素飢餓時の分裂酵母は、グルコース飢餓時とは異なる転写応答や代謝プロファイルを示すことが報告されており\href{https://www.researchgate.net/publication/262226866_Does_a_shift_to_limited_glucose_activate_checkpoint_control_in_fission_yeast\#:~:text=,1995\%2C\%20Takeda\%20et}{researchgate.net}、休眠に至る過程で\textbf{異なる経路のチェックポイント}が働いている可能性がある。実際、窒素飢餓で休眠させた分裂酵母では細胞骨格の再編成(極性の消失や安定な微小管束の形成)が起こるが\href{https://rupress.org/jcb/article/210/1/99/38063/A-stable-microtubule-array-drives-fission-yeast\#:~:text=The\%20fission\%20yeast\%20cytoskeleton\%20is,is\%20drastically\%20reshaped\%20as}{rupress.org}、それ以上の長期間にわたる細胞質粘度変化についてはまだ詳細な解析が不足している。炭素飢餓で見られたような\textbf{深い固体化状態(CF)\textbf{に窒素飢餓細胞も達しうるのか、あるいは窒素飢餓ではそこまで劇的な固化は起こらず準安定的な「ゲル様状態」に留まるのかは、今後の検討課題である。とはいえ、窒素飢餓下の出芽酵母でもプロテアソームや酵素の会合体形成など基本的な応答は共有されていることから\href{https://pmc.ncbi.nlm.nih.gov/articles/PMC6857596/\#:~:text=shown\%20to\%20introduce\%20structural\%20changes,2016}{pmc.ncbi.nlm.nih.gov}、}「水分減少・分子混雑増大によるガラス化」\textbf{や}「高分子会合体ネットワークの形成」\textbf{といった固体化メカニズムの基本骨子は多かれ少なかれ共通していると考えられる\href{https://pmc.ncbi.nlm.nih.gov/articles/PMC6857596/\#:~:text=organisation,high\%20amounts\%20of\%20carbohydrates\%2C\%20possibly}{pmc.ncbi.nlm.nih.gov}\href{https://pmc.ncbi.nlm.nih.gov/articles/PMC6857596/\#:~:text=shown\%20to\%20introduce\%20structural\%20changes,2016}{pmc.ncbi.nlm.nih.gov}。要約すれば、窒素飢餓では}固体化現象は緩徐的かつ部分的}に進行し、炭素飢餓ほど極端ではないものの、細胞質の粘度上昇・動態抑制を伴う傾向があると言える。

\textbf{(b)その他の栄養飢餓(リン酸・硫黄など):}

 炭素・窒素以外の必須栄養素の欠乏も細胞を休眠に導く。例えば無機リン酸の欠乏は酵母にG0遷移を誘導し、代謝や遺伝子発現の大きな変化を伴う\href{https://journals.asm.org/doi/10.1128/mbio.00241-25\#:~:text=Fission\%20yeast\%20metabolome\%20dynamics\%20during,a\%20state\%20of\%20G0\%20quiescence}{journals.asm.org}。長期のリン酸飢餓(2日以上)では、分裂酵母で遺伝子発現パターンが大きく変化し、オートファジーや蓄積炭水化物(グリコーゲンなど)の動員が起こる\href{https://academic.oup.com/nar/article/51/7/3094/7041946\#:~:text=Cellular\%20responses\%20to\%20long,expression\%20whereby\%20the\%20mRNAs}{academic.oup.com}。リン酸飢餓細胞でも細胞質が固めの状態になることが推察され、実際に細胞体積の減少や高分子凝集体の形成が確認されているとの報告もある(未出版データ含む)。硫黄やその他ミネラル飢餓については詳細な細胞質物性の研究例は少ないが、一般に栄養飢餓全般で共通するのは\textbf{細胞増殖停止}・\textbf{代謝ダウン調節}・\textbf{ストレス応答経路活性化}である\href{https://pmc.ncbi.nlm.nih.gov/articles/PMC3123465/\#:~:text=,0\%E2\%80\%93111\%20mm}{pmc.ncbi.nlm.nih.gov}。したがって細胞質固体化は、栄養素の種別にかかわらず、\textbf{エネルギー不足に陥った細胞がとりうる普遍的な適応現象}と位置づけることができる。

\textbf{(c)酸化ストレスの場合:}

 過酸化水素(H₂O₂)や重金属、放射線などによる酸化的ストレスは、栄養飢餓とは異なる種類の細胞ダメージを与える。しかし、重度の酸化ストレス下では細胞内のエネルギー産生系(呼吸鎖や酵素)が損傷を受け、結果的に\textbf{細胞質のATP枯渇やpH異常}を招く点で共通する。実際、熱ストレスや酸化ストレスに晒された酵母細胞でも、一時的に細胞質pHが低下しタンパク質の可溶性が変化する報告がある\href{https://www.biorxiv.org/content/10.1101/2022.07.29.502016v1.full\#:~:text=Exit\%20of\%20spore\%20dormancy\%20transforms,including\%20during\%20heat}{biorxiv.org}。Munderらは、熱ストレス時にも細胞質の酸性化と硬直化が起こりうることを示唆している\href{https://www.biorxiv.org/content/10.1101/2022.07.29.502016v1.full\#:~:text=Exit\%20of\%20spore\%20dormancy\%20transforms,including\%20during\%20heat}{biorxiv.org}。また酸化ストレス下では\textbf{ストレス顆粒(SG)\textbf{や}Pボディ}といったRNP凝集体が細胞質に出現することが知られ、これは細胞が翻訳を抑制しつつmRNAを蓄えておく戦略である。これらの顆粒形成は典型的な\textbf{液-液相分離現象}であり、細胞質の微視的領域で相分離が起こって局所的に高濃度フェーズが生じる\href{https://www.molbiolcell.org/doi/10.1091/mbc.E20-02-0125\#:~:text=Reorganization\%20of\%20budding\%20yeast\%20cytoplasm,de\%20novo\%20formation\%20of}{molbiolcell.org}。酸化ストレスによるSG形成は可逆的で、ストレス解除後には元の拡散状態に戻るが、その間細胞質は部分的に「ゲル化」しているとも見做せる。したがって酸化ストレス条件下でも、栄養飢餓時と\textbf{共通の要素}(エネルギー低下による拡散制限、タンパク質会合体の形成)と\textbf{相違点}(ストレス特有の誘導体経路や損傷タイプ)があるものの、**広義の細胞質固体化(流動性低下)**は生じるといえる。

\textbf{(d)胞子形成・乾燥ストレスの場合:}

 極限的なストレス環境として、胞子形成(乾燥耐性)や極低温・脱水などが挙げられる。細菌芽胞や酵母胞子、植物種子では、細胞内水分が大幅に減少し固体に近い状態になることが古くから知られている\href{https://pmc.ncbi.nlm.nih.gov/articles/PMC2780810/\#:~:text=Whether\%20glass,17}{pmc.ncbi.nlm.nih.gov}\href{https://www.sciencedirect.com/topics/agricultural-and-biological-sciences/bacterial-spore\#:~:text=However\%2C\%20according\%20to\%20the\%20second\%2C,like\%20solid}{sciencedirect.com}。胞子の細胞質は\textbf{ガラス様の固い状態}で、細胞質含有物の移動はほぼ停止している\href{https://pmc.ncbi.nlm.nih.gov/articles/PMC11214080/\#:~:text=spores\%20and\%20uncovered\%20signaling\%20pathways,such\%20as\%20ribosomes\%2C\%20is\%20restricted}{pmc.ncbi.nlm.nih.gov}。この状態は胞子の高い耐久性(耐熱性・耐乾燥性)に寄与しており、水分がごく僅かしかないために化学反応も拡散も極度に抑制されている\href{https://pmc.ncbi.nlm.nih.gov/articles/PMC2780810/\#:~:text=Whether\%20glass,17}{pmc.ncbi.nlm.nih.gov}。分裂酵母の場合も、栄養飢餓で形成される\textbf{胞子}は休眠細胞の一種であり、その細胞質は親細胞の休眠状態以上に強固に固化していると考えられる。実際、胞子の細胞質は\textbf{高密度かつ高粘度}で、内部pHも酸性に偏っているとの報告がある\href{https://pmc.ncbi.nlm.nih.gov/articles/PMC10118125/\#:~:text=Breaking\%20spore\%20dormancy\%20in\%20budding,by\%20the\%20neutralization\%20and}{pmc.ncbi.nlm.nih.gov}。胞子発芽時には上述のようにトレハロース分解による急激な流動化が起こり\href{https://pmc.ncbi.nlm.nih.gov/articles/PMC11214080/\#:~:text=cytoplasmic\%20properties,of\%20diffusion\%20coefficients\%20with\%20tracer}{pmc.ncbi.nlm.nih.gov}、固体化状態が解除される。胞子や種子における細胞質固体化(細胞質ガラス化)は、まさに栄養飢餓だけでなく\textbf{乾燥そのものに対する適応}であり、生物が環境変化に応じて細胞質の相状態を変化させうることを示す極端な例と言える\href{https://pmc.ncbi.nlm.nih.gov/articles/PMC6857596/\#:~:text=organisation,high\%20amounts\%20of\%20carbohydrates\%2C\%20possibly}{pmc.ncbi.nlm.nih.gov}。このように、固体化現象は多様なストレス状況下で見られる\textbf{汎適応戦略}であり、その具体的メカニズムや程度はストレス種別によって変わるものの、本質的には「エネルギーや水分の不足に応じて細胞質の物理化学的性質を調節する」という\textbf{共通原理}に支えられている\href{https://pmc.ncbi.nlm.nih.gov/articles/PMC6857596/\#:~:text=organisation,high\%20amounts\%20of\%20carbohydrates\%2C\%20possibly}{pmc.ncbi.nlm.nih.gov}。

\subsection{5. 液-液相分離(LLPS)との関連性}

細胞質固体化現象を分子的視点から捉えると、しばしば\textbf{液-液相分離(Liquid-Liquid Phase Separation, LLPS)\textbf{との関係が議論される。LLPSとは、細胞質や核内で膜に囲まれないタンパク質や核酸の集合体(凝縮体)が可逆的に形成される過程であり、液滴のような独立相を作る現象である。近年、生理条件下で様々なタンパク質凝集体(ストレス顆粒、Pボディ、シグナルクラスターなど)がLLPSによって形成され、細胞機能のコンパートメント化に寄与していることが明らかになってきた。栄養飢餓時にも、多くの酵素やリボソームタンパク質が一時的に会合し凝縮体を作ることが観察されており、これらはLLPSの典型例といえる\href{https://pmc.ncbi.nlm.nih.gov/articles/PMC6857596/\#:~:text=shown\%20to\%20introduce\%20structural\%20changes,2016}{pmc.ncbi.nlm.nih.gov}。例えば、出芽酵母ではグルコース飢餓により解糖系酵素や翻訳因子が大きな顆粒または繊維状構造を形成し、酵素活性を抑制しつつ貯蔵形態となることが報告されている\href{https://pmc.ncbi.nlm.nih.gov/articles/PMC6857596/\#:~:text=shown\%20to\%20introduce\%20structural\%20changes,2016}{pmc.ncbi.nlm.nih.gov}。こうしたタンパク質集合体(}「飢餓応答コンデンサ」\textbf{とも呼べるもの)が互いに相互作用・連結することで、細胞質全体が}高分子ネットワーク化}し、巨視的には固体様の挙動を示すようになる\href{https://pmc.ncbi.nlm.nih.gov/articles/PMC4850707/\#:~:text=match\%20at\%20L1536\%20agreement\%20with,have\%20to\%20determine\%20the\%20molecular}{pmc.ncbi.nlm.nih.gov}。Munderらは、酸性化した飢餓細胞質中で\textbf{多数のタンパク質が高密度ネットワークを形成し、それが全体としてガラス的な物理特性をもつ}とのモデルを提唱している\href{https://pmc.ncbi.nlm.nih.gov/articles/PMC4850707/\#:~:text=match\%20at\%20L1536\%20agreement\%20with,have\%20to\%20determine\%20the\%20molecular}{pmc.ncbi.nlm.nih.gov}。このモデルは、LLPSによる各種凝集体がさらに高次に集積・網絡化して\textbf{一種のゲル状態}を作り出すイメージに相当する。

分裂酵母の細胞質固体化においても、LLPSは重要な役割を果たすと考えられる。実際に観察される現象(例えば\textbf{アクチン束の形成}、\textbf{脂質滴のクラスター化}、\textbf{eIF2B酵素のフィラメント化}など\href{https://www.molbiolcell.org/doi/10.1091/mbc.E20-02-0125\#:~:text=induces\%20a\%20massive\%20reorganization\%20of,a\%20\%E2\%80\%9Csolidification\%E2\%80\%9D\%20of\%20the\%20cytoplasm}{molbiolcell.org})は、いずれも分子レベルではタンパク質や脂質の\textbf{相分離・凝集}によって説明できる。特に顕著なのは、オルガネラに属さない\textbf{酵素フィラメント}の存在である。先述のように出芽酵母では翻訳因子eIF2Bが飢餓時に巨大なフィラメント束(通称「ヴォズ(Voz)粒子」など)を形成する\href{https://www.molbiolcell.org/doi/10.1091/mbc.E20-02-0125\#:~:text=induces\%20a\%20massive\%20reorganization\%20of,protective\%20mechanism\%20for\%20molecular\%20components}{molbiolcell.org}。分裂酵母でも類似の凝縮現象が起これば、細胞質中に\textbf{広範な足場}が出現し、他の分子の拡散を妨げるであろう。また\textbf{リボソームサブユニット}や\textbf{翻訳前複合体}は飢餓時にストレス顆粒へと移行する可能性があり、それもLLPSの一種である。Narayanaswamyら(2009)の包括的研究では、酵母で栄養ストレス時に数百種類ものタンパク質が新たな顆粒や点状構造に局在することが示されており\href{https://pmc.ncbi.nlm.nih.gov/articles/PMC6857596/\#:~:text=shown\%20to\%20introduce\%20structural\%20changes,2016}{pmc.ncbi.nlm.nih.gov}、これらは後に膜のないオルガネラとして再評価されている\href{https://elifesciences.org/articles/02409\#:~:text=Filament\%20formation\%20by\%20metabolic\%20enzymes,Gln1\%20has}{elifesciences.org}\href{https://www.molbiolcell.org/doi/10.1091/mbc.E20-02-0125\#:~:text=Reorganization\%20of\%20budding\%20yeast\%20cytoplasm,de\%20novo\%20formation\%20of}{molbiolcell.org}。これらの知見は、栄養飢餓に応答した\textbf{細胞質内の相分離現象が普遍的かつ大量に起こっている}ことを示唆している。

では、LLPSによる凝縮体形成と細胞質固体化はどのように関連付けられるのか。鍵となるのは\textbf{濃度閾値}と\textbf{時間スケール}である。LLPSは特定のタンパク質濃度や条件下で局所的に発生する現象だが、細胞質固体化のような全体現象になるには、複数の相分離が連動して\textbf{全細胞的なネットワーク or メッシュ}を形成する必要がある。これは物理的にはコロイド溶液がゲル化・ガラス化する過程に類似しており、局所的相分離がやがて連結して\textbf{巨大な不均一クラスター}を作り出すことで起こる\href{https://pmc.ncbi.nlm.nih.gov/articles/PMC4850707/\#:~:text=match\%20at\%20L1536\%20agreement\%20with,have\%20to\%20determine\%20the\%20molecular}{pmc.ncbi.nlm.nih.gov}。飢餓で誘導される多数のタンパク質凝縮体(酵素集合体、RNP粒、ストレス顆粒など)は、互いに独立ではなく何らかのクロストーク(たとえば競合的な水分・溶媒の取り合い、共有成分の存在など)によって\textbf{一斉に出現・成長}する可能性がある。その結果、細胞質マトリックス全体が充填されるほど凝縮体が増えれば、もはや液体的挙動は見られなくなり\textbf{ガラス転移}に至る\href{https://pmc.ncbi.nlm.nih.gov/articles/PMC6857596/\#:~:text=N\%C3\%B8rrelykke\%20et\%20al,to\%20environmental\%20changes\%20requires\%20a}{pmc.ncbi.nlm.nih.gov}\href{https://pmc.ncbi.nlm.nih.gov/articles/PMC6857596/\#:~:text=a\%20gel,N\%C3\%B8rrelykke\%20et}{pmc.ncbi.nlm.nih.gov}。言い換えれば、LLPSは細胞質固体化の\textbf{ミクロな原動力}であり、固体化はLLPSの\textbf{マクロな帰結}であるとまとめることができる。

ただし、細胞質固体化が常にLLPSだけで説明できるわけではない点にも注意が必要である。固体化には、単なる「液滴が集まった状態」を超えて\textbf{ガラス的(非平衡)な振る舞い}がある。LLPSで形成された液滴は一般に可逆的かつ動的平衡状態にあり、液体として融合・分裂もする。しかし飢餓深部で観察される固体化細胞質では、タンパク質集合体同士の融合はほとんど起こらず、\textbf{非可逆的なガラス固化}に近い様相を呈する\href{https://pmc.ncbi.nlm.nih.gov/articles/PMC6857596/\#:~:text=run\%20out\%20of\%20nutrients,2016\%20\%3B\%20\%2043}{pmc.ncbi.nlm.nih.gov}\href{https://pmc.ncbi.nlm.nih.gov/articles/PMC6857596/\#:~:text=virtually\%20no\%20motion\%2C\%20while\%20molecules,2016\%20\%3B\%20\%2043}{pmc.ncbi.nlm.nih.gov}。これは、相分離でできた液滴がさらに内部で乾燥・硬化したり、追加の共有結合や架橋(例えば酸化的架橋や異種タンパク間相互作用ネットワーク)が進行した結果かもしれない。したがって、LLPSは固体化の初期段階を説明し、**その先の成熟(hardening)**については追加の分子機構(例えばトレハロースガラスの形成や分子シャペロンによる凝集抑制の喪失など)を仮定する必要があるだろう。

総合すると、液-液相分離は栄養飢餓ストレス下で広く起こる現象であり、\textbf{細胞質固体化の分子的基盤}を提供する。多数のタンパク質や高分子が一斉に可逆凝集することで細胞質の動的粘度が上昇し、さらに長時間経過に伴って凝集体同士が協調し\textbf{細胞質全域にわたる固化ネットワーク}を形成する\href{https://pmc.ncbi.nlm.nih.gov/articles/PMC4850707/\#:~:text=match\%20at\%20L1536\%20agreement\%20with,have\%20to\%20determine\%20the\%20molecular}{pmc.ncbi.nlm.nih.gov}。この視点から、細胞質固体化現象は\textbf{生物が自らの細胞内容物をLLPSにより再編成し、高分子密度相へと誘導する戦略}と捉えることができ、平衡状態でのLLPS研究と非平衡状態での休眠細胞研究を結びつける重要な概念となっている



\section{細胞内の高分子混雑環境と生化学反応に与える影響}


人工細胞(合成細胞または細胞模倣区画とも呼ばれる)は、生細胞の特定の機能を模倣するように設計された最小限のシステムである。天然細胞であれ人工細胞であれ、あらゆる細胞の重要な側面は、生化学反応が起こる\textbf{細胞質環境}である。生細胞において、細胞質は高分子(タンパク質、核酸など)が密集した溶液であり、細胞体積の約20~40\%を占める\href{https://bionumbers.hms.harvard.edu/bionumber.aspx?s=n&v=12&id=105814\#:~:text=volume\%20for\%20excluded\%20volume\%20interactions\%2C,range\%20wasn\%27t\%20located\%20in\%20Kao}{bionumbers.hms.harvard.edu}\href{https://www.nature.com/articles/s41598-017-14883-y?error=cookies_not_supported&code=b0600d9e-c852-48ed-ab9c-7526a5c6893b\#:~:text=glassy\%20behaviors8\%20\%2C\%2033\%2C10\%20\%2C,driven}{nature.com}。この高い\textbf{細胞質密度}(通常、高分子の濃度が約0.2~0.3 g/mL)は\textit{高分子クラウディング}として知られている。これは\textbf{粘度}などの物理的特性や、酵素反応速度、タンパク質フォールディング、分子拡散などの生物学的プロセスに深刻な影響を及ぼす\href{https://www.nature.com/articles/nrmicro.2017.17?error=cookies_not_supported&code=87b4b2eb-5370-49be-862f-ed6bad1e1b44\#:~:text=Macromolecular\%20crowding\%20affects\%20the\%20mobility,we\%20propose\%20that\%20the\%20term}{nature.com}。人工細胞が\textit{反応容器}として機能するためには、この密集した内部環境を再現しつつ(流動性のある液体状態を維持しながら)することが不可欠である。本総説では、細胞質密度と粘度の定量的関係、クラウディングによる粘度変化が生化学反応や拡散に及ぼす影響、最小限の流動性を保つ反応環境を維持するために必要な「閾値」細胞質密度について検討する。さらに人工細胞系における細胞質粘度の測定・制御手法についても論じる。

\subsection{細胞質密度と粘度の定量的関係}

\textbf{粘度は高分子密度に非線形的に増加する。}希薄溶液では粘度上昇は緩やかである。例えばGFPのような小分子溶質の場合、生細胞細胞質の有効粘度は水の約3倍に過ぎない\href{https://bionumbers.hms.harvard.edu/bionumber.aspx?id=102561&ver=2\#:~:text=Value\%203,S65T\%20in}{bionumbers.hms.harvard.edu}。(絶対値で言えば、水の粘度が約1cPであるのに対し、小分子タンパク質に対する細胞質粘度は約3cP程度である。)しかし高分子濃度が上昇すると、粘度は劇的かつ非線形的に増加する。経験的研究やモデルでは、細胞質は「ジャミング」点に近づくコロイド懸濁液として扱われる。\textit{in vitro}モデル細胞質(例:精製タンパク質溶液や代謝のない細胞溶解液)において、研究者は濃度が臨界値を超えると\textbf{急峻な超アレニウス上昇}を示すことを観察している\href{https://www.nature.com/articles/s41598-017-14883-y?error=cookies_not_supported&code=b0600d9e-c852-48ed-ab9c-7526a5c6893b\#:~:text=materials\%20rapidly\%20increased\%20with\%20the,0.3\%20g\%2FmL}{nature.com}\href{https://www.nature.com/articles/s41598-017-14883-y?error=cookies_not_supported&code=b0600d9e-c852-48ed-ab9c-7526a5c6893b\#:~:text=colloidal\%20suspensions\%20close\%20to\%20their,0.3\%20g\%2FmL}{nature.com}。具体的には、西澤\textit{ら}による実験では、様々な細胞質抽出物およびタンパク質溶液のゼロせん断粘度を測定した。全てが臨界濃度\textit{c}\_* ≈* 0.3–0.34 g/mL(30–34\% w/v)\href{https://www.nature.com/articles/s41598-017-14883-y?error=cookies_not_supported&code=b0600d9e-c852-48ed-ab9c-7526a5c6893b\#:~:text=colloidal\%20suspensions\%20close\%20to\%20their,0.3\%20g\%2FmL}{nature.com}において急激な粘度発散を示した。この濃度は生細胞内の生理的巨分子密度(約0.3 g/mL)\href{https://www.nature.com/articles/s41598-017-14883-y?error=cookies_not_supported&code=b0600d9e-c852-48ed-ab9c-7526a5c6893b\#:~:text=colloidal\%20suspensions\%20close\%20to\%20their,0.3\%20g\%2FmL}{nature.com}に驚くほど近い。つまり、活性プロセスを伴わない平衡状態の細胞質は、実細胞内部と同程度の密度で実質的に固体(「ガラス状」)化する。この閾値を超えると、理論上粘度は無限大(非流動的なジャミング状態)に近づく\href{https://www.nature.com/articles/s41598-017-14883-y?error=cookies_not_supported&code=b0600d9e-c852-48ed-ab9c-7526a5c6893b\#:~:text=cancer\%20cells,0.3\%20g\%2FmL}{nature.com}\href{https://www.nature.com/articles/s41598-017-14883-y?error=cookies_not_supported&code=b0600d9e-c852-48ed-ab9c-7526a5c6893b\#:~:text=materials\%20rapidly\%20increased\%20with\%20the,0.3\%20g\%2FmL}{nature.com}。実際、単純な球状タンパク質溶液(緩衝液中のBSA)に対するコロイドガラスモデルへの外挿では、粘度が約0.78 g/mL(体積分率約60\%)で発散すると予測される\href{https://www.nature.com/articles/s41598-017-14883-y?error=cookies_not_supported&code=b0600d9e-c852-48ed-ab9c-7526a5c6893b\#:~:text=where\%20A\%20\%3D\%201,crowding\%2C\%20primarily\%20because\%20of\%20repulsive}{nature.com}\href{https://www.nature.com/articles/s41598-017-14883-y?error=cookies_not_supported&code=b0600d9e-c852-48ed-ab9c-7526a5c6893b\#:~:text=concentrations\%20predicted\%20that\%20\%CE\%B7\%20would,constituent\%20protein\%20molecules15\%20\%2C\%2069}{nature.com}。これは球体のランダム密充填が体積分率約60–65\%で達成される概念と一致する。実際の細胞質(高分子が体積比約30\%を占める\href{https://bionumbers.hms.harvard.edu/bionumber.aspx?s=n&v=12&id=105814\#:~:text=volume\%20for\%20excluded\%20volume\%20interactions\%2C,range\%20wasn\%27t\%20located\%20in\%20Kao}{bionumbers.hms.harvard.edu})はこの充填限界のほぼ中間点に位置し、固化はしていないものの著しく高い粘性を示す。定量的には、 細胞抽出物研究では、粘度(水に対する相対値)が約0.2–0.3 g/mLまでは中程度に留まるが、濃度が約0.4–0.5 g/mLに近づくと急激に上昇することが示されている\href{https://www.nature.com/articles/s41598-017-14883-y?error=cookies_not_supported&code=b0600d9e-c852-48ed-ab9c-7526a5c6893b\#:~:text=As\%20shown\%20in\%20Fig,55\%2C39\%20fit\%20the\%20data\%20well}{nature.com}\href{https://www.nature.com/articles/s41598-017-14883-y?error=cookies_not_supported&code=b0600d9e-c852-48ed-ab9c-7526a5c6893b\#:~:text=Figure\%201}{nature.com}。例えば、0.45 g/mLのBSAでは粘度は既にアレニウス則から逸脱し、約0.6 g/mLでは桁違いに増加する\href{https://www.nature.com/articles/s41598-017-14883-y?error=cookies_not_supported&code=b0600d9e-c852-48ed-ab9c-7526a5c6893b\#:~:text=As\%20shown\%20in\%20Fig,55\%2C39\%20fit\%20the\%20data\%20well}{nature.com}\href{https://www.nature.com/articles/s41598-017-14883-y?error=cookies_not_supported&code=b0600d9e-c852-48ed-ab9c-7526a5c6893b\#:~:text=where\%20A\%20\%3D\%201,that\%20the\%20kinetics\%20of\%20the}{nature.com}。これらのデータは\textbf{細胞質粘度が混雑に極めて敏感であること}、特にジャミング領域に近づくほど顕著であることを強調している。

\textbf{生細胞における能動的流動化。}注目すべきは、生細胞が高分子含有量にもかかわらずガラス状凝固を\textit{回避}している点である。代謝活性を持つ細胞内部は固体ではなく粘弾性流体として振る舞う。これは能動的プロセス(例:ATP駆動型細胞骨格再構築、モータータンパク質活性、代謝回転)が\textbf{内部攪拌と流動化}をもたらすためである\href{https://www.nature.com/articles/s41598-017-14883-y?error=cookies_not_supported&code=b0600d9e-c852-48ed-ab9c-7526a5c6893b\#:~:text=These\%20observations\%20indicate\%20that\%20the,is\%20typical\%20for\%20strong\%20glass}{nature.com}\href{https://www.nature.com/articles/s41598-017-14883-y?error=cookies_not_supported&code=b0600d9e-c852-48ed-ab9c-7526a5c6893b\#:~:text=of\%20living\%20cells\%20was\%20lost,can\%20be\%20attained\%20in\%20activity}{nature.com}。研究によれば、\textbf{細胞の代謝を停止させる}(ATP枯渇や低温など)と、細胞質は流体様から固体様への転移を起こし得る。Parry \textit{et al.} (2014) は、細菌の細胞質がガラス形成挙動を示し、代謝活動が停止すると固体ゲル状態へ移行するが、代謝が活発な動きで「刺激」を与えると流動性を維持することを観察した\href{https://www.nature.com/articles/s41598-017-14883-y?error=cookies_not_supported&code=b0600d9e-c852-48ed-ab9c-7526a5c6893b\#:~:text=These\%20observations\%20indicate\%20that\%20the,is\%20typical\%20for\%20strong\%20glass}{nature.com}\href{https://www.nature.com/articles/s41598-017-14883-y?error=cookies_not_supported&code=b0600d9e-c852-48ed-ab9c-7526a5c6893b\#:~:text=of\%20living\%20cells\%20was\%20lost,the\%20loss\%20of\%20fragility\%2C\%20in}{nature.com}。同様に、西澤\textit{ら}は\textbf{不活性細胞質が生理的濃度で「凍結」する}(約0.3 g/mLで流動性を喪失)一方、\textbf{活性細胞質は同濃度でも流動性を維持する}ことを発見した\href{https://www.nature.com/articles/s41598-017-14883-y?error=cookies_not_supported&code=b0600d9e-c852-48ed-ab9c-7526a5c6893b\#:~:text=cancer\%20cells,0.3\%20g\%2FmL}{nature.com}。生細胞では粘度対濃度曲線が実質的にシフトする:30\% w/v付近での超指数関数的急上昇ではなく、より広い範囲で粘度が緩やかに増加する(ほぼアレニウス則的)\href{https://www.nature.com/articles/s41598-017-14883-y?error=cookies_not_supported&code=b0600d9e-c852-48ed-ab9c-7526a5c6893b\#:~:text=of\%20living\%20cells\%20was\%20lost,the\%20loss\%20of\%20fragility\%2C\%20in}{nature.com}。これは、細胞内部が(過度に粘性が高くなることなく)高分子濃度下でも機能し続けることを意味する。これは活性的な\textit{混合/緩和}プロセスによるものである\href{https://www.nature.com/articles/s41598-017-14883-y?error=cookies_not_supported&code=b0600d9e-c852-48ed-ab9c-7526a5c6893b\#:~:text=of\%20living\%20cells\%20was\%20lost,the\%20loss\%20of\%20fragility\%2C\%20in}{nature.com}。このような活性的な流動化機構を持たない人工細胞においては、高分子濃度で過負荷をかけないよう注意が必要である。逆に、代謝による流動化効果を模倣するため、何らかの活性要素を導入する必要が生じる可能性もある。

\textbf{細胞質粘度の絶対値}細胞質の「粘度」に単一の数値を割り当てることは困難である。周波数依存性(粘弾性)、スケール依存性、条件依存性を示すためだ。それでも、いくつかの基準値は参考になる。哺乳類細胞における蛍光光漂白実験では、小さなトレーサータンパク質に対して\textbf{細胞質全体の粘度が水の約2~4倍}であることが判明した\href{https://bionumbers.hms.harvard.edu/bionumber.aspx?id=102561&ver=2\#:~:text=Value\%203,S65T\%20in}{bionumbers.hms.harvard.edu}。細菌細胞では、小さな代謝産物の拡散はわずかに遅くなるだけ(多くの場合依然としてブラウン運動)だが、より大きなタンパク質複合体はサブ拡散または数倍高い有効粘度を経験する\href{https://www.nature.com/articles/nrmicro.2017.17?error=cookies_not_supported&code=87b4b2eb-5370-49be-862f-ed6bad1e1b44\#:~:text=Macromolecular\%20crowding\%20affects\%20the\%20mobility,we\%20propose\%20that\%20the\%20term}{nature.com}。細胞ストレス下(例:過浸透状態による密集化、エネルギー枯渇による活性運動減少)では、有効粘度が大幅に増加する可能性がある——時には1桁以上上昇し、分子移動性が著しく制限される。人工系では報告される粘度が様々である: 単純な経験則として、30\%のポリマー/タンパク質凝集剤は、ポリマーの種類や条件に依存するが、粘度を水の10~100倍程度まで上昇させる可能性がある。\textbf{コアセルベート原生細胞}(液-液相分離した液滴で「原始細胞」のモデルとなる)は極端な例を提供する:水分含有量は70~90\%だが、10~30\%のポリマー豊富相は水の\textit{10\^2~10\^3}倍の粘度を示す\href{https://pmc.ncbi.nlm.nih.gov/articles/PMC11256357/\#:~:text=The\%20high\%20macromolecular\%20content\%20of,ribozymes\%20and\%20other\%20large\%20molecules}{pmc.ncbi.nlm.nih.gov}。このようなコアセルベートは依然として液体であるが、非常に粘性が高く、拡散が遅い\textit{シロップ状}の微小環境を形成する\href{https://pmc.ncbi.nlm.nih.gov/articles/PMC11256357/\#:~:text=The\%20high\%20macromolecular\%20content\%20of,ribozymes\%20and\%20other\%20large\%20molecules}{pmc.ncbi.nlm.nih.gov}。これは「密度が高いほど粘性が高い」という一般則を示すと同時に、詳細な組成(例:柔軟なポリマー対球状タンパク質)が重要であることを示している。要約すると、生理的分子量濃度(約30\% w/v)の細胞質様溶液は、活性プロセスが存在しない状態では10 cP(水の約10倍)程度の粘度を示す可能性がある。しかし、生物学的活性状態では、 \textit{見かけの}粘度は、活性な揺らぎが混雑効果を緩和するため、小さな溶質に対して2~5 cPに近い値となる可能性がある\href{https://www.nature.com/articles/nrmicro.2017.17?error=cookies_not_supported&code=87b4b2eb-5370-49be-862f-ed6bad1e1b44\#:~:text=Macromolecular\%20crowding\%20affects\%20the\%20mobility,we\%20propose\%20that\%20the\%20term}{nature.com}\href{https://www.nature.com/articles/s41598-017-14883-y?error=cookies_not_supported&code=b0600d9e-c852-48ed-ab9c-7526a5c6893b\#:~:text=of\%20living\%20cells\%20was\%20lost,the\%20loss\%20of\%20fragility\%2C\%20in}{nature.com}.

\subsection{混雑と粘度が生化学反応に及ぼす影響}

核心的な疑問は、こうした物理的特性(高密度・高粘度)が人工細胞内の\textbf{反応速度論と分子拡散}にどう影響するかである。高密度で粘性の高い細胞質は生化学にとって両刃の剣であり、\textbf{有益な効果と阻害効果}の両面を持つ:

\begin{itemize}
    \item \textbf{分子間相互作用と安定性の増強:}高分子クラウディングは反応物の実効濃度を増加させ、分子間結合を促進することで特定の反応を\textit{加速}する。密集した媒体では、二分子結合速度と平衡状態が複合体形成に有利にシフトする(排除体積によるル・シャトリエの原理)\href{https://www.nature.com/articles/nrmicro.2017.17?error=cookies_not_supported&code=87b4b2eb-5370-49be-862f-ed6bad1e1b44\#:~:text=Macromolecular\%20crowding\%20affects\%20the\%20mobility,we\%20propose\%20that\%20the\%20term}{nature.com}。また密集は、大きな空隙をエントロピー的に排除することで、折り畳まれたタンパク質構造や多タンパク質集合体を安定化させる傾向がある\href{https://www.nature.com/articles/nrmicro.2017.17?error=cookies_not_supported&code=87b4b2eb-5370-49be-862f-ed6bad1e1b44\#:~:text=Macromolecular\%20crowding\%20affects\%20the\%20mobility,we\%20propose\%20that\%20the\%20term}{nature.com}。例えば、合成遺伝子発現系では、密集剤を添加すると必要な酵素と基質がより小さな有効体積に閉じ込められるため、タンパク質収量が増加する。ある研究では、リポソーム人工細胞内でのカプセル化無細胞タンパク質合成(CFPS)が\textbf{高分子クラウディング剤の添加により大幅に促進された}。GarenneとNoireaux(2020)は、最小限の合成細胞にクラウディングポリマー(PEGなど)を導入することで転写・翻訳収率が向上し、細胞内構造の自己組織化さえ可能になることを実証した\href{https://www.researchgate.net/publication/341582539_Analysis_of_Cytoplasmic_and_Membrane_Molecular_Crowding_in_Genetically_Programmed_Synthetic_Cells\#:~:text=,}{researchgate.net}。特に注目すべきは、\textit{膜上}でのみクラウディングを達成する(PEGを脂質二重層内面に固定する)だけで、細胞質タンパク質の発現を高レベルに維持し、細菌細胞骨格タンパク質MreBのフィラメント化を誘発するのに十分であった点である\href{https://www.researchgate.net/publication/341582539_Analysis_of_Cytoplasmic_and_Membrane_Molecular_Crowding_in_Genetically_Programmed_Synthetic_Cells\#:~:text=,}{researchgate.net}。これは、クラウディングが(細胞骨格ネットワークのような)\textbf{機能的複合体の組織化}を促進し、おそらく関連分子を局所的に濃縮することで高い反応速度を維持することを示唆している。要するに、密集した人工細胞質はより\textit{生物学的に有効な}反応環境として機能し、希薄溶液では起こらない多段階経路や集合体を可能にする。希薄条件下では非常に遅い、あるいは不利な反応(例:鋳型ポリマー化、タンパク質オリゴマー化)も、密集した細胞様内部環境では効率的に進行し得る。
    \item \textbf{拡散速度の低下と反応速度制限の可能性:}一方、粘度の上昇と混雑は\textit{分子拡散を阻害}し、反応物の移動性に依存する反応(すなわち拡散制限反応)を遅らせる可能性がある。粘性のある密集溶液では、基質と酵素が遭遇するまでに時間がかかる。例えば、大きなタンパク質や複合体は、細胞質内では水中でよりも1桁遅く拡散する。人工細胞の細胞質が過度に粘性化すると、酵素が基質を十分に速く見つけられず、反応全体のフラックスが低下する可能性がある。コアセルベート原生細胞に関する研究では、これらの液滴が反応物を濃縮する一方で (平衡状態と触媒速度にとって有利)である一方、その**内部粘度は水の100~1000倍であり、これが\textbf{拡散制限ステップを遅延させる}可能性がある\href{https://pmc.ncbi.nlm.nih.gov/articles/PMC11256357/\#:~:text=The\%20high\%20macromolecular\%20content\%20of,ribozymes\%20and\%20other\%20large\%20molecules}{pmc.ncbi.nlm.nih.gov}。例えば、凝集体内部での2つの巨大ポリマー間の反応は、それらの平衡結合が有利であっても、ポリマー自体の拡散速度が遅いため阻害される可能性がある。したがって、トレードオフが存在する:\textbf{過密状態は局所濃度を増加させ遷移状態を安定化させることで反応を加速するが、過剰な過密は分子移動性を低下させることで反応を遅延させる可能性がある}\href{https://pmc.ncbi.nlm.nih.gov/articles/PMC11256357/\#:~:text=The\%20high\%20macromolecular\%20content\%20of,ribozymes\%20and\%20other\%20large\%20molecules}{pmc.ncbi.nlm.nih.gov}。人工細胞を設計する際には、細胞化学を模倣するのに十分な過密状態を確保しつつ、システムが動力学的に不活性化するほど過密にならないよう、バランスを取る必要がある。
    \item \textbf{特定プロセスの選択的増強:}生化学的プロセスは混雑への応答性が異なる。\textbf{低分子反応}や小分子基質を用いる酵素反応は、拡散よりも酵素キネティクスによって制限されることが多いため、混雑した細胞質における濃度上昇と溶媒特性の変化からほぼ純粋な恩恵を受け得る。対照的に、\textbf{巨大複合体}を伴うプロセス(リボソーム組立、染色体動態)は、粘性のある細胞質では拡散が著しく制限される。クラウディングは特定の成分の\textit{相分離}(例:RNA/タンパク質凝縮体の形成)も誘発し、反応を区画化できる。実際の細胞では、クラウディング状態が相分離により膜を持たない細胞小器官(ストレス顆粒など)の形成を促進する。同様に人工細胞では、閾値密度を超えると成分が凝集体様液滴へ分離する現象が観察される可能性がある。これにより反応がそれらの液滴に局在化され、液滴内では特定の反応が増強される一方、他の反応は分離または抑制されることがある。生命起源研究における一例として、凝集液微小液滴は、特異的な混雑微小環境を提供することで、酵素的RNA触媒反応やアミロイド集合などの反応を\textbf{加速し調整する}ことができる\href{https://pmc.ncbi.nlm.nih.gov/articles/PMC11256357/\#:~:text=selectivity\%20for\%20amino\%20acids\%20that,strongly\%20with\%20the\%20coacervate\%20matrix}{pmc.ncbi.nlm.nih.gov}\href{https://pmc.ncbi.nlm.nih.gov/articles/PMC11256357/\#:~:text=The\%20high\%20macromolecular\%20content\%20of,ribozymes\%20and\%20other\%20large\%20molecules}{pmc.ncbi.nlm.nih.gov}。設計上の教訓は、\textbf{混雑が全ての反応を均一に促進または抑制するわけではない}という点である。むしろ、反応ネットワークをより複雑な方法で変化させ、時には各目的機能ごとに混雑レベルの最適化を必要とする。
    \item \textbf{生物学的意義と最小閾値:}生細胞からの実証的証拠は、効率的な反応を維持するために細胞が積極的に混雑レベルを管理していることを示している。大腸菌(\textit{E. coli})や他の微生物は\textbf{「混雑恒常性(crowding homeostasis)」}を示し、総高分子濃度を狭い範囲内に維持している\href{https://www.nature.com/articles/nrmicro.2017.17?error=cookies_not_supported&code=87b4b2eb-5370-49be-862f-ed6bad1e1b44\#:~:text=cytoplasm,relatively\%20constant\%20levels\%20of\%20macromolecules}{nature.com}\href{https://www.nature.com/articles/nrmicro.2017.17?error=cookies_not_supported&code=87b4b2eb-5370-49be-862f-ed6bad1e1b44\#:~:text=microbial\%20physiology,process\%20by\%20which\%20cells\%20maintain}{nature.com}。細胞が希釈されると(低浸透圧条件下での水分吸収により)、生体高分子の濃度が低下し、生命維持プロセスが鈍化または誤調節される。例えば、非特異的クラウディングが低いと転写因子のDNA結合が減少する\href{https://www.nature.com/articles/nrmicro.2017.17?error=cookies_not_supported&code=87b4b2eb-5370-49be-862f-ed6bad1e1b44\#:~:text=6,17\%2C\%20488\%E2\%80\%93496\%20\%282004}{nature.com}\href{https://www.nature.com/articles/nrmicro.2017.17?error=cookies_not_supported&code=87b4b2eb-5370-49be-862f-ed6bad1e1b44\#:~:text=7,191\%2C\%20231\%E2\%80\%93237\%20\%282009}{nature.com}。逆に、細胞が過密状態(脱水や高浸透圧など)になると、拡散が過度に遅くなり一部の反応が停滞し、最終的に成長が制限される\href{https://www.nature.com/articles/nrmicro.2017.17?error=cookies_not_supported&code=87b4b2eb-5370-49be-862f-ed6bad1e1b44\#:~:text=6,17\%2C\%20488\%E2\%80\%93496\%20\%282004}{nature.com}。したがって、生命にとって、そしておそらく人工細胞にとっても、\textbf{最適なクラウディング範囲}が存在するようだ。これは「最小反応環境」に必要な\textit{閾値細胞質密度}の概念に関連している。人工細胞の内部が\textbf{過度に希薄(閾値以下)}な場合、連鎖した生化学反応を支えられない可能性がある——基質が酵素と十分に頻繁に遭遇せず、多酵素経路が正しく機能しない。一方、内部が\textbf{過密状態(上限閾値以上)}になると、流動性を失い分子が閉じ込められ、同様に機能が損なわれる。正確な値はシステムに依存するが、細胞データから推測すると、典型的な代謝反応を維持するには少なくとも約15~20%程度の巨分子体積分率が必要である(それ以下では遭遇率の低下により代謝が「停滞」する可能性がある)。実際、古い生化学的研究では、極度に希薄な細胞質(クラウディング率10\%未満)では代謝ネットワークが維持できないことが示唆されている\href{https://bionumbers.hms.harvard.edu/bionumber.aspx?s=n&v=12&id=105814\#:~:text=Comments\%20In\%20both\%20stationary\%20and,range\%20wasn\%27t\%20located\%20in\%20Kao}{bionumbers.hms.harvard.edu}。高濃度側では、流体としての機能維持の上限は高分子分画30~40%程度である。40%超(すなわち0.4 g/mL以上)では、活性細胞でさえ可動性に支障をきたしガラス状状態に陥る可能性がある\href{https://bionumbers.hms.harvard.edu/bionumber.aspx?s=n&v=12&id=105814\#:~:text=Method\%20Using\%20equation\%2010\%20p,however\%20range}{bionumbers.hms.harvard.edu}\href{https://www.nature.com/articles/s41598-017-14883-y?error=cookies_not_supported&code=b0600d9e-c852-48ed-ab9c-7526a5c6893b\#:~:text=cancer\%20cells,0.3\%20g\%2FmL}{nature.com}。要約すると、 \textbf{最小限でありながら機能的な細胞質のための妥当な閾値範囲は、約20~30\%の高分子含有量}(おおよそ0.2~0.3 g/mL)である。この範囲内では、細胞質は生化学反応を促進するのに十分な密度でありながら、輸送には十分な流動性を保っている。人工細胞は、細胞のような反応性を達成するために、少なくともこの範囲の下限を満たすように設計されるべきである。
\end{itemize}

\subsection{人工細胞における細胞質粘度の測定と制御}

\textbf{内部粘度と拡散の測定:}人工細胞質を合理的に設計するには、微小な細胞模倣体(数十マイクロメートルサイズの微小液滴や小胞)内部の過密状態と粘度を定量化するツールが必要である。この目的のために細胞生物学からいくつかの技術が応用されている。\textbf{蛍光回復法(FRAP)}は一般的な手法である:蛍光プローブ(微小色素または標識タンパク質)を特定部位で消光させ、未消光分子の拡散復帰時間を測定する。これにより拡散係数、ひいては粘度が反映される。FRAPは\textit{生体内}(例:細胞内GFP拡散\href{https://pmc.ncbi.nlm.nih.gov/articles/PMC1184383/\#:~:text=Photobleaching\%20recovery\%20and\%20anisotropy\%20decay,2}{pmc.ncbi.nlm.nih.gov}\href{https://bionumbers.hms.harvard.edu/bionumber.aspx?id=102561&ver=2\#:~:text=Value\%203,S65T\%20in}{bionumbers.hms.harvard.edu})や、蛍光性溶質を含む人工細胞系でも同様に用いられている。別の手法として\textbf{蛍光相関分光法(FCS)}があり、自発的な強度変動を解析して拡散速度を抽出する。この手法は極小体積や単一液滴での測定が可能という利点がある。\textbf{粒子追跡マイクロレオロジー}は特に有益で、人工細胞質内に微小なトレーサービーズを封入し、顕微鏡下でそのブラウン運動を追跡する。ビーズの時間経過に伴う平均二乗変位から、媒体の粘弾性係数と粘度が得られる\href{https://www.nature.com/articles/s41598-017-14883-y?error=cookies_not_supported&code=b0600d9e-c852-48ed-ab9c-7526a5c6893b\#:~:text=We\%20performed\%20high,As\%20detailed\%20later}{nature.com}\href{https://www.nature.com/articles/s41598-017-14883-y?error=cookies_not_supported&code=b0600d9e-c852-48ed-ab9c-7526a5c6893b\#:~:text=we\%20found\%20that\%20the\%20BSA,a\%20low\%20shear\%20rate\%20limit38}{nature.com}。西澤\textit{ら}は光学トラップを用いて細胞抽出液中の1µmビーズを追跡し、詳細なレオロジー曲線を得た\href{https://www.nature.com/articles/s41598-017-14883-y?error=cookies_not_supported&code=b0600d9e-c852-48ed-ab9c-7526a5c6893b\#:~:text=We\%20performed\%20high,The\%20viscosity\%20\%CE\%B7\%20of\%20the}{nature.com} – 同様の手法は合成細胞コンパートメントにも適用可能である。また、環境の粘度や移動度に応じて蛍光寿命や蛍光効率が変化する\textbf{分子ローター色素}や\textbf{FRETベースの粘度センサー}も存在する。例えばBoersma\textit{ら}は生細胞内の高分子クラウディングを定量化する遺伝子コード型FRETセンサーを開発した\href{https://www.nature.com/articles/nrmicro.2017.17?error=cookies_not_supported&code=87b4b2eb-5370-49be-862f-ed6bad1e1b44\#:~:text=Scholar\%20scholar}{nature.com}。こうしたプローブを人工細胞に発現・組み込むことで局所的クラウディングレベルを報告可能だ。これらの技術を組み合わせることで、人工細胞の内部粘度が組成(タンパク質やポリマーの添加など)にどう依存するかをマッピングできる。

\textbf{細胞質密度と粘度の制御:}人工細胞の構築では、通常、細胞様容器(脂質小胞、ハイドロゲルマイクロカプセル、またはコアセルベート液滴)を用意し、生体分子成分のカクテルで満たす。適切な細胞質密度を達成するには、細胞の充填された内部を模倣するため、「クラウディング剤」―不活性ポリマーまたは高濃度の可溶性タンパク質―を添加することが多い。一般的なクラウディング剤にはPEG(ポリエチレングリコール)、デキストラン、フィコール、またはBSAがある。それらの濃度を調整することでクラウディングを制御できる。例えば、約30\%重量/体積のクラウディングを達成するには、封入前に溶液に100~150 mg/mLの高分子量ポリマーを添加する。実際、合成生物学実験では\textbf{適度なクラウディング(ポリマー約5~15\% w/v)}が遺伝子発現などのプロセス収率向上に頻繁に用いられる\href{https://www.researchgate.net/publication/341582539_Analysis_of_Cytoplasmic_and_Membrane_Molecular_Crowding_in_Genetically_Programmed_Synthetic_Cells\#:~:text=,}{researchgate.net}。人工細胞に脂質膜がある場合、\textbf{膜表面のみをクラウディング}する戦略が興味深い。Garenne \textit{et al.} は、リポソーム膜の内葉にPEGポリマーを付着させることで局所的なクラウディング層を形成し、内部全体を均一にクラウディングする必要なく高タンパク質合成を維持できることを示した\href{https://www.researchgate.net/publication/341582539_Analysis_of_Cytoplasmic_and_Membrane_Molecular_Crowding_in_Genetically_Programmed_Synthetic_Cells\#:~:text=,}{researchgate.net}。この「表面クラウディング」は周辺自己組織化(例:細胞骨格タンパク質が膜上で殻や足場を形成するのを助ける)も促進する\href{https://www.researchgate.net/publication/341582539_Analysis_of_Cytoplasmic_and_Membrane_Molecular_Crowding_in_Genetically_Programmed_Synthetic_Cells\#:~:text=,}{researchgate.net}。

\subsection{細胞質密度と粘性の関係性}
\mynote{ボイノイツリーマップ}
\mynote{macromolecular crowdingの総説}
\section{細胞質密度の制御}

\section{細胞質密度の制御メカニズムと分散の定量解析}

\subsection{背景と重要性}

細胞質密度とは、細胞内部の乾燥重量(主にタンパク質やRNAなどの総量)を細胞体積で割った値であり、一種の「細胞内濃度」を表します\href{https://www.frontiersin.org/journals/cell-and-developmental-biology/articles/10.3389/fcell.2022.1017499/full\#:~:text=size\%20control,cells\%2C\%20that\%20is\%2C\%20significantly\%20tighter}{frontiersin.org}。生物は成長に伴い細胞質の質量と体積を増やしますが、この二つの増加がずれると**細胞質の密度(濃さ)**が変化します\href{https://www.nature.com/articles/s42003-022-03348-2\#:~:text=Single,link\%20between\%20the\%20levels\%20of}{nature.com}。密度が高すぎると分子が過密状態になって反応拡散が阻害されたり、低すぎると反応に必要な酵素や基質濃度が不足したりするため、細胞は機能不全に陥ります\href{https://www.nature.com/articles/s42003-022-03348-2\#:~:text=differentiation\%20yields\%20a\%20density\%20homeostasis,the\%20\%E2\%80\%9Cinvisible\%E2\%80\%9D\%20microfluidic\%20arrays\%20that}{nature.com}\href{https://pubmed.ncbi.nlm.nih.gov/30739799/\#:~:text=impairs\%20gene\%20induction\%2C\%20cell,loss\%20of\%20scaling\%20beyond\%20a}{pubmed.ncbi.nlm.nih.gov}。例えば、過剰に大きく成長した細胞では核酸やタンパク質の合成が体積拡大に追いつかず、細胞質が希釈されてシグナル伝達や増殖が障害されることが報告されています\href{https://pubmed.ncbi.nlm.nih.gov/30739799/\#:~:text=impairs\%20gene\%20induction\%2C\%20cell,loss\%20of\%20scaling\%20beyond\%20a}{pubmed.ncbi.nlm.nih.gov}\href{https://pubmed.ncbi.nlm.nih.gov/30739799/\#:~:text=that\%20growing\%20budding\%20yeast\%20and,outside\%20these\%20bounds\%20contribute\%20to}{pubmed.ncbi.nlm.nih.gov}。逆に細胞壁を持つ生物では、体積拡大が質量合成に追いつかないと過剰な高密度=過密状態となり、浸透圧ストレスで増殖停止や細胞破裂を招きます\href{https://pmc.ncbi.nlm.nih.gov/articles/PMC11065075/\#:~:text=In\%20bacteria\%2C\%20algae\%2C\%20fungi\%2C\%20and,density\%20remains\%20constant\%20across\%20a}{pmc.ncbi.nlm.nih.gov}。したがって、\textbf{細胞質密度を一定に保つ(恒常性を維持する)仕組み}は、細胞の正常な増殖と機能維持において基本的かつ重要な課題です。

近年、この細胞質密度の制御精度やばらつきについて、生物学的・物理生物学的手法を融合して解明する研究が進んでいます。特に、細菌、酵母、哺乳類細胞という異なる系統の細胞で、細胞質密度がどの程度一定に維持されているのか、その分散(ばらつき)の定量解析と制御メカニズムの比較が注目されています\href{https://www.frontiersin.org/journals/cell-and-developmental-biology/articles/10.3389/fcell.2022.1017499/full\#:~:text=density\%2C\%20which\%20is\%20simply\%20computed,extraordinary\%20stability\%20of\%20cellular\%20mass}{frontiersin.org}。以下では、各細胞タイプ(細菌、酵母、哺乳類)について、細胞質密度の分布の狭さ(制御精度)と、それを支える調節機構や影響因子(細胞周期・成長・環境変化・分子機構など)を最新の知見に基づいて整理します。

\subsection{細菌における細胞質密度恒常性}

\textbf{大腸菌}などの細菌は、細胞質密度を非常に安定に維持していることが知られています。栄養条件によって増殖速度が変化しても、細胞あたりの乾燥質量と体積がほぼ比例関係を保ち、結果的に密度は一定に保たれます\href{https://pmc.ncbi.nlm.nih.gov/articles/PMC11065075/\#:~:text=these\%20processes\%20are\%20indeed\%20coupled4,picture\%20that\%20emerges\%20is\%20that}{pmc.ncbi.nlm.nih.gov}。実際、最近の研究でも、大腸菌は貧栄養から高栄養培地まで\textbf{広範な成長条件でバイオマス密度(質量/体積)を驚くほど一定に維持している}ことが示されています\href{https://pmc.ncbi.nlm.nih.gov/articles/PMC11065075/\#:~:text=remarkably\%20constant\%20across\%20a\%20large,higher\%20biomass\%20density\%20and\%20slower}{pmc.ncbi.nlm.nih.gov}。これは細胞が増殖速度に応じて細胞壁拡張(体積拡大)を調節し、質量合成とのバランスを取っているためと考えられます。

細菌における密度制御の鍵の一つは、\textbf{細胞内浸透圧(ターガー圧)\textbf{と}細胞壁合成}の連携です。Harvard Medical Schoolのグループの研究\href{https://pmc.ncbi.nlm.nih.gov/articles/PMC11065075/\#:~:text=these\%20processes\%20are\%20indeed\%20coupled4,picture\%20that\%20emerges\%20is\%20that}{pmc.ncbi.nlm.nih.gov}\href{https://pmc.ncbi.nlm.nih.gov/articles/PMC11065075/\#:~:text=these\%20apparently\%20confounding\%20observations\%20but,mechanism\%20of\%20action\%20of\%20antibiotics6}{pmc.ncbi.nlm.nih.gov}によれば、速く成長する条件ではリボソームRNAが増えてタンパク質合成が活発化し、それに伴い細胞内の陽イオン(リボソームのカウンターイオン)が蓄積して浸透圧が上昇します。浸透圧が上がると細胞は水を取り込み壁に膨張圧をかけますが、細胞壁の流動性(柔軟さ)が高まることで壁拡張が促進され、結果として体積が質量増加に見合うよう拡大します\href{https://pmc.ncbi.nlm.nih.gov/articles/PMC11065075/\#:~:text=these\%20apparently\%20confounding\%20observations\%20but,mechanism\%20of\%20action\%20of\%20antibiotics6}{pmc.ncbi.nlm.nih.gov}。このように\textbf{リボソーム産生(質量増加)とターガー圧による細胞壁拡張が連動}することで、増殖速度に応じた体積調節が行われ、細胞質密度が一定に保たれるというモデルが提唱されています\href{https://pmc.ncbi.nlm.nih.gov/articles/PMC11065075/\#:~:text=these\%20apparently\%20confounding\%20observations\%20but,mechanism\%20of\%20action\%20of\%20antibiotics6}{pmc.ncbi.nlm.nih.gov}。実際、人工的に浸透圧バランスを乱す実験では、例えば培地に高濃度の溶質を加えて細胞から水分を奪うと、通常より高い細胞密度となり増殖も遅くなることが観察されています\href{https://pmc.ncbi.nlm.nih.gov/articles/PMC11065075/\#:~:text=remarkably\%20constant\%20across\%20a\%20large,higher\%20biomass\%20density\%20and\%20slower}{pmc.ncbi.nlm.nih.gov}。これは平常時の密度恒常性機構が破綻した状態で、密度が上昇してしまう現象です。

単一細胞レベルで見ると、細菌の密度は一瞬一瞬で全く不変というわけではなく、細胞成長に伴って**微妙なゆらぎ(密度の一過的変動)\textbf{が生じます。Nematiら (2022)\href{https://www.nature.com/articles/s42003-022-03348-2\#:~:text=Single,link\%20between\%20the\%20levels\%20of}{nature.com}の単一細胞追跡研究では、\textbf{大腸菌は質量も体積も指数関数的に増加する一方で、密度(群集度)は成長中に一時的に変動する}ことが報告されています\href{https://www.nature.com/articles/s42003-022-03348-2\#:~:text=Single,link\%20between\%20the\%20levels\%20of}{nature.com}。具体的には、ある瞬間には質量増加が体積増加をわずかに上回り密度が上昇し、別の時期には逆に体積が先行して密度が低下する、といった揺らぎが見られます\href{https://www.nature.com/articles/s42003-022-03348-2\#:~:text=Single,link\%20between\%20the\%20levels\%20of}{nature.com}。しかし興味深いことに、こうした変動は無秩序ではなく}長期的には密度が平均値にとどまるような調節(ホームオスタシス)**が働いていました\href{https://www.nature.com/articles/s42003-022-03348-2\#:~:text=fluctuate\%20during\%20growth,the\%20\%E2\%80\%9Cinvisible\%E2\%80\%9D\%20microfluidic\%20arrays\%20that}{nature.com}。モデル解析によると、細胞は質量と体積の増加率を適宜ずらすことで密度を一定範囲内に収める仕組みを持つと示唆されています\href{https://www.nature.com/articles/s42003-022-03348-2\#:~:text=fluctuate\%20during\%20growth,the\%20\%E2\%80\%9Cinvisible\%E2\%80\%9D\%20microfluidic\%20arrays\%20that}{nature.com}。さらに、この微小な密度変動は細胞の分裂タイミングや増殖率にも影響を与えることが示されており、\textbf{細胞質の群集度(クラウディング)のレベルが代謝や集団の適応度とリンクしている}可能性が指摘されています\href{https://www.nature.com/articles/s42003-022-03348-2\#:~:text=differentiation\%20yields\%20a\%20density\%20homeostasis,genetic\%20variability\%20into\%20consideration}{nature.com}\href{https://www.nature.com/articles/s42003-022-03348-2\#:~:text=differentiation\%20yields\%20a\%20density\%20homeostasis,the\%20\%E2\%80\%9Cinvisible\%E2\%80\%9D\%20microfluidic\%20arrays\%20that}{nature.com}。つまり、細菌は分子的なノイズにより密度が揺らぎつつも、それを生理的に有利な範囲に収めるよう制御し、結果として単一集団内の密度分布は比較的狭い幅(おそらく数%程度の変動幅)に保たれていると考えられます。

\subsection{酵母における細胞質密度の制御}

\textbf{出芽酵母や分裂酵母}といった真核単細胞生物でも、基本的に細胞質密度は一定に近い値に維持されています。近年の定量イメージング研究によれば、酵母細胞における乾燥質量密度の\textbf{細胞個体間ばらつきは数%程度と非常に小さい}ことが示されています\href{https://pubmed.ncbi.nlm.nih.gov/34100714/\#:~:text=variation\%20,dependent\%20density}{pubmed.ncbi.nlm.nih.gov}。例えばPascal Odermattら (2021)\href{https://pubmed.ncbi.nlm.nih.gov/34100714/\#:~:text=Intracellular\%20density\%20impacts\%20the\%20physical,Spatially\%20heterogeneous\%20patterns\%20of\%20density}{pubmed.ncbi.nlm.nih.gov}は、新しい定量位相顕微鏡法を用いて分裂酵母 \textit{Schizosaccharomyces pombe} の細胞質密度を測定し、集団全体での密度の**変動係数(CV)は約6\%**と報告しました\href{https://pubmed.ncbi.nlm.nih.gov/34100714/\#:~:text=variation\%20,dependent\%20density}{pubmed.ncbi.nlm.nih.gov}。これは酵母においても細胞質密度がかなり厳密に一定に近い値に制御されていることを意味します。

一方で酵母では、哺乳類細胞と異なり\textbf{細胞周期に伴う規則的な密度変動}が観察されています。Odermattらの研究によると、分裂酵母では細胞質密度はG2期(分裂直前の成長期)に向かって徐々に低下し、有糸分裂と細胞分裂時に再び上昇、そして新たに分裂して生まれた直後の娘細胞で急低下する、という周期的変化を示しました\href{https://pubmed.ncbi.nlm.nih.gov/34100714/\#:~:text=Intracellular\%20density\%20impacts\%20the\%20physical,Spatially\%20heterogeneous\%20patterns\%20of\%20density}{pubmed.ncbi.nlm.nih.gov}。すなわち、\textbf{細胞質密度は細胞周期の進行に応じてゆっくり減少し、分裂時にピークを迎え、分裂直後に最も低くなる}というパターンです\href{https://pubmed.ncbi.nlm.nih.gov/34100714/\#:~:text=Intracellular\%20density\%20impacts\%20the\%20physical,cycle}{pubmed.ncbi.nlm.nih.gov}。興味深いことに、こうした密度の揺らぎは\textbf{体積成長の調節}によって説明できることが判明しています\href{https://pubmed.ncbi.nlm.nih.gov/34100714/\#:~:text=rapidly\%20at\%20cell\%20birth,cycle\%20arrests}{pubmed.ncbi.nlm.nih.gov}。酵母細胞は細胞周期を通じて乾燥重量(バイオマス)をほぼ一定の速度で合成し続けますが\href{https://pubmed.ncbi.nlm.nih.gov/34100714/\#:~:text=rapidly\%20at\%20cell\%20birth,to\%20modulation\%20of\%20volume\%20expansion}{pubmed.ncbi.nlm.nih.gov}、細胞周期の特定の段階で\textbf{体積(細胞壁の拡張)増加が減速・加速する}ことで密度が変動します\href{https://pubmed.ncbi.nlm.nih.gov/34100714/\#:~:text=rapidly\%20at\%20cell\%20birth,cycle\%20arrests}{pubmed.ncbi.nlm.nih.gov}。具体的には、分裂に入ると細胞壁の伸張が一時的に遅くなるため質量合成が勝って密度が上がり、分裂が完了すると娘細胞は急激に体積を拡大する(あるいは親細胞から質量が分配される)ため密度がストンと下がる、という仕組みです\href{https://pubmed.ncbi.nlm.nih.gov/34100714/\#:~:text=rapidly\%20at\%20cell\%20birth,cycle\%20arrests}{pubmed.ncbi.nlm.nih.gov}。このように\textbf{細胞周期に伴って質量と体積のバランスを動的に変化させる}ことで、一細胞内でも密度の調節が行われているのです。通常の増殖条件下では、細胞周期を一巡すれば密度は平均値付近にリセットされ、集団全体で見れば前述の通り狭い分布に収まります\href{https://pubmed.ncbi.nlm.nih.gov/34100714/\#:~:text=variation\%20,dependent\%20density}{pubmed.ncbi.nlm.nih.gov}。しかし細胞周期の特定段階で細胞を人工的に停止させると、密度が通常以上に上昇または下降し、細胞機能に影響を及ぼすことも確認されています\href{https://pubmed.ncbi.nlm.nih.gov/34100714/\#:~:text=division\%20and\%20rapid\%20expansion\%20post,to\%20modulation\%20of\%20volume\%20expansion}{pubmed.ncbi.nlm.nih.gov}。これは細胞周期進行と密度制御がリンクしていることを示唆します。

出芽酵母 \textit{Saccharomyces cerevisiae} でも基本原理は類似しており、各細胞はそのサイズ(体積)に見合った質量を蓄えることで適切な密度を維持しています。通常、出芽酵母は「サイズチェックポイント」(Start)により、十分な質量を蓄えてから細胞周期を進めるため、極端に大きくなりすぎて密度が低下することはありません。しかし人為的操作や突然変異で細胞が異常に大型化した場合には、\textbf{核DNA量やリボソーム量が不足して体積拡大に見合うタンパク質合成ができず、細胞質が希釈されてしまう}ことが明らかになっています\href{https://pubmed.ncbi.nlm.nih.gov/30739799/\#:~:text=impairs\%20gene\%20induction\%2C\%20cell,loss\%20of\%20scaling\%20beyond\%20a}{pubmed.ncbi.nlm.nih.gov}。Neurohrら (2019)\href{https://pubmed.ncbi.nlm.nih.gov/30739799/\#:~:text=that\%20growing\%20budding\%20yeast\%20and,outside\%20these\%20bounds\%20contribute\%20to}{pubmed.ncbi.nlm.nih.gov}の研究では、出芽酵母および初代培養哺乳類細胞を通常よりも大きく成長させると、細胞あたりDNA量の相対的不足から転写・翻訳能力が追いつかなくなり、結果として細胞質密度の低下(希釈)が起こると報告されました\href{https://pubmed.ncbi.nlm.nih.gov/30739799/\#:~:text=that\%20growing\%20budding\%20yeast\%20and,outside\%20these\%20bounds\%20contribute\%20to}{pubmed.ncbi.nlm.nih.gov}。この希釈により細胞周期進行やシグナル伝達に異常をきたし、巨大化した細胞は増殖停止や老化(細胞老化)の表現型を示しました\href{https://pubmed.ncbi.nlm.nih.gov/30739799/\#:~:text=that\%20growing\%20budding\%20yeast\%20and,outside\%20these\%20bounds\%20contribute\%20to}{pubmed.ncbi.nlm.nih.gov}。すなわち\textbf{出芽酵母を含む真核細胞では、適切なDNA対細胞質比(核:細胞質比)が密度維持に重要}であり、その範囲を逸脱すると細胞機能が損なわれるのです。以上のように、酵母細胞は通常のサイズ範囲内では密度を高い精度で一定に保ちつつ、細胞周期や環境変化(栄養・浸透圧など)に応じてその密度を動的に調節する仕組みを備えていると言えます。

\subsection{哺乳類細胞における細胞質密度の恒常性}

哺乳類由来の培養細胞(例えばヒトやマウスの培養細胞株)でも、細胞質密度の均一性と安定性が近年詳細に調べられています。Xili Liuら (2022)\href{https://www.frontiersin.org/journals/cell-and-developmental-biology/articles/10.3389/fcell.2022.1017499/full\#:~:text=perturbations\%20in\%20three\%20cultured\%20mammalian,perturbations\%20such\%20as\%20starvation\%20or}{frontiersin.org}はラマン散乱顕微鏡による新手法を用いて、ヒト培養細胞(HeLa細胞など)におけるタンパク質・脂質の乾燥質量密度を測定しました。その結果、\textbf{同一細胞株の集団内で細胞質密度は非常に狭い分布を示し、細胞質密度のばらつき(CV値)がわずか約7〜9\%程度しかない}ことが報告されています\href{https://www.frontiersin.org/journals/cell-and-developmental-biology/articles/10.3389/fcell.2022.1017499/full\#:~:text=match\%20at\%20L637\%20observed\%20CV,cell\%20volume\%20in\%20individual\%20cells}{frontiersin.org}。これは同じ集団で約30\%程度ばらつく細胞体積や乾燥質量の分布に比べて格段に揃っており、質量と体積が強く協調して増加していることを意味します\href{https://www.frontiersin.org/journals/cell-and-developmental-biology/articles/10.3389/fcell.2022.1017499/full\#:~:text=population\%2C\%20as\%20quantified\%20in\%20terms,However\%2C\%20the}{frontiersin.org}。さらに興味深い点として、哺乳類培養細胞では\textbf{細胞周期のいずれの段階においても細胞質密度に有意な違いが見られず、細胞分裂前後でも密度はほぼ一定に保たれる}ことが示されました\href{https://www.frontiersin.org/journals/cell-and-developmental-biology/articles/10.3389/fcell.2022.1017499/full\#:~:text=perturbations\%20in\%20three\%20cultured\%20mammalian,mass\%20density\%20when\%20mass\%20is}{frontiersin.org}。すなわち、G1期からG2期にかけて細胞質が2倍以上になるにもかかわらず、密度は変化しないということです\href{https://www.frontiersin.org/journals/cell-and-developmental-biology/articles/10.3389/fcell.2022.1017499/full\#:~:text=narrow\%20mass\%20density\%20distribution\%20within,resistant\%20to\%20pharmacological\%20perturbations\%20of}{frontiersin.org}。酵母のような明瞭な周期変動が検出されないのは、哺乳類細胞では質量と体積の増加が全周期を通じて滑らかに並行して進行するためと考えられます。この点は細胞質密度制御の生物種間差異として興味深い所見です。

哺乳類細胞はまた、外的・内的な刺激に対して細胞質密度を調節する応答も示します。Liuらの研究では、培地に浸透圧ストレス(高濃度のソルートを加える)を与えて細胞から水分を引き出すと細胞質密度が上昇し、逆に細胞骨格を薬剤で破壊して細胞形態を変化させた場合にも密度が変化し得ることが示されました\href{https://www.frontiersin.org/journals/cell-and-developmental-biology/articles/10.3389/fcell.2022.1017499/full\#:~:text=than\%20the\%20variability\%20of\%20mass,perturbations\%20such\%20as\%20starvation\%20or}{frontiersin.org}。これは細胞外部の物理化学的条件や内部構造の変化によって、一時的に体積と質量のバランスが変わるためです。しかし一方で、タンパク質合成を阻害したり分解を促進したりする薬剤で\textbf{細胞の質量増減速度に乱れを加えても、細胞質密度自体は意外なほど安定に保たれる}ことが報告されています\href{https://www.frontiersin.org/journals/cell-and-developmental-biology/articles/10.3389/fcell.2022.1017499/full\#:~:text=is\%20independent\%20of\%20the\%20cell,during\%20transitions\%20in\%20physiological\%20state}{frontiersin.org}。例えば一時的にタンパク質合成を止めても、細胞は体積の成長を停止させるか体積を縮小させることで密度の低下を防ぐフィードバック機構を持つ可能性があります\href{https://www.frontiersin.org/journals/cell-and-developmental-biology/articles/10.3389/fcell.2022.1017499/full\#:~:text=is\%20independent\%20of\%20the\%20cell,during\%20transitions\%20in\%20physiological\%20state}{frontiersin.org}。実際この研究では、質量増加や減少への薬理学的介入に対し密度がほとんど変化しなかったことから、「\textbf{細胞質密度を一定に維持する何らかのフィードバック制御}」の存在が示唆されています\href{https://www.frontiersin.org/journals/cell-and-developmental-biology/articles/10.3389/fcell.2022.1017499/full\#:~:text=is\%20independent\%20of\%20the\%20cell,during\%20transitions\%20in\%20physiological\%20state}{frontiersin.org}。このフィードバックの分子メカニズムはまだ完全には解明されていませんが、細胞は自分の「濃さ」を感知して体積調節(浸透圧調節やイオン排出入)や質量合成速度を調整する仕組みを持つと考えられます\href{https://www.frontiersin.org/journals/cell-and-developmental-biology/articles/10.3389/fcell.2022.1017499/full\#:~:text=,density\%20control\%20through\%20active\%20feedback}{frontiersin.org}。古典的な物理モデルでは、細胞は半透膜とポンプを持つ「袋」として振る舞い、内部の不透過性溶質(タンパク質など)の濃度によって浸透圧平衡が決まるため、自発的にも密度は安定するとされます\href{https://www.frontiersin.org/journals/cell-and-developmental-biology/articles/10.3389/fcell.2022.1017499/full\#:~:text=The\%20simplest\%20physical\%20model\%20of,pump\%20leak\%20mechanism\%20predicts\%20that}{frontiersin.org}。しかし実際には細胞は発達した恒常性維持機構を備えており、栄養枯渇時や老化過程など生理的状態の大きな転換期には密度のセットポイント自体を変化させることが分かっています\href{https://www.frontiersin.org/journals/cell-and-developmental-biology/articles/10.3389/fcell.2022.1017499/full\#:~:text=mass\%20density\%20is\%20surprisingly\%20resistant,remains\%20fixed\%20against\%20some\%20perturbations}{frontiersin.org}。例えば\textbf{飢餓(栄養欠乏)に陥った細胞では、内部資源の分解やオートファジーによる質量減少に対し、細胞が体積をどの程度維持・収縮するかによって密度が変動}します。また\textbf{老化した細胞(分裂停止に陥った細胞)は若年細胞よりも細胞質密度が変化する}との報告もあり\href{https://www.frontiersin.org/journals/cell-and-developmental-biology/articles/10.3389/fcell.2022.1017499/full\#:~:text=mass\%20density\%20is\%20surprisingly\%20resistant,remains\%20fixed\%20against\%20some\%20perturbations}{frontiersin.org}、これは老化に伴う細胞質の質的変化(例えば不要タンパク質の蓄積や代謝低下)を反映している可能性があります。近年、細胞容積センサーとしてのメカノトランスダクション因子(例えばYAP/TAZ経路)や、浸透圧変化を感知するmTOR経路など、密度に関連したシグナル経路の存在も示唆されています。これらは細胞が自らの大きさと中身のバランスを検知し、イオンポンプを介した水分出し入れや合成系のオンオフを調節することで、密度恒常性を実現していると考えられます。

\subsection{細胞質密度変動が細胞機能に与える影響}

前述の通り、細胞質密度は細胞の生命活動に直結する要因です。適切な密度からの\textbf{逸脱(過密あるいは希薄化)\textbf{は様々な機能障害を引き起こします。密度が高すぎる場合、細胞内は分子で溢れた「混雑状態」となり、拡散による物質輸送や酵素反応の速度が低下します\href{https://www.nature.com/articles/s42003-022-03348-2\#:~:text=differentiation\%20yields\%20a\%20density\%20homeostasis,the\%20\%E2\%80\%9Cinvisible\%E2\%80\%9D\%20microfluidic\%20arrays\%20that}{nature.com}。例えば大腸菌では、密度ゆらぎによって一部の細胞で群集度が高まると、その細胞の分裂速度(増殖率)が低下することが観察されており、過度の高密度は代謝や増殖に不利に作用することが示唆されています\href{https://www.nature.com/articles/s42003-022-03348-2\#:~:text=differentiation\%20yields\%20a\%20density\%20homeostasis,the\%20\%E2\%80\%9Cinvisible\%E2\%80\%9D\%20microfluidic\%20arrays\%20that}{nature.com}。一方、密度が低すぎる(細胞質が薄まりすぎた)場合には、生命活動に必要な酵素やリボソーム濃度が低下してしまいます\href{https://pubmed.ncbi.nlm.nih.gov/30739799/\#:~:text=impairs\%20gene\%20induction\%2C\%20cell,loss\%20of\%20scaling\%20beyond\%20a}{pubmed.ncbi.nlm.nih.gov}。出芽酵母で人工的に巨大化に伴う希薄化を起こした実験では、転写因子の誘導や細胞周期の進行が著しく阻害され、シグナル伝達系も正常に働かなくなることが示されています\href{https://pubmed.ncbi.nlm.nih.gov/30739799/\#:~:text=impairs\%20gene\%20induction\%2C\%20cell,loss\%20of\%20scaling\%20beyond\%20a}{pubmed.ncbi.nlm.nih.gov}。これは}細胞内のあらゆる化学反応系が適切な濃度条件から外れると効率よく機能しなくなる}ことを意味します。実際、その研究では大きすぎる細胞では細胞内のDNA:質量比が低下し、必要なRNAやタンパク質産生量を維持できなくなることが老化(エイジング)の一因になる可能性が示されました\href{https://pubmed.ncbi.nlm.nih.gov/30739799/\#:~:text=defects\%20are\%20due\%20to\%20the,these\%20bounds\%20contribute\%20to\%20aging}{pubmed.ncbi.nlm.nih.gov}。以上より、\textbf{細胞質密度は生命現象の効率と健全性に深く関与するパラメータ}であり、各種の生物はその密度を狭い最適範囲に維持するよう進化させてきたと考えられます。

\subsection{おわりに(総合的考察)}

細胞質密度の恒常性は、細菌から真核生物まで共通して見られる現象であり、その\textbf{制御精度は驚くほど高い}ことが近年の研究で明らかになってきました。\href{https://www.frontiersin.org/journals/cell-and-developmental-biology/articles/10.3389/fcell.2022.1017499/full\#:~:text=match\%20at\%20L637\%20observed\%20CV,cell\%20volume\%20in\%20individual\%20cells}{frontiersin.org}\href{https://pubmed.ncbi.nlm.nih.gov/34100714/\#:~:text=variation\%20,dependent\%20density}{pubmed.ncbi.nlm.nih.gov} 各細胞タイプとも、通常の条件下では細胞質密度の分布は数%程度のばらつきに収まり、細胞は質量と体積の増加を巧みに協調させています。細胞質密度の制御には、生物物理学的な受動要因(浸透圧平衡や膜透過性に基づくポンプ・リーク機構\href{https://www.frontiersin.org/journals/cell-and-developmental-biology/articles/10.3389/fcell.2022.1017499/full\#:~:text=The\%20simplest\%20physical\%20model\%20of,pump\%20leak\%20mechanism\%20predicts\%20that}{frontiersin.org})と生物学的な能動制御(細胞周期チェックポイント、栄養やストレス応答経路、細胞骨格のリモデリングなど)が統合的に関与しています。細菌では細胞壁合成とリボソーム合成のカップリングによる密度調節、酵母では細胞周期進行に伴う体積成長速度の調節、哺乳類細胞ではイオンチャネル・ポンプや機械刺激センサーを介した能動的な体積・質量フィードバックなど、様々なレベルの制御機構が存在します。それぞれ詳細な分子メカニズムは完全には解明されていないものの、\textbf{細胞は自らの濃度を計測し、最適な範囲に収めるよう調節するフィードバックループを持つ}と総括できます。

\section{細胞質の流動性と適応的意義}
\mynote{Viscoadaptationの論文をまとめる}\cite{persson2020viscoadaptation}
\section{細胞質の溶解度}
\mynote{pHが下がって溶解度が下がることが示唆されている}
細胞質内の混雑や粘性が生化学反応に与える影響は、反応物の拡散、酵素活性、構造形成ダイナミクスを通じて複雑な形で現れる。高分子混雑は「除外体積効果(excluded volume effect)」を介して反応物の有効濃度を高め、熱力学的に構造化された状態(例:フォールディング、集合体形成)を安定化させる。これは特定の酵素反応や自己集合プロセスにおいて反応速度を増加させる効果がある。

一方で、混雑環境に伴う粘性上昇は分子の拡散を阻害し、反応物や酵素の空間的接触頻度を低下させることにより、反応速度を減速させる可能性がある。
\section{Mother Machineによる一細胞解析}
\subsection{長期計測のためのデバイス}
Mother MachineはP.wangによって発表されたマイクロ流体デバイスであり、新鮮な培地が一定の流速で流れるメイン流路と細胞がトラップされその内部で成長分裂する観察チャネルからなる。観察チャネルの一方の端はメイン流路に接続されており、ここから常に栄養が供給される。もう一方の端は閉じられており、閉じられた端にいつ細胞が成長分裂をするとメイン流路側に他の細胞を押し出していき、押し出された細胞は最終的に培地の流れによって排除される。一方で、閉じられた側にいる細胞は長期間にわたり観察チャネルの中に止まる。この細胞を観察することで細胞の状態を長期間追跡することができる。
\section{Off-Axis QPIの原理}
Off-axis型QPIは、サンプルによって位相が変化した光波と、基準波(リファレンスビーム)を斜め方向から干渉させ、その干渉パターンから定量的な位相情報を復元する干渉計測手法である。ホログラフィック原理とフーリエ変換を基礎とし、単一ショットで高精度な位相画像を取得できる。この技術は非侵襲・高精度に生体サンプルの屈折率変化や質量密度の評価に応用される。
サンプルによって変調された光を$\bm{E}_{\mathrm{obj}}(\bm{r})$、基準波を$\bm{E}_{\mathrm{ref}}(\bm{r})$とすると、撮像面における強度$I(\bm{r})$は次式で表される:

\begin{align}
I(\bm{r}) &= \abs{\bm{E}_{\mathrm{obj}}(\bm{r}) + \bm{E}_{\mathrm{ref}}(\bm{r})}^2 \\
&= \abs{\bm{E}_{\mathrm{obj}}(\bm{r})}^2 + \abs{\bm{E}_{\mathrm{ref}}(\bm{r})}^2 + \bm{E}_{\mathrm{obj}}(\bm{r}) \bm{E}_{\mathrm{ref}}^*(\bm{r}) + \bm{E}_{\mathrm{obj}}^*(\bm{r}) \bm{E}_{\mathrm{ref}}(\bm{r})
\end{align}

ここで$^*$は複素共役を表す。
基準波がオフアクシス角$\theta$で傾いていると、平面波として以下のように記述できる:

\begin{equation}
\bm{E}_{\mathrm{ref}}(\bm{r}) = A_{\mathrm{ref}} e^{i (\bm{k}_{\mathrm{ref}} \cdot \bm{r})}
\end{equation}

これにより干渉項は、空間周波数的にオフセットされた側帯成分を含む:

\begin{equation}
I(\bm{r}) = \text{DC} + A_{\mathrm{obj}}(\bm{r}) A_{\mathrm{ref}} e^{i (\phi_{\mathrm{obj}}(\bm{r}) - \bm{k}_{\mathrm{ref}} \cdot \bm{r})} + \text{c.c.}
\end{equation}
撮像した干渉画像$I(\bm{r})$を2次元フーリエ変換してスペクトル$I(\bm{f})$に変換する:
\begin{equation}
I(\bm{f}) = \mathcal{F}\{ I(\bm{r}) \}
\end{equation}
ここで、$\mathcal{F}$はフーリエ変換、$\bm{f}$は空間周波数である。干渉項は空間周波数空間においてオフセットされた位置に現れる。目的とする干渉項(サイドバンド)を周波数空間で切り出し、フィルタを適用したのち、逆フーリエ変換を行う:
\begin{equation}
H(\bm{r}) = \mathcal{F}^{-1}\{ \text{Filtered}[I(\bm{f})] \}
\end{equation}
これにより、複素振幅$H(\bm{r}) = A(\bm{r}) e^{i \phi(\bm{r})}$を得る。ここで、$\phi(\bm{r})$が取得した位相分布となる。
最終的な位相画像は次式で定義される:
\begin{equation}
\phi(\bm{r}) = \arg \left( H(\bm{r}) \right)
\end{equation}
なお、取得された位相には$2\pi$の不連続があるため、位相アンラッピング処理が必要である。

