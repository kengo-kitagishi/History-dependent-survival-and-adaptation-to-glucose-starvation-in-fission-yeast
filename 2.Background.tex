\chapter{Background}

\section{細胞の休眠状態における細胞質の状態}

酵母細胞は、温度変化、浸透圧、酸化ストレスなど、さまざまなストレスに直面すると、代謝的および形態的な変化を起こす。

Munderら(2023)は、エネルギー不足により細胞内pHが低下すると、酵母細胞が休眠状態への移行を促進することを示しました。細胞質が酸性pHになると、細胞は体積を縮小し、高分子集合体を形成しました。これらの細胞は、オルガネラの移動性や外因性トレーサー粒子の移動性が低下し、活性細胞に比べて測定可能なほど硬くなりました。

栄養が枯渇した環境では、単細胞生物である大腸菌や酵母、哺乳類細胞は増殖を停止し、生存のための休眠状態に入る。細胞がエネルギー枯渇状態に陥ったり細胞質pHが低下したりすると、細胞質全体がガラス状態とも形容される非常に拡散の遅い状態に転移する。

\textbf{細胞質固体化}とは、栄養枯渇により細胞質が流動的な状態から\textbf{固体的(ガラス様)の状態}に移行し、細胞内部の構造物の動きが大きく制限される現象を指す\href{https://pmc.ncbi.nlm.nih.gov/articles/PMC6857596/\#:~:text=run\%20out\%20of\%20nutrients,2016\%20\%3B\%20\%2043}{pmc.ncbi.nlm.nih.gov}。分裂酵母 \textit{Schizosaccharomyces pombe} では、グルコース源が枯渇してから数日後に細胞質の動的性質が劇的に変化し、多くの細胞内顆粒や小器官が\textbf{ほとんど拡散しなくなる}ことが観察されている\href{https://pmc.ncbi.nlm.nih.gov/articles/PMC6857596/\#:~:text=run\%20out\%20of\%20nutrients,2016\%20\%3B\%20\%2043}{pmc.ncbi.nlm.nih.gov}。一方で、小さな分子(例:GFP程度のサイズ, 約4–5 nm)は固体化後も細胞質内を\textbf{比較的自由に拡散}できる\href{https://pmc.ncbi.nlm.nih.gov/articles/PMC6857596/\#:~:text=run\%20out\%20of\%20nutrients,2016\%20\%3B\%20\%2043}{pmc.ncbi.nlm.nih.gov}。すなわち、固体化した細胞質では\textbf{サイズ依存的な可動性の差}が生じ、リボソームのような\textbf{大きな分子複合体}の拡散は著しく抑制される一方で、小分子や低分子量タンパク質は移動可能なままとなる\href{https://pmc.ncbi.nlm.nih.gov/articles/PMC11214080/\#:~:text=spores\%20and\%20uncovered\%20signaling\%20pathways,such\%20as\%20ribosomes\%2C\%20is\%20restricted}{pmc.ncbi.nlm.nih.gov}。この固体化状態は可逆的であり、適切な条件下で再び流動的な細胞質へ戻りうる。

細胞質固体化とは、栄養枯渇により細胞質が流動的な状態から\textbf{固体的(ガラス様)の状態}に移行し、細胞内部の構造物の動きが大きく制限される現象を指す\href{https://pmc.ncbi.nlm.nih.gov/articles/PMC6857596/\#:~:text=run\%20out\%20of\%20nutrients,2016\%20\%3B\%20\%2043}{pmc.ncbi.nlm.nih.gov}。分裂酵母 \textit{Schizosaccharomyces pombe} では、グルコース源が枯渇してから数日後に細胞質の動的性質が劇的に変化し、多くの細胞内顆粒や小器官が\textbf{ほとんど拡散しなくなる}ことが観察されている\href{https://pmc.ncbi.nlm.nih.gov/articles/PMC6857596/\#:~:text=run\%20out\%20of\%20nutrients,2016\%20\%3B\%20\%2043}{pmc.ncbi.nlm.nih.gov}。一方で、小さな分子(例:GFP程度のサイズ, 約4–5 nm)は固体化後も細胞質内を\textbf{比較的自由に拡散}できる\href{https://pmc.ncbi.nlm.nih.gov/articles/PMC6857596/\#:~:text=run\%20out\%20of\%20nutrients,2016\%20\%3B\%20\%2043}{pmc.ncbi.nlm.nih.gov}。すなわち、固体化した細胞質では\textbf{サイズ依存的な可動性の差}が生じ、リボソームのような\textbf{大きな分子複合体}の拡散は著しく抑制される一方で、小分子や低分子量タンパク質は移動可能なままとなる\href{https://pmc.ncbi.nlm.nih.gov/articles/PMC11214080/\#:~:text=spores\%20and\%20uncovered\%20signaling\%20pathways,such\%20as\%20ribosomes\%2C\%20is\%20restricted}{pmc.ncbi.nlm.nih.gov}。この固体化状態は可逆的であり、適切な条件下で再び流動的な細胞質へ戻りうる。

\section{細胞内の高分子混雑環境と生化学反応に与える影響}
細胞質は非常に高密度な溶液環境であり、多数のタンパク質や核酸などの高分子が詰め込まれている。典型的な細胞ではその質量の約70\%前後が水で占められ、残りの約30\%がタンパク質・核酸・脂質・多糖などの乾燥重量で構成されている。\cite{monterroso2024macromolecular}
高分子混雑環境は、細胞の物性に著しい影響を及ぼし、その結果代謝反応を調節する。
\subsection{細胞質密度と粘性の関係性}
\mynote{ボイノイツリーマップ}
\mynote{macromolecular crowdingの総説}
\section{細胞質密度の制御}
一般にヒトなど真核生物から細菌まで細胞種ごとに固有の細胞質密度が維持されており、同じ細胞型の間では細胞質密度のばらつきが細胞サイズのばらつきより遥かに小さいことが知られている。\cite{neurohr2020relevance}
真核生物と知られる分裂酵母の細胞質密度は\mynote{Cell Biology by the Number読めば何か書いてあるんじゃね?}であり、これは\mynote{何か}と比べて遥かに多量の分子が混在する分子クラウディング(高分子混雑)環境であり、その状態が細胞機能に与える影響や細胞内制御機構は、生物学・生物物理学の重要な研究課題となっている。細胞質の分子密度は生理条件下で厳密に制御されており、細胞は過剰な高密度にも過度の低密度にもならないように制御を行っている。
\section{細胞質の流動性と適応的意義}
\mynote{Viscoadaptationの論文をまとめる}\cite{persson2020viscoadaptation}
\section{細胞質の溶解度}
\mynote{pHが下がって溶解度が下がることが示唆されている}
細胞質内の混雑や粘性が生化学反応に与える影響は、反応物の拡散、酵素活性、構造形成ダイナミクスを通じて複雑な形で現れる。高分子混雑は「除外体積効果(excluded volume effect)」を介して反応物の有効濃度を高め、熱力学的に構造化された状態(例:フォールディング、集合体形成)を安定化させる。これは特定の酵素反応や自己集合プロセスにおいて反応速度を増加させる効果がある。

一方で、混雑環境に伴う粘性上昇は分子の拡散を阻害し、反応物や酵素の空間的接触頻度を低下させることにより、反応速度を減速させる可能性がある。
\section{Mother Machineによる一細胞解析}
\section{}
\section{Off-Axis QPIの原理}
Off-axis型QPIは、サンプルによって位相が変化した光波と、基準波(リファレンスビーム)を斜め方向から干渉させ、その干渉パターンから定量的な位相情報を復元する干渉計測手法である。ホログラフィック原理とフーリエ変換を基礎とし、単一ショットで高精度な位相画像を取得できる。この技術は非侵襲・高精度に生体サンプルの屈折率変化や質量密度の評価に応用される。
サンプルによって変調された光を$\bm{E}_{\mathrm{obj}}(\bm{r})$、基準波を$\bm{E}_{\mathrm{ref}}(\bm{r})$とすると、撮像面における強度$I(\bm{r})$は次式で表される:

\begin{align}
I(\bm{r}) &= \abs{\bm{E}_{\mathrm{obj}}(\bm{r}) + \bm{E}_{\mathrm{ref}}(\bm{r})}^2 \\
&= \abs{\bm{E}_{\mathrm{obj}}(\bm{r})}^2 + \abs{\bm{E}_{\mathrm{ref}}(\bm{r})}^2 + \bm{E}_{\mathrm{obj}}(\bm{r}) \bm{E}_{\mathrm{ref}}^*(\bm{r}) + \bm{E}_{\mathrm{obj}}^*(\bm{r}) \bm{E}_{\mathrm{ref}}(\bm{r})
\end{align}

ここで$^*$は複素共役を表す。
基準波がオフアクシス角$\theta$で傾いていると、平面波として以下のように記述できる:

\begin{equation}
\bm{E}_{\mathrm{ref}}(\bm{r}) = A_{\mathrm{ref}} e^{i (\bm{k}_{\mathrm{ref}} \cdot \bm{r})}
\end{equation}

これにより干渉項は、空間周波数的にオフセットされた側帯成分を含む:

\begin{equation}
I(\bm{r}) = \text{DC} + A_{\mathrm{obj}}(\bm{r}) A_{\mathrm{ref}} e^{i (\phi_{\mathrm{obj}}(\bm{r}) - \bm{k}_{\mathrm{ref}} \cdot \bm{r})} + \text{c.c.}
\end{equation}
撮像した干渉画像$I(\bm{r})$を2次元フーリエ変換してスペクトル$I(\bm{f})$に変換する:
\begin{equation}
I(\bm{f}) = \mathcal{F}\{ I(\bm{r}) \}
\end{equation}
ここで、$\mathcal{F}$はフーリエ変換、$\bm{f}$は空間周波数である。干渉項は空間周波数空間においてオフセットされた位置に現れる。目的とする干渉項(サイドバンド)を周波数空間で切り出し、フィルタを適用したのち、逆フーリエ変換を行う:
\begin{equation}
H(\bm{r}) = \mathcal{F}^{-1}\{ \text{Filtered}[I(\bm{f})] \}
\end{equation}
これにより、複素振幅$H(\bm{r}) = A(\bm{r}) e^{i \phi(\bm{r})}$を得る。ここで、$\phi(\bm{r})$が取得した位相分布となる。
最終的な位相画像は次式で定義される:
\begin{equation}
\phi(\bm{r}) = \arg \left( H(\bm{r}) \right)
\end{equation}
なお、取得された位相には$2\pi$の不連続があるため、位相アンラッピング処理が必要である。

