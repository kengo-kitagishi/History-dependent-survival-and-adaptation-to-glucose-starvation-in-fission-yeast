\chapter{Experimental setup}
\section{Preparation of samples}

\section{Off-Axis QPI principle and the schematic of QPI}
QPI is the optimal method for the quantitative measurement of intracellular molecular distributions. 
We employ QPI for quantitative measures of the induced phase shift.
The 632 nm pulses delivered with a single-mode optical fiber are collimated 
The light transmitted through the sample is replicated by a diffraction grating, and the zeroth-order diffracted light is low-pass filtered with a pinhole placed in the Fourier plane, thus converted to a quasi-plane wave that acts as the reference light. 
The first-order diffraction light is used as the object light, which contains information on the optical phase delay induced by the sample. 
Interference fringes between the two lights are captured as an off-axis hologram with an image sensor after relay lenses in a 4f configuration, from which the phase  image is numerically reconstructed. 
The experimentally evaluated spatial resolution of QPI is hoge nm, determined by the NA of the objective lens. 
Our system is resistant to hoge noise due to hoge.

\section{Optical system of QPI}
The light source consisted of a 658 nm, 20 mW single-mode fiber-pigtailed laser diode (LP660-SF20, Thorlabs) mounted in an LDM9LP mount. The laser was controlled by an LDC202C benchtop LD current controller (±200 mA) and a TED200C temperature controller (12 W, Thorlabs), connected via CAB400 and CAB420-15 cables, respectively.
The output beam was collimated using an FC/PC fiber collimator (CFC2-B, f = 2.0 mm, Thorlabs). The optical path included two achromatic doublet lenses (ACT508-200-A, f = 200 mm, Ø2", Thorlabs) and a Ronchi ruling grating (120 lines per mm, Edmund Optics #66-342) and passed through a 25 μm pinhole (P25K, Thorlabs).
Images were acquired using a monochrome USB 3.0 CMOS camera (acA2440-75, Basler ace, 2448 × 2048 pixels).

4f光学系は、2枚のレンズとその間に配置されたフーリエ面から構成される光学システムであり、フーリエ変換および空間フィルタリングを実現するために用いられる。レンズの焦点距離を$f$とした場合、以下のように構成される:

\mynote{図は自分でまた時間を作ってaffinity designerで書きましょう。}
\begin{figure}[h]
\centering
\begin{tikzpicture}[scale=1.0]

% Planes

\end{tikzpicture}
\caption{回折格子によって分離された回折光を、4f光学系で空間フィルタリングし、+1次光のみを選択してカメラに導く構成。フーリエ面のピンホールが+1次成分だけを通過させ、角度を持った干渉縞を再構成する。}
\end{figure}

入力光波を $U_0(x)$ とし、レンズ1の前に置く。この光波は、レンズ1を通過するとフーリエ面上に以下のフーリエ変換された光が形成される:
\begin{equation}
U_f(f_x) = \mathcal{F}\{ U_0(x) \}
\end{equation}
ここで $\mathcal{F}$ はフーリエ変換、$f_x$ は空間周波数である。
レンズ2はこのフーリエ成分を再度フーリエ変換するため、出力像は入力像の反転・等倍像として得られる:
\begin{equation}
U_{\text{out}}(x) = U_0(-x)
\end{equation}
フーリエ面に空間フィルタ $H(f_x)$ を配置することで、任意の周波数成分を選択的に通過・遮断できる。これは以下のような処理を意味する:
今回はピンホールを0次光に対するローパスフィルターとしても用い、平面波化することで参照光として使えるようにした。
\begin{equation}
U_f'(f_x) = H(f_x) \cdot U_f(f_x)
\end{equation}

\begin{equation}
U_{\text{out}}(x) = \mathcal{F}^{-1} \{ U_f'(f_x) \}
\end{equation}
\section{Reconstruction Procedure of QPI}
We describe the image-reconstruction procedure in our QPI. 
In Fig.??, we use experimental data to explain image-processing workflow.
まず、サンプル画像とバックグラウンド画像を取得し、2次元高速フーリエ変換(2D-FFT)を適用し、周波数空間における干渉パターンを可視化した。オフ軸ホログラフィーにおいて、周波数空間では0次光と±1次光のピークが分離して観測される。我々は+1次光のピーク位置を特定し、その中心座標 $(y_{\text{off}}, x_{\text{off}})$ を手動で決定した。開口サイズは開口数(NA)と波長($\lambda$)、ピクセルサイズから以下の式で計算される
$$
r_{\text{aperture}} = \frac{NA}{\lambda} \cdot \text{pixelsize} \cdot N
$$
ここで、$N$は画像サイズである。
周波数空間において、+1次光のピーク位置を中心とする円形開口フィルタを適用し、空間フィルタリングを実行した。フィルタリング後の周波数データに対して逆フーリエ変換(IFFT)を行い、複素電場 $U(x,y) = A(x,y) \exp[i\phi(x,y)]$ を取得した。ここで、$A(x,y)$は振幅、$\phi(x,y)$は位相を表す。
位相情報 $\phi(x,y)$ は $-\pi$ から $\pi$ の範囲で折り返されているため、位相アンラッピングアルゴリズムを適用し、連続的な位相分布を復元した。
サンプル画像とバックグラウンド画像の両方に対して同じ処理を実施し、最終的な位相画像は以下の差分処理によって得られた
$$
\phi_{\text{final}}(x,y) = \phi_{\text{sample}}(x,y) - \phi_{\text{bg}}(x,y)
$$
さらに、バックグラウンド領域の平均値を差し引くことで、位相画像のオフセットを除去し、絶対位相値を0に規格化した。
この一連の処理により、サンプルによる光路長差を定量的に可視化した位相画像が得られる。位相値 $\phi$ から光路長差 $\Delta OPL$ への変換は、$\Delta OPL = \phi \lambda / 2\pi$ によって行われる。

\subsection{Fig. ?? Reconstruction of the images angle}
\section{Image analysis}
Omniposeによる輪郭線抽出をあのsucjuunの論文くらい詳しく書く。

\section{Microfluidic device fabrication}

\section{Fission yeast strains}

\section{Long-term time-lapse experients}
