\chapter{Summary and outlook}
\section{Summary and outlook}
[summary]
1)Together our data show that hogehoge
2)In this study, we investiagted the relationship between intracellular mass density and ability to regrowth of S.Pombe yeast cells.

We employed QPM to measure the RI-values and evaluate the mass density of yeast cells, and founf that mass density is increased in the low glucose midium.We then showd that increasing mass density triggered by osmotic stress 

[open question and discussion]
S.pombeの再復活能とLow-glucoseにおける細胞質密度の変化
細胞質密度というパラメータは生物の生きている状態を維持するために必要なものなのではないか。
Maintaining intracellular mass density is essential for starvation survival. Because nutrient starvation is a ubiquitous condition for microorganisms in natural environment, the principals underlying starvation survival uncovered in this work could elucidate microbial ecology in many nutrient poor environments. 

Why S.pombe need to maintain above minimal intracellular mass density remains an open question. 
Parhaps 最低限の細胞質密度 helps 
Another possibility is that cellular processes are influenced by the biophysical properties of the cytoplasm such as crowding and viscoelasticity. It has been suggested that cells tune the cytoplasmic properties in response to environmental changes.

Finally, some of the concepts elucidated in this work may be more general and apply to hogehoge.
1) Here, we found ATP consuming adaptation
this pattern is probably not unique to S.pombe. For example, [plasmolysis],frizid cerevisiae

adaptive solidificationとそうでないもの
1)そもそもgrowthの段階にあるものについて、今回測定された細胞質の密度の値はglass-formingな段階であり、流動性を持っているように見えるのはこれはATPを用いて流動性をmaintainしている。Nishizawa et. al., Ebata et. al.

生物種によらず高い密度を維持しており、細胞は生きている状態を維持するためにエネルギーを用いて流動性を維持している。

ATPが足りないとそもそも固体化するような高い密度であり、生きている状態を維持することと流動性を持っていることは、死んだ状態において流動性が下がりglass-like-stateになるのは当たり前。

2)化学反応場として最低限の密度を保ちながら流動性を確保するバランスを保っている。
3)これはYanagita et al.のspecific biomarkerとしてglycerolやtrehaloseが上昇することとconsistentである・
4)

 The idea that the order requires constant energy expanditure whereas decay into diosrder occurs spontaneously is intuitive and probably applies broadly, including to systems far from thermodynamic equilibrium and even to complex non-thermodynamis systems.(最後の文はNature Physicsだからかな。出す投稿誌によるだろう。最初のintroductionと対応させたほうがいい。)

流動性の話

元の細胞質密度の話。
 
流動性
液胞の役割

この計測系の優位性を示す
たとえばカニバリズムが起こっていないために死にやすくなった可能性がある。
Stationary phaseの状態をどう考えるか?





cell proteome composition
A possible explanation of this history-dependent behavior is that the cell death rate is determined by the cell’s proteome composition, i.e. the set of all the proteins in a cell at a certain time.
the final proteome composition will then determine the death rate.




