\chapter{Summary and outlook}
\section{Summary and outlook}
[summary]
1)Together our data show that hogehoge
2)In this study, we investiagted the relationship between intracellular mass density and ability to regrowth of S.Pombe yeast cells.

We employed QPM to measure the RI-values and evaluate the mass density of yeast cells, and found that mass density is increased in the low glucose midium. We then showd that increasing mass density triggered by osmotic stress 

[open question and discussion]
S.pombeの再復活能とLow-glucoseにおける細胞質密度の変化
細胞質密度というパラメータは生物の生きている状態を維持するために必要なものなのではないか。

The cytoplasm is crowded with macromolecules, with proteins being the most abundant class. 

The cytoplasmic protein concentration ranges from \~75 mg/mL in mammalian cell lines to 200–320 mg/mL in \textit{E}. \textit{coli}. Milo

For a given cell type, the concentration of macromolecules in the cytoplasm is tightly regulated and nearly constant. chen2024viscosity, oh2022protein,delarue2018mtorc1

そういえば3番目の論文に関連して、mTOC1はcarbon starvationにもnitrogen starationにも反応する広範なストレス応答因子ではなかったっけ? -> ,delarue2018mtorc1を読む。
(mTORC1 in Schizosaccharomyces pombe acts as a central sensor and mediator of stress by integrating environmental and intracellular signals. Several reports show that nitrogen starvation consistently inactivates mTORC1, which in turn induces G1 cell cycle arrest, triggers sexual differentiation, and promotes autophagy. In addition to nitrogen depletion, studies indicate that amino acid deprivation as well as osmotic, oxidative, and temperature stresses influence mTORC1 activity.
The complex directly phosphorylates effectors involved in protein synthesis (for example, S6 kinase and ribosomal protein S6) and modulates mitochondrial integrity and calcium homeostasis through components such as Wat1/mLst8. Regulation via the Tsc1/2-Rheb axis, along with input from AMP-activated protein kinase and mitogen-activated protein kinase pathways, allows mTORC1 to balance growth, autophagy, and developmental transitions. Loss of Tor2 in particular mimics nutrient shortage by shifting the cell from an anabolic state to one favoring differentiation and stress adaptation) from 251021 Elicit

carbon以外のストレスには効いてそう、普遍的に反応するもの?じゃあViscoadaptaionも普遍的で、increasing mass densityも普遍的かもしれない。
(補足)
absolute intracellular mass density, marguerat2012quantitative
ここにNitrogen starvationにおけるprotein abandanceが載っている。細胞サイズとスケールして小さくなっている。
より詳細なNitrogenのgeneticの変化、書くべきことがあったはずだか忘れたでまたかく(251021 15:00)
細胞内タンパク質量は細胞サイズとスケールしている。
mRNA量やリボソーム量はそれよりも少なくなっている。
この状態の細胞は生き返るとされているが本当だろうか。
sajiki2009genetic

Maintaining intracellular mass density is essential for starvation survival. Because nutrient starvation is a ubiquitous condition for microorganisms in natural environment, the principals underlying starvation survival uncovered in this work could elucidate microbial ecology in many nutrient poor environments. 

Why S.pombe needs to maintain above minimal intracellular mass density remains an open question. 
Parhaps 最低限の細胞質密度 helps hogehoge.
Another possibility is that cellular processes are influenced by the biophysical properties of the cytoplasm such as crowding and viscoelasticity. It has been suggested that cells tune the cytoplasmic properties in response to environmental changes.

Finally, some of the concepts elucidated in this work may be more general and apply to hogehoge.

1) Here, we found ATP consuming adaptation
this pattern is probably not unique to S.pombe. For example, [plasmolysis],[frizid cerevisiae]

adaptive solidificationとそうでないもの
1)そもそもgrowthの段階にあるものについて、今回測定された細胞質の密度の値はglass-formingな段階であり、流動性を持っているように見えるのはこれはATPを用いて流動性をmaintainしている。Nishizawa et. al., Ebata et. al.

生物種によらず高い密度を維持しており、細胞は生きている状態を維持するためにエネルギーを用いて流動性を維持している。

ATPが足りないとそもそも固体化するような高い密度であり、生きている状態を維持することと流動性を持っていることは、死んだ状態において流動性が下がりglass-like-stateになるのは当たり前。

2)化学反応場として最低限の密度を保ちながら流動性を確保するバランスを保っている。

3)これはYanagita et al.のspecific biomarkerとしてglycerolやtrehaloseが上昇することとconsistent
4)

 The idea that the order requires constant energy expanditure whereas decay into diosrder occurs spontaneously is intuitive and probably applies broadly, including to systems far from thermodynamic equilibrium

液胞の役割
autophasyによる準備,ATP dependent(なのかどうかはわからない)なmaintanance

Viscoadaptaion
persson2020cellular

この計測系の優位性を示す
他のhogehogeの現象を見るのに適用可能である。

0\%で休眠状態に入らないこととが不自然に感じられるかもしれない?一応過去の固体化論文では復活の過程を見ていないのでそのような疑問は生まれないが、読んでいてなぜ?って思うかもしれない。(CFは復活まで見ているが、joynerやmander(大腸菌)も見ていない。)

バッチよりも死にやすいのは、たとえばカニバリズムが起こっていないために死にやすくなった可能性がある。(ここに論拠を示す必要性が感じられないけど、frigid論文やion homeo論文(孫引き要確認)をあげられる)

cell cycle dependentな生存について
intracellular dry mass density of fission yeast cells fluctuates over the cell cycle, with density decreasing during interphase and increasing during mitosis and cytokinesis. odermatt2021variationsだったら簡単だけどわからない

(補足)Stationary phaseの状態をどう考えるか?MM系における細胞数の密度は?

(補足、今後)cell proteome composition
A possible explanation of this history-dependent behavior is that the cell death rate is determined by the cell’s proteome composition, i.e. the set of all the proteins in a cell at a certain time.
the final proteome composition will then determine the death rate.



繋げ方がわからないが書けそうなこと
MolecularなPathwayとの関係性(glucose starvationとcAMP-PKA pathway, nitrogen starvationと?)

glucose-triggered cAMP-PKA-Ntp1 signaling is required for cytoplasmic fluidization during germination.(Sakai et. al.)
胞子形成の方法は書いていない。Nitrogen starvationにおける経路は違うのだろうか。


その他のオルガネラの挙動(ex.ミトコンドリア、液胞)
Mitochondrion numbers increase during glucose starvation

液胞阻害剤については検討中
液胞形成の遺伝子の欠損株は成長できるのか?
液胞周りで流動性が低いとかもあるから、何か普段の成長中にも行っているような反応を細胞質全体で行っている可能性もある。garner2023vast

ミトコンドリア阻害剤についても検討中。zheng2019glucose

大腸菌はresource allocationで細胞壁の膨圧と細胞質の成長の関係性を調整しているらしいが、もしかすると酵母の場合もそういったものがあって、細胞壁合成ができないような状況にあるのかもしれない。

トレハロース