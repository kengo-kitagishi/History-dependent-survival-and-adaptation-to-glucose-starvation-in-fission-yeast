\chapter*{Appendix A}
\subsection{precision of QPM system}
Here, we demonstrate the temporal OPD noise reduction with high-intensity visible light and an
ultrahigh full-well capacity CMOS image sensor (2 Me−/pixel, Q-2HFW, Adimec), capable of detecting 200 times more photons than a standard CMOS sensor (10 ke−/pixel, e.g., acA2440-75μm,
Basler). Optical shot-noise-limited measurement (i.e., not limited by mechanical noise) is deemed
feasible by stabilizing the DH system, indicating that the greater the number of photons reaching
the image sensor, the higher the OPD precision.
We measure the dependence of temporal OPD noise on the number of electrons that contribute to
the reconstruction of the OPD images per sensor’s pixel (= Nelectron), corresponding to the number
of electrons in the noninterferometric term of the hologram shown in (2.4.2), i.e., Jnon−int. The
maximum Nelectron is limited to half of the full-well capacity of the image sensor. We note that
Nelectron is equal to the number of incident photons multiplied by the quantum efficiency of the
sensor. We record 100 holograms without a sample and calculate the OPD difference between
adjacent frames, producing 50 differential OPD images for the high-full-well-capacity 2M-e− image
sensor and conventional 10k-e− image sensor. We then obtain the temporal standard deviation
(STD) at each pixel and plot the average of 80 pixels × 80 pixels in the temporal STD map as the
temporal OPD noise in Fig. 3.6.
We next compare the measurement values and theoretical values derived from

where v represents the visibility of the hologram, denoted as 2Jint/Jnon−int (see Appendix A1 for
more details). In (3.2.1), the OPD noise in (2.4.16) is multiplied by 2 because the temporal OPD
noises are plotted from the differential OPD images. Asensor and Aaperture, representing the number
of pixels in the total and cropped areas of the spatial-frequency space (see Fig. 7.1(a)) are 2,073,600
(1,440 pixels × 1,440 pixels) and 47,144 ("/4 × 245 pixels × 245 pixels) for the 2 M-e− sensor,
and 1,046,529 (1,023 pixels × 1,023 pixels) and 31,731 ("/4 × 201 pixels × 201 pixels) for the 10
k-e− sensor, respectively. Nelectron and the visibility are calculated based on the amplitude images
of the non-interferometric and interferometric terms shown in Appendix A1 Fig. 7.1. Note that the
sensor’s digital output values are converted to the number of electrons with the sensor’s parameters
of full-well capacity, bit depth (2 M-e− sensor: 11 bit, 10 k-e− sensor: 16 bit), and gain (2 M-e−
sensor: 1.73, 10 k-e− sensor: 1).

\subsection{Optical damage to biological samples }
Optical damage could occur in the biological samples
exposed to the illumination in ADRIFT-QPM, which is dozens of times stronger than that in conventional
QPM. However, as the illumination intensity in QPM is generally much lower than that in
other live-cell imaging techniques, such as fluorescence [142] and Raman microscopy [30], the strong
illumination in the ADRIFT method does not create a significant negative effect. For example, the
illumination intensity in our demonstration is 1 nW/μm2, which is 2 and 6-8 orders of magnitude
lower than that in fluorescence and Raman imaging, respectively. Note that imaging based on light
scattering generally results in less optical damage than that based on light absorption even with
the same intensity. In fact, even with the application of a high-speed image sensor at !kHz, the
illumination intensity is 4-6 orders of magnitude lower than that in Raman imaging. Furthermore,
the optical throughput of our system becomes !5 times higher than the current condition by placing
the SLM before a sample, thereby reducing the illumination intensity to the sample.

\section*{Derivation of the Helmholtz Equation}

The Helmholtz equation describes the spatial part of a monochromatic electromagnetic wave. Here we derive it from Maxwell's equations.

\subsection*{A.1 Maxwell's equations}

In a homogeneous, isotropic, source-free medium, Maxwell's equations are
\begin{align}
\nabla \cdot \mathbf{E} &= 0, \\
\nabla \times \mathbf{E} &= -\frac{\partial \mathbf{B}}{\partial t}, \\
\nabla \cdot \mathbf{B} &= 0, \\
\nabla \times \mathbf{B} &= \mu_0 \epsilon \frac{\partial \mathbf{E}}{\partial t},
\end{align}
where $\mathbf{E}$ is the electric field, $\mathbf{B}$ is the magnetic field, $\mu_0$ is the permeability of free space, and $\epsilon = \epsilon_0 n^2$ is the permittivity with refractive index $n$.

\subsection*{A.2 Wave equation}

Taking the curl of Eq. (A.2) and using the vector identity $\nabla \times (\nabla \times \mathbf{E}) = \nabla(\nabla \cdot \mathbf{E}) - \nabla^2 \mathbf{E}$:
\begin{equation}
\nabla \times (\nabla \times \mathbf{E}) = -\frac{\partial}{\partial t}(\nabla \times \mathbf{B}).
\end{equation}
Substituting Eq. (A.1) and Eq. (A.4):
\begin{equation}
-\nabla^2 \mathbf{E} = -\mu_0 \epsilon \frac{\partial^2 \mathbf{E}}{\partial t^2}.
\end{equation}
This yields the wave equation for the electric field:
\begin{equation}
\nabla^2 \mathbf{E} - \mu_0 \epsilon \frac{\partial^2 \mathbf{E}}{\partial t^2} = 0.
\end{equation}
Using the relation $c^2 = 1/(\mu_0 \epsilon_0)$ and $\epsilon = \epsilon_0 n^2$, we can write $\mu_0 \epsilon = n^2/c^2$, giving
\begin{equation}
\nabla^2 \mathbf{E} - \frac{n^2}{c^2} \frac{\partial^2 \mathbf{E}}{\partial t^2} = 0.
\end{equation}

\subsection*{A.3 Time-harmonic fields and the Helmholtz equation}

For a monochromatic wave with angular frequency $\omega$, we assume a time-harmonic form:
\begin{equation}
\mathbf{E}(\mathbf{r}, t) = \mathbf{E}(\mathbf{r}) e^{-i\omega t}.
\end{equation}
The time derivatives become
\begin{equation}
\frac{\partial \mathbf{E}}{\partial t} = -i\omega \mathbf{E}(\mathbf{r}) e^{-i\omega t}, \quad \frac{\partial^2 \mathbf{E}}{\partial t^2} = -\omega^2 \mathbf{E}(\mathbf{r}) e^{-i\omega t}.
\end{equation}

Substituting into the wave equation (A.8):
\begin{equation}
\nabla^2 \mathbf{E}(\mathbf{r}) e^{-i\omega t} + \frac{n^2 \omega^2}{c^2} \mathbf{E}(\mathbf{r}) e^{-i\omega t} = 0.
\end{equation}
Canceling the time-dependent factor $e^{-i\omega t}$:
\begin{equation}
\nabla^2 \mathbf{E}(\mathbf{r}) + \frac{n^2 \omega^2}{c^2} \mathbf{E}(\mathbf{r}) = 0.
\end{equation}

Defining the wavenumber
\begin{equation}
k = \frac{n\omega}{c} = \frac{2\pi n}{\lambda},
\end{equation}
where $\lambda = 2\pi c/\omega$ is the vacuum wavelength, we obtain the Helmholtz equation:
\begin{equation}
\nabla^2 \mathbf{E}(\mathbf{r}) + k^2 \mathbf{E}(\mathbf{r}) = 0.
\end{equation}

For a scalar field $U(\mathbf{r})$ representing any component of the electric field or a complex amplitude, the Helmholtz equation is written as
\begin{equation}
\nabla^2 U(\mathbf{r}) + k^2 U(\mathbf{r}) = 0,
\end{equation}
where $\nabla^2 = \partial^2/\partial x^2 + \partial^2/\partial y^2 + \partial^2/\partial z^2$ is the Laplacian operator. This is the fundamental equation governing the spatial distribution of monochromatic light fields.
\subsection{4f光学系}
空間周波数フィルタリング
