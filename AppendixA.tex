\chapter*{Appendix A}
\section*{Derivation of the Helmholtz Equation}

The Helmholtz equation describes the spatial part of a monochromatic electromagnetic wave. Here we derive it from Maxwell's equations.

\subsection*{A.1 Maxwell's equations}

In a homogeneous, isotropic, source-free medium, Maxwell's equations are
\begin{align}
\nabla \cdot \mathbf{E} &= 0, \\
\nabla \times \mathbf{E} &= -\frac{\partial \mathbf{B}}{\partial t}, \\
\nabla \cdot \mathbf{B} &= 0, \\
\nabla \times \mathbf{B} &= \mu_0 \epsilon \frac{\partial \mathbf{E}}{\partial t},
\end{align}
where $\mathbf{E}$ is the electric field, $\mathbf{B}$ is the magnetic field, $\mu_0$ is the permeability of free space, and $\epsilon = \epsilon_0 n^2$ is the permittivity with refractive index $n$.

\subsection*{A.2 Wave equation}

Taking the curl of Eq. (A.2) and using the vector identity $\nabla \times (\nabla \times \mathbf{E}) = \nabla(\nabla \cdot \mathbf{E}) - \nabla^2 \mathbf{E}$:
\begin{equation}
\nabla \times (\nabla \times \mathbf{E}) = -\frac{\partial}{\partial t}(\nabla \times \mathbf{B}).
\end{equation}
Substituting Eq. (A.1) and Eq. (A.4):
\begin{equation}
-\nabla^2 \mathbf{E} = -\mu_0 \epsilon \frac{\partial^2 \mathbf{E}}{\partial t^2}.
\end{equation}
This yields the wave equation for the electric field:
\begin{equation}
\nabla^2 \mathbf{E} - \mu_0 \epsilon \frac{\partial^2 \mathbf{E}}{\partial t^2} = 0.
\end{equation}
Using the relation $c^2 = 1/(\mu_0 \epsilon_0)$ and $\epsilon = \epsilon_0 n^2$, we can write $\mu_0 \epsilon = n^2/c^2$, giving
\begin{equation}
\nabla^2 \mathbf{E} - \frac{n^2}{c^2} \frac{\partial^2 \mathbf{E}}{\partial t^2} = 0.
\end{equation}

\subsection*{A.3 Time-harmonic fields and the Helmholtz equation}

For a monochromatic wave with angular frequency $\omega$, we assume a time-harmonic form:
\begin{equation}
\mathbf{E}(\mathbf{r}, t) = \mathbf{E}(\mathbf{r}) e^{-i\omega t}.
\end{equation}
The time derivatives become
\begin{equation}
\frac{\partial \mathbf{E}}{\partial t} = -i\omega \mathbf{E}(\mathbf{r}) e^{-i\omega t}, \quad \frac{\partial^2 \mathbf{E}}{\partial t^2} = -\omega^2 \mathbf{E}(\mathbf{r}) e^{-i\omega t}.
\end{equation}

Substituting into the wave equation (A.8):
\begin{equation}
\nabla^2 \mathbf{E}(\mathbf{r}) e^{-i\omega t} + \frac{n^2 \omega^2}{c^2} \mathbf{E}(\mathbf{r}) e^{-i\omega t} = 0.
\end{equation}
Canceling the time-dependent factor $e^{-i\omega t}$:
\begin{equation}
\nabla^2 \mathbf{E}(\mathbf{r}) + \frac{n^2 \omega^2}{c^2} \mathbf{E}(\mathbf{r}) = 0.
\end{equation}

Defining the wavenumber
\begin{equation}
k = \frac{n\omega}{c} = \frac{2\pi n}{\lambda},
\end{equation}
where $\lambda = 2\pi c/\omega$ is the vacuum wavelength, we obtain the Helmholtz equation:
\begin{equation}
\nabla^2 \mathbf{E}(\mathbf{r}) + k^2 \mathbf{E}(\mathbf{r}) = 0.
\end{equation}

For a scalar field $U(\mathbf{r})$ representing any component of the electric field or a complex amplitude, the Helmholtz equation is written as
\begin{equation}
\nabla^2 U(\mathbf{r}) + k^2 U(\mathbf{r}) = 0,
\end{equation}
where $\nabla^2 = \partial^2/\partial x^2 + \partial^2/\partial y^2 + \partial^2/\partial z^2$ is the Laplacian operator. This is the fundamental equation governing the spatial distribution of monochromatic light fields.

\section{人工細胞系における細胞質密度とその粘性制御}
人工細胞は、生細胞の特定の機能を模倣するように設計された最小限のシステムである。細胞の重要な要素の一つは、生化学反応が進行する細胞質環境である。生きた細胞では、細胞質はタンパク質や核酸などの高分子物質が密集した溶液であり、細胞体積の約20~40\%を占めている。この高い細胞質密度(通常、高分子約0.2~0.3 g/mL)は「高分子混雑」と呼ばれ、粘度のような物理的特性や、酵素反応速度、タンパク質折り畳み、分子拡散などの生物学的プロセスに重大な影響を及ぼす。人工細胞が反応容器として機能するためには、この混雑した内部環境を再現しつつ、流動性のある液体状態を維持することが不可欠である。本節では、細胞質密度と粘度の定量的関係、混雑による粘度が生化学反応や拡散に与える影響について議論する。

\subsection{Cytoplasmic Density vs. Viscosity: Quantitative Relationships}

\sunsubsection{Viscosity rises nonlinearly with macromolecular density}

\subsection{Effects of Crowding and Viscosity on Biochemical Reactions}

\subsection{Measurement and Control of Cytoplasmic Viscosity in Artificial Cells}
