\chapter*{Appendix B}
\section{QPI}
\subsection{4f光学系}
空間周波数フィルタリング
\subsection{Optical damage to biological samples }
Optical damage could occur in the biological samples
exposed to the illumination in ADRIFT-QPM, which is dozens of times stronger than that in conventional
QPM. However, as the illumination intensity in QPM is generally much lower than that in
other live-cell imaging techniques, such as fluorescence [142] and Raman microscopy [30], the strong
illumination in the ADRIFT method does not create a significant negative effect. For example, the
illumination intensity in our demonstration is 1 nW/μm2, which is 2 and 6-8 orders of magnitude
lower than that in fluorescence and Raman imaging, respectively. Note that imaging based on light
scattering generally results in less optical damage than that based on light absorption even with
the same intensity. In fact, even with the application of a high-speed image sensor at !kHz, the
illumination intensity is 4-6 orders of magnitude lower than that in Raman imaging. Furthermore,
the optical throughput of our system becomes !5 times higher than the current condition by placing
the SLM before a sample, thereby reducing the illumination intensity to the sample.