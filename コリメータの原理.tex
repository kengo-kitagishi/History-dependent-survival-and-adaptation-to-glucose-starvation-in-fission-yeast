
1. ビーム径の広がり(スポットサイズの進化) 光ファイバーの端面(z=0)にあるビームウェストでは、ビーム径は最小の w 
0
​
  であり、波面は平面(曲率半径 R=∞)です。ここから距離 z だけ進んだ地点でのビーム半径(スポットサイズ)w(z) は、以下の式に従って変化します。
w(z)=w 
0
​
  
1+( 
z 
R
​
 
z
​
 ) 
2
 

​
 
ここで、z 
R
​
  は**レイリー長(Rayleigh range)**と呼ばれる重要なパラメータです。この式は双曲線の形をしており、ビームの包絡線(エンベロープ)が距離と共に双曲線を描いて広がることを示しています,。
2. レイリー長(z 
R
​
 )と広がりの基準 ビームがどれくらいの距離まで「平行(コリメート)」に近い状態を保ち、どこから急激に広がり始めるかを決定するのがレイリー長です。これは以下の式で定義されます,。
z 
R
​
 = 
λ
πw 
0
2
​
 
​
 
• 近傍界(Near Field, z≪z 
R
​
 ): ファイバー端面から z 
R
​
  よりも十分近い距離では、ビーム径はほとんど w 
0
​
  のままで変化しません。
• レイリー長地点(z=z 
R
​
 ): この距離でビーム断面積は2倍(半径は  
2

​
 w 
0
​
 )になります。
• 遠方界(Far Field, z≫z 
R
​
 ): レイリー長を越えると、上記の式の「1」が無視できるようになり、w(z)≈w 
0
​
 (z/z 
R
​
 )=(λ/πw 
0
​
 )z となります。つまり、距離 z に比例してビームが直線的に(円錐状に)広がります。
光ファイバーの場合、コア径(2w 
0
​
 )が非常に小さいため(数ミクロン〜数十ミクロン)、z 
R
​
  も非常に短くなります。その結果、ファイバーを出た光は「コリメートされた領域」をすぐに過ぎ去り、急激に広がり始めることになります。
3. 波面の曲率の変化 ビームが広がると同時に、その波面の形も変化します。ファイバー端面では平面波(R=∞)ですが、距離 z が進むにつれて波面は球面状に湾曲していきます。波面の曲率半径 R(z) は以下の式で表されます,。
R(z)=z[1+( 
z
z 
R
​
 
​
 ) 
2
 ]
ファイバーから遠く離れると(z≫z 
R
​
 )、R(z)≈z となり、ビームウェストを中心とする球面波として振る舞うようになります。
結論 したがって、光ファイバーから出た光が広がる原理は、**回折(Diffraction)**によって説明されます。非常に狭い領域(w 
0
​
 )に閉じ込められた光は、不確定性原理に似た回折効果により、進行方向に対して横方向の運動量成分を持つため、距離と共に広がろうとします。その広がり方が、近距離での平面波的な性質と遠距離での球面波的な性質を滑らかに繋ぐ「双曲線関数」として記述されるのがガウシアンビームの特徴です。