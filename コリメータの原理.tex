\chapter{ガウシアンビームの伝搬と光ファイバー}

本付録では、光ファイバーを通った光の伝搬特性について、ガウシアンビームの理論から詳細に導出する。特に、波動方程式の近軸近似から始めて、ガウシアンビームの伝搬特性、レンズによる集束、そして光導波路および光ファイバーにおける光の閉じ込めまでを系統的に説明する。

\section{指向性}

\subsection{ガウスビーム}

レーザー光は通常の光源と異なり、高い指向性を持つ。この指向性を数学的に記述する最も基本的かつ重要な解が、ガウシアンビーム(Gaussian beam)である。ガウシアンビームは、最低次の横モード解であり、実際のレーザー発振器からの出力ビームを良く近似する。

ガウシアンビームの電場振幅は、伝搬軸($z$軸とする)に垂直な平面内で、ガウス関数

\begin{equation}
\exp\left(-\frac{r^2}{w^2}\right)
\end{equation}

\noindent
の形で分布する。ここで、$r = \sqrt{x^2 + y^2}$は伝搬軸からの半径方向距離、$w$はビームのスポットサイズ(spot size)と呼ばれるパラメータである。スポットサイズ$w$は、電場振幅が中心軸上の値の$1/e$倍になる半径を表す。したがって、強度(電場振幅の2乗)は$1/e^2 \approx 0.135$倍となる。

\subsection{波動方程式と近軸解}

ガウシアンビームを厳密に導出するため、マクスウェル方程式から出発する。真空中または一様な誘電体中において、電場$\bm{E}$は波動方程式

\begin{equation}
\nabla^2 \bm{E} - \frac{1}{c^2}\frac{\partial^2 \bm{E}}{\partial t^2} = 0
\end{equation}

\noindent
を満たす。ここで、$c$は媒質中の光速である。

角周波数$\omega$、波数$k = \omega/c = 2\pi/\lambda$($\lambda$は波長)の単色平面波を考え、時間依存性を$e^{-i\omega t}$として分離すると、ヘルムホルツ方程式(Helmholtz equation)

\begin{equation}
\nabla^2 \bm{E} + k^2 \bm{E} = 0
\label{eq:helmholtz}
\end{equation}

\noindent
が得られる。

\subsubsection{近軸近似}

$z$方向に伝搬する準平面波を考え、電場を

\begin{equation}
\bm{E}(x,y,z) = u(x,y,z)e^{-ikz}\hat{\bm{x}}
\label{eq:paraxial_ansatz}
\end{equation}

\noindent
と表す。ここで、$u(x,y,z)$は緩やかに変化する包絡線関数(envelope function)、$\hat{\bm{x}}$は偏光方向の単位ベクトルである。この形は、波が主に$z$方向に伝搬し、伝搬軸からわずかにずれた方向にも広がることを表現している。

式(\ref{eq:paraxial_ansatz})をヘルムホルツ方程式(\ref{eq:helmholtz})に代入する。まず、各偏微分を計算する:

\begin{align}
\frac{\partial E}{\partial x} &= \frac{\partial u}{\partial x}e^{-ikz} \\
\frac{\partial^2 E}{\partial x^2} &= \frac{\partial^2 u}{\partial x^2}e^{-ikz} \\
\frac{\partial E}{\partial z} &= \left(\frac{\partial u}{\partial z} - iku\right)e^{-ikz} \\
\frac{\partial^2 E}{\partial z^2} &= \left(\frac{\partial^2 u}{\partial z^2} - 2ik\frac{\partial u}{\partial z} - k^2 u\right)e^{-ikz}
\end{align}

これらをヘルムホルツ方程式に代入すると、

\begin{equation}
\left(\frac{\partial^2 u}{\partial x^2} + \frac{\partial^2 u}{\partial y^2} + \frac{\partial^2 u}{\partial z^2} - 2ik\frac{\partial u}{\partial z} - k^2 u\right)e^{-ikz} + k^2 ue^{-ikz} = 0
\end{equation}

\noindent
となり、整理すると

\begin{equation}
\frac{\partial^2 u}{\partial x^2} + \frac{\partial^2 u}{\partial y^2} + \frac{\partial^2 u}{\partial z^2} - 2ik\frac{\partial u}{\partial z} = 0
\end{equation}

\noindent
が得られる。

ここで、**近軸近似(paraxial approximation)**を導入する。ビームが伝搬軸に対してわずかな角度しか持たず、包絡線関数$u$の$z$方向の変化が緩やかであると仮定する:

\begin{equation}
\left|\frac{\partial^2 u}{\partial z^2}\right| \ll k\left|\frac{\partial u}{\partial z}\right|
\label{eq:paraxial_condition}
\end{equation}

この条件のもとで、$\partial^2 u/\partial z^2$の項を無視すると、**近軸波動方程式(paraxial wave equation)**

\begin{equation}
\frac{\partial^2 u}{\partial x^2} + \frac{\partial^2 u}{\partial y^2} - 2ik\frac{\partial u}{\partial z} = 0
\label{eq:paraxial_wave_eq}
\end{equation}

\noindent
が得られる。これを

\begin{equation}
\nabla_{\perp}^2 u - 2ik\frac{\partial u}{\partial z} = 0
\label{eq:paraxial_compact}
\end{equation}

\noindent
と書くこともできる。ここで、$\nabla_{\perp}^2 = \partial^2/\partial x^2 + \partial^2/\partial y^2$は横方向のラプラシアンである。

\subsubsection{ガウシアンビーム解}

近軸波動方程式(\ref{eq:paraxial_compact})の最も基本的な解が、最低次ガウシアンビーム解である。この解を求めるため、

\begin{equation}
u(x,y,z) = A(z)\exp\left[-ik\frac{x^2+y^2}{2q(z)}\right]
\label{eq:gaussian_ansatz}
\end{equation}

\noindent
の形を仮定する。ここで、$A(z)$は振幅、$q(z)$は**複素ビームパラメータ(complex beam parameter)**と呼ばれる複素関数である。

まず、必要な偏微分を計算する:

\begin{align}
\frac{\partial u}{\partial x} &= A(z)\left(-ik\frac{x}{q(z)}\right)\exp\left[-ik\frac{x^2+y^2}{2q(z)}\right] \\
\frac{\partial^2 u}{\partial x^2} &= A(z)\left[\left(-ik\frac{x}{q(z)}\right)^2 - ik\frac{1}{q(z)}\right]\exp\left[-ik\frac{x^2+y^2}{2q(z)}\right] \\
&= A(z)\left[-k^2\frac{x^2}{q^2(z)} - ik\frac{1}{q(z)}\right]\exp\left[-ik\frac{x^2+y^2}{2q(z)}\right]
\end{align}

同様に、

\begin{equation}
\frac{\partial^2 u}{\partial y^2} = A(z)\left[-k^2\frac{y^2}{q^2(z)} - ik\frac{1}{q(z)}\right]\exp\left[-ik\frac{x^2+y^2}{2q(z)}\right]
\end{equation}

したがって、

\begin{equation}
\nabla_{\perp}^2 u = A(z)\left[-k^2\frac{x^2+y^2}{q^2(z)} - ik\frac{2}{q(z)}\right]\exp\left[-ik\frac{x^2+y^2}{2q(z)}\right]
\end{equation}

次に、$z$微分を計算する:

\begin{align}
\frac{\partial u}{\partial z} &= \frac{dA}{dz}\exp\left[-ik\frac{x^2+y^2}{2q(z)}\right] + A(z)\exp\left[-ik\frac{x^2+y^2}{2q(z)}\right] \cdot \left(-ik\frac{x^2+y^2}{2}\right) \cdot \frac{d}{dz}\left(\frac{1}{q(z)}\right) \nonumber \\
&= \left[\frac{dA}{dz} + A(z) \cdot ik\frac{x^2+y^2}{2q^2(z)}\frac{dq}{dz}\right]\exp\left[-ik\frac{x^2+y^2}{2q(z)}\right]
\end{align}

これらを近軸波動方程式(\ref{eq:paraxial_compact})に代入すると、

\begin{multline}
A(z)\left[-k^2\frac{x^2+y^2}{q^2(z)} - ik\frac{2}{q(z)}\right] \\
- 2ik\left[\frac{dA}{dz} + A(z) \cdot ik\frac{x^2+y^2}{2q^2(z)}\frac{dq}{dz}\right] = 0
\end{multline}

整理すると、

\begin{equation}
A(z)\left[-k^2\frac{x^2+y^2}{q^2(z)} - ik\frac{2}{q(z)} + k^2\frac{x^2+y^2}{q^2(z)}\frac{dq}{dz}\right] - 2ik\frac{dA}{dz} = 0
\end{equation}

この式が任意の$x, y$について成立するためには、$(x^2+y^2)$の係数と定数項が独立にゼロでなければならない。

$(x^2+y^2)$の係数:

\begin{equation}
-k^2\frac{1}{q^2(z)} + k^2\frac{1}{q^2(z)}\frac{dq}{dz} = 0
\end{equation}

\noindent
より、

\begin{equation}
\frac{dq}{dz} = 1
\label{eq:q_propagation}
\end{equation}

定数項:

\begin{equation}
-ik\frac{2}{q(z)} - 2ik\frac{1}{A(z)}\frac{dA}{dz} = 0
\end{equation}

\noindent
より、

\begin{equation}
\frac{1}{A(z)}\frac{dA}{dz} = -\frac{1}{q(z)}
\end{equation}

\noindent
すなわち、

\begin{equation}
\frac{d\ln A}{dz} = -\frac{1}{q(z)}
\label{eq:A_evolution}
\end{equation}

式(\ref{eq:q_propagation})の解は、

\begin{equation}
q(z) = q_0 + z
\label{eq:q_solution}
\end{equation}

\noindent
である。ここで、$q_0$は$z=0$における複素ビームパラメータの値である。

式(\ref{eq:A_evolution})に式(\ref{eq:q_solution})を代入して積分すると、

\begin{equation}
\ln A(z) = -\ln(q_0 + z) + C = -\ln q(z) + C
\end{equation}

\noindent
より、

\begin{equation}
A(z) = \frac{A_0}{q(z)/q_0} = A_0\frac{q_0}{q(z)}
\label{eq:A_solution}
\end{equation}

\noindent
となる。ここで、$A_0$は積分定数である。

\subsection{ガウスビームの伝搬}

\subsubsection{複素ビームパラメータの物理的意味}

複素ビームパラメータ$q(z)$を、実部と虚部に分けて

\begin{equation}
\frac{1}{q(z)} = \frac{1}{R(z)} - i\frac{\lambda}{\pi w^2(z)}
\label{eq:q_decomposition}
\end{equation}

\noindent
と定義する。ここで、$R(z)$は波面の曲率半径(radius of curvature)、$w(z)$はスポットサイズである。

式(\ref{eq:gaussian_ansatz})に式(\ref{eq:q_decomposition})を代入すると、

\begin{align}
u(x,y,z) &= A_0\frac{q_0}{q(z)}\exp\left[-ik\frac{r^2}{2q(z)}\right] \nonumber \\
&= A_0\frac{q_0}{q(z)}\exp\left[-ik\frac{r^2}{2}\left(\frac{1}{R(z)} - i\frac{\lambda}{\pi w^2(z)}\right)\right] \nonumber \\
&= A_0\frac{q_0}{q(z)}\exp\left[-ik\frac{r^2}{2R(z)}\right]\exp\left[-\frac{r^2}{w^2(z)}\right]
\label{eq:gaussian_full}
\end{align}

\noindent
となる。ここで、$k\lambda = 2\pi$を用いた。

式(\ref{eq:gaussian_full})から、ガウシアンビームの物理的意味が明確になる:

\begin{itemize}
\item $\exp[-r^2/w^2(z)]$の項は、横方向の振幅分布がガウス関数で与えられることを示す
\item $\exp[-ikr^2/2R(z)]$の項は、波面が曲率半径$R(z)$の球面であることを示す
\item $A_0 q_0/q(z)$の項は、振幅の$z$依存性を表す
\end{itemize}

\subsubsection{ビームウエストとレイリー長}

ビームが最も細くなる位置を**ビームウエスト(beam waist)**と呼ぶ。この位置を$z=0$とすると、ここで波面は平面となる、すなわち$R(0) = \infty$である。したがって、式(\ref{eq:q_decomposition})より、

\begin{equation}
\frac{1}{q_0} = -i\frac{\lambda}{\pi w_0^2}
\end{equation}

\noindent
すなわち、

\begin{equation}
q_0 = i\frac{\pi w_0^2}{\lambda} = iz_R
\label{eq:q0_definition}
\end{equation}

\noindent
となる。ここで、

\begin{equation}
z_R = \frac{\pi w_0^2}{\lambda}
\label{eq:rayleigh_range}
\end{equation}

\noindent
は**レイリー長(Rayleigh range)**と呼ばれる重要なパラメータである。

式(\ref{eq:q_solution})と式(\ref{eq:q0_definition})より、

\begin{equation}
q(z) = z + iz_R
\label{eq:q_explicit}
\end{equation}

\noindent
が得られる。

\subsubsection{スポットサイズと波面曲率の$z$依存性}

式(\ref{eq:q_explicit})を用いて、$1/q(z)$を計算する:

\begin{equation}
\frac{1}{q(z)} = \frac{1}{z + iz_R} = \frac{z - iz_R}{z^2 + z_R^2}
\end{equation}

これを式(\ref{eq:q_decomposition})と比較すると、

\begin{align}
\frac{1}{R(z)} &= \frac{z}{z^2 + z_R^2} \label{eq:R_z} \\
\frac{\lambda}{\pi w^2(z)} &= \frac{z_R}{z^2 + z_R^2} \label{eq:w_z_inv}
\end{align}

式(\ref{eq:R_z})より、波面の曲率半径は

\begin{equation}
R(z) = z\left[1 + \left(\frac{z_R}{z}\right)^2\right]
\label{eq:curvature_explicit}
\end{equation}

\noindent
である。$z=0$で$R(0) = \infty$(平面波)、$|z| \to \infty$で$R(z) \to z$(球面波)となる。

式(\ref{eq:w_z_inv})より、スポットサイズは

\begin{equation}
w^2(z) = \frac{\pi}{\lambda} \cdot \frac{z^2 + z_R^2}{z_R} = \frac{\pi z_R}{\lambda}\left[1 + \left(\frac{z}{z_R}\right)^2\right]
\end{equation}

式(\ref{eq:rayleigh_range})を用いて整理すると、

\begin{equation}
w(z) = w_0\sqrt{1 + \left(\frac{z}{z_R}\right)^2}
\label{eq:spot_size_evolution}
\end{equation}

\noindent
が得られる。

式(\ref{eq:spot_size_evolution})から、以下の重要な性質がわかる:

\begin{itemize}
\item $z=0$で$w(0) = w_0$(ビームウエスト)
\item $z = \pm z_R$で$w(\pm z_R) = \sqrt{2}w_0$(ビーム面積が2倍)
\item $|z| \gg z_R$で$w(z) \approx w_0|z|/z_R = \lambda|z|/\pi w_0$(線形拡大)
\end{itemize}

レイリー長$z_R$は、ビームがほぼ平行に伝搬する距離の目安を与える。これを**コンフォーカルパラメータ(confocal parameter)** $b = 2z_R$で表すこともある。

\subsubsection{遠方界での発散角}

$z \gg z_R$の遠方界(far field)では、式(\ref{eq:spot_size_evolution})より、

\begin{equation}
w(z) \approx \frac{w_0 z}{z_R} = \frac{\lambda z}{\pi w_0}
\end{equation}

\noindent
となる。したがって、ビームの発散半角$\theta$は、

\begin{equation}
\theta = \lim_{z\to\infty}\frac{w(z)}{z} = \frac{\lambda}{\pi w_0} = \frac{1}{z_R/w_0}
\label{eq:divergence_angle}
\end{equation}

\noindent
で与えられる。これは、ビームウエストでのスポットサイズが小さいほど、発散角が大きくなることを示している。これは回折現象の基本的な性質である。

$1/e$点での電場振幅を基準とすると、全発散角は

\begin{equation}
2\theta_{1/e} = \frac{2\lambda}{\pi w_0}
\label{eq:full_divergence}
\end{equation}

\noindent
となる。

\subsubsection{Guoy位相シフト}

ガウシアンビームの完全な表式は、式(\ref{eq:gaussian_full})に振幅因子を含めて、

\begin{equation}
u(x,y,z) = \frac{w_0}{w(z)}\exp\left[-\frac{r^2}{w^2(z)}\right]\exp\left[-ik\frac{r^2}{2R(z)}\right]\exp[-ikz + i\psi(z)]
\label{eq:gaussian_complete}
\end{equation}

\noindent
と書ける。ここで、$\psi(z)$は**Guoy位相シフト(Gouy phase shift)**と呼ばれる追加の位相項である。

$A_0 q_0/q(z) = w_0/w(z) \cdot \exp[i\psi(z)]$となるように$\psi(z)$を定義すると、

\begin{equation}
\frac{q_0}{q(z)} = \frac{iz_R}{z + iz_R} = \frac{z_R}{\sqrt{z^2+z_R^2}}\exp[i\psi(z)]
\end{equation}

\noindent
より、

\begin{equation}
\psi(z) = \arctan\left(\frac{z}{z_R}\right)
\label{eq:gouy_phase}
\end{equation}

\noindent
が得られる。

Guoy位相シフトの重要な性質:

\begin{itemize}
\item $z=0$で$\psi(0) = 0$
\item $z = z_R$で$\psi(z_R) = \pi/4$
\item $z \to \infty$で$\psi(\infty) = \pi/2$
\item $z \to -\infty$で$\psi(-\infty) = -\pi/2$
\end{itemize}

したがって、ビームがビームウエストを通過する際に、合計$\pi$の位相シフトを獲得する。この位相シフトは、ビームの集束と発散に伴う本質的な現象である。

\section{集束性}

\subsection{球面凸レンズによる集束}

ガウシアンビームを小さなスポットに集束させることは、光ディスクへのデータ記録、レーザー顕微鏡、材料加工など、多くの応用において重要である。ここでは、理想的な薄肉レンズによるガウシアンビームの集束を解析する。

\subsubsection{薄肉レンズの変換則}

焦点距離$f$の理想的な薄肉レンズを考える。レンズは波面の曲率半径を変化させるが、ビームのスポットサイズは変化させない。球面波の場合、レンズの前後での曲率半径$R_1$、$R_2$の関係は、

\begin{equation}
\frac{1}{R_2} = \frac{1}{R_1} - \frac{1}{f}
\label{eq:lens_law_spherical}
\end{equation}

\noindent
で与えられる。ここで、レンズから波面中心に向かう方向を正とする曲率半径の符号規約を用いている。

ガウシアンビームの場合も、複素ビームパラメータに対して同様の変換則が成立する:

\begin{equation}
\frac{1}{q_2} = \frac{1}{q_1} - \frac{1}{f}
\label{eq:lens_law_gaussian}
\end{equation}

これは、式(\ref{eq:q_decomposition})の実部が式(\ref{eq:lens_law_spherical})を満たし、虚部(スポットサイズに関係)は不変であることから導かれる。

\subsubsection{平行ビームの集束}

レンズの位置を$z=0$、レンズ直前でのビームパラメータを$q_1$、直後を$q_2$とする。レンズから距離$z$の位置でのビームパラメータは、

\begin{equation}
q(z) = q_2 + z
\end{equation}

\noindent
である。

レンズに入射する平行ビーム($R_1 = \infty$、スポットサイズ$w_1$)を考える。このとき、

\begin{equation}
\frac{1}{q_1} = -i\frac{\lambda}{\pi w_1^2}
\end{equation}

式(\ref{eq:lens_law_gaussian})より、

\begin{equation}
\frac{1}{q_2} = \frac{1}{f} - i\frac{\lambda}{\pi w_1^2}
\end{equation}

レンズ後方の距離$z$での複素ビームパラメータは、

\begin{equation}
q(z) = \frac{1}{\frac{1}{f} - i\frac{\lambda}{\pi w_1^2} - \frac{1}{z}}
\end{equation}

集束されたビームのウエスト($R = \infty$)の位置$z_0$は、$1/q(z_0)$の実部がゼロとなる条件から求まる:

\begin{equation}
\frac{1}{f} - \frac{1}{z_0} = 0 \quad \Rightarrow \quad z_0 = f
\end{equation}

ただし、これは近似的な結果である。正確には、

\begin{equation}
z_0 = \frac{f}{1 + (f/z_R)^2}
\label{eq:focal_shift}
\end{equation}

\noindent
となり、**焦点シフト(focal shift)**と呼ばれる効果が生じる。しかし、通常$f \gg z_R$であるため、$z_0 \approx f$と近似できる。

集束されたビームウエストでのスポットサイズ$w_0$は、$z=z_0$での$q(z_0)$の虚部から、

\begin{equation}
\frac{\lambda}{\pi w_0^2} = \frac{\lambda}{\pi w_1^2}
\end{equation}

より正確には、

\begin{equation}
w_0 = \frac{\lambda f}{\pi w_1}\left[1 + \left(\frac{z_R}{f}\right)^2\right]^{-1/2} \approx \frac{\lambda f}{\pi w_1}
\label{eq:focused_spot_size}
\end{equation}

\noindent
となる。

\subsubsection{開口径とF値}

レンズの直径を$D$とする。実用的な設計基準として、ガウシアンビームが99\%の電力を透過するためには、

\begin{equation}
D = \pi w_1
\label{eq:aperture_criterion}
\end{equation}

\noindent
が必要である($d = \pi w$基準)。

レンズの**F値(f-number)**は、

\begin{equation}
F\# = \frac{f}{D}
\label{eq:f_number}
\end{equation}

\noindent
で定義される。式(\ref{eq:focused_spot_size})と式(\ref{eq:aperture_criterion})を用いると、集束スポットの直径$d_0 = 2w_0$は、

\begin{equation}
d_0 \approx \frac{2\lambda f}{\pi w_1} = \frac{2\lambda}{\pi} \cdot \frac{f}{w_1} = \frac{2\lambda}{\pi} \cdot \pi F\# = 2\lambda F\#
\label{eq:diffraction_limit}
\end{equation}

\noindent
となる。これは、**回折限界(diffraction limit)**を表す基本的な関係式である。

\subsection{レンズによる集束の限界}

\subsubsection{最小スポットサイズ}

式(\ref{eq:diffraction_limit})から、集束スポットサイズを小さくするには、

\begin{enumerate}
\item 波長$\lambda$を短くする
\item F値を小さくする(大口径レンズを使う)
\end{enumerate}

\noindent
ことが必要である。しかし、実用上の制約がある:

\begin{itemize}
\item 波長:可視光の場合、$\lambda = 400\text{--}700$ nm
\item F値:$F\# < 1$のレンズは非常に高価で、収差が大きい
\end{itemize}

典型的には、$F\# \approx 1$のレンズで、$d_0 \approx 2\lambda \approx 1$ μm程度が実現可能な最小スポットサイズである。

\subsubsection{焦点深度}

集束されたビームのレイリー長は、式(\ref{eq:rayleigh_range})より、

\begin{equation}
z_R = \frac{\pi w_0^2}{\lambda}
\end{equation}

式(\ref{eq:focused_spot_size})を用いると、

\begin{equation}
z_R = \frac{\pi}{\lambda}\left(\frac{\lambda f}{\pi w_1}\right)^2 = \frac{\lambda f^2}{\pi w_1^2} = \frac{\lambda (F\#)^2}{\pi} \cdot \pi^2 = \pi \lambda (F\#)^2
\end{equation}

**焦点深度(depth of focus)**を$2z_R$と定義すると、

\begin{equation}
\text{焦点深度} = 2z_R \approx 2\pi \lambda (F\#)^2
\label{eq:depth_of_focus}
\end{equation}

\noindent
となる。

スポットサイズを$d_0 = N\lambda$($N$波長分)に集束すると、焦点深度は約$N^2$波長分となる。すなわち、強く集束するほど、焦点深度は浅くなる。

\subsubsection{レンズのフレネル数}

レンズの**フレネル数(Fresnel number)**は、

\begin{equation}
N_F = \frac{a^2}{\lambda f}
\label{eq:fresnel_number}
\end{equation}

\noindent
で定義される。ここで、$a = D/2$はレンズの半径である。

式(\ref{eq:aperture_criterion})の基準($D = \pi w_1$)を用いると、

\begin{equation}
N_F = \frac{(\pi w_1/2)^2}{\lambda f} = \frac{\pi^2 w_1^2}{4\lambda f}
\end{equation}

集束スポットサイズとレンズ径の関係は、

\begin{equation}
\frac{d_0}{D} \approx \frac{2\lambda F\#}{\pi w_1} = \frac{2\lambda}{\pi w_1} \cdot \frac{f}{\pi w_1} = \frac{2\lambda f}{\pi^2 w_1^2} = \frac{1}{2N_F}
\label{eq:spot_diameter_ratio}
\end{equation}

\noindent
となる。強い集束($d_0 \ll D$)には、大きなフレネル数($N_F \gg 1$)が必要である。

\subsection{ビーム品質}

\subsubsection{$M^2$因子}

実際のレーザービームは、理想的なガウシアンビームからずれていることが多い。ビームの品質を定量化するため、**$M^2$因子(M-squared factor)**が導入される。

$M^2$は、実際のビームの拡がりと理想的なガウシアンビームの拡がりの比として定義される:

\begin{equation}
M^2 = \frac{\theta_{\text{actual}}}{\theta_{\text{Gaussian}}} = \frac{w(z)/z}{w_0/z_R}
\label{eq:m_squared_definition}
\end{equation}

実際のビームのパラメータを、理想的なガウシアンビームの式に$M^2$を含めた形で表すと、

\begin{align}
w(z) &= w_0\sqrt{1 + \left(\frac{M^2 z}{z_R}\right)^2} \label{eq:beam_waist_m2} \\
z_R &= \frac{\pi w_0^2}{M^2 \lambda} \label{eq:rayleigh_m2} \\
\theta &= \frac{M^2 \lambda}{\pi w_0} \label{eq:divergence_m2}
\end{align}

$M^2$の性質:

\begin{itemize}
\item 理想的なガウシアンビーム($\text{TEM}_{00}$モード):$M^2 = 1$
\item 実際のレーザービーム:$M^2 \geq 1$
\item 高次モードを含むマルチモードビーム:$M^2 > 1$
\end{itemize}

$M^2$が大きいほど、ビームの発散が大きく、集束性が悪い。

\subsubsection{ビーム径積(BPP)}

別のビーム品質指標として、**ビーム径積(Beam Parameter Product, BPP)**がある:

\begin{equation}
\text{BPP} = w_0 \cdot \theta = \frac{M^2 \lambda}{\pi}
\label{eq:bpp}
\end{equation}

BPPは、ビームウエストでのスポットサイズと遠方界での発散角の積であり、ビームの伝搬によって不変な量である。理想的なガウシアンビームの場合、

\begin{equation}
\text{BPP}_{\text{ideal}} = \frac{\lambda}{\pi}
\end{equation}

\noindent
である。

\subsubsection{高次モードの影響}

実際のレーザー共振器では、高次のHermite-GaussianモードやLaguerre-Gaussianモードが励起されることがある。$\text{TEM}_{nm}$モード(Hermite-Gaussian)または$\text{TEM}_{pl}$モード(Laguerre-Gaussian)の$M^2$因子は、

\begin{align}
M^2_{nm} &= \sqrt{2n + 2m + 1} \quad \text{(Hermite-Gaussian)} \\
M^2_{pl} &= \sqrt{2p + l + 1} \quad \text{(Laguerre-Gaussian)}
\end{align}

\noindent
で与えられる。例えば、

\begin{itemize}
\item $\text{TEM}_{00}$:$M^2 = 1$
\item $\text{TEM}_{10}$ または $\text{TEM}_{01}$:$M^2 = \sqrt{3} \approx 1.73$
\item $\text{TEM}_{11}$:$M^2 = \sqrt{5} \approx 2.24$
\end{itemize}

複数のモードが混在する場合、全体の$M^2$は各モードの重み付き平均として計算される。

\section{導波路}

光を長距離にわたって効率的に伝送するには、自由空間の伝搬ではなく、導波路による誘導が必要である。ここでは、平面導波路と円筒型光ファイバーにおける光の閉じ込めと伝搬モードについて説明する。

\subsection{薄膜導波路}

\subsubsection{平面導波路の構造}

最も単純な導波路は、高屈折率の薄膜層(コア、屈折率$n_1$)を低屈折率の媒質(クラッド、屈折率$n_2$)で挟んだ3層構造である(図\ref{fig:planar_waveguide}参照)。

\begin{figure}[h]
\centering
\begin{tikzpicture}[scale=1.2]
% クラッド(上)
\fill[gray!20] (0,1) rectangle (6,2);
\node at (6.5,1.5) {$n_2$ (クラッド)};
% コア
\fill[gray!50] (0,0) rectangle (6,1);
\node at (6.5,0.5) {$n_1$ (コア)};
% クラッド(下)
\fill[gray!20] (0,-1) rectangle (6,0);
\node at (6.5,-0.5) {$n_2$ (クラッド)};
% 座標軸
\draw[->] (0,-1.5) -- (0,2.5) node[above] {$y$};
\draw[->] (-0.5,0.5) -- (6.5,0.5) node[right] {$z$};
% コア厚さ
\draw[<->] (-0.3,0) -- (-0.3,1) node[midway,left] {$d$};
\end{tikzpicture}
\caption{平面導波路の構造。コア(厚さ$d$、屈折率$n_1$)がクラッド(屈折率$n_2$、$n_1 > n_2$)で挟まれている。}
\label{fig:planar_waveguide}
\end{figure}

光の閉じ込めの原理は、コアとクラッドの境界面における**全反射(total internal reflection)**である。全反射が起こる臨界角$\theta_c$は、

\begin{equation}
\sin\theta_c = \frac{n_2}{n_1}
\label{eq:critical_angle}
\end{equation}

\noindent
で与えられる。入射角が臨界角より大きければ、光は境界面で完全に反射され、コア内に閉じ込められる。

\subsubsection{導波モードの条件}

$z$方向に伝搬する光の電場を、

\begin{equation}
\bm{E}(y,z,t) = \bm{E}(y)e^{i(\beta z - \omega t)}
\end{equation}

\noindent
と表す。ここで、$\beta$は**伝搬定数(propagation constant)**である。

光がコア内を全反射しながら伝搬するとき、コア内での位相変化とクラッド境界での反射による位相変化の和が$2\pi$の整数倍でなければならない。この条件から、**固有方程式(eigenvalue equation)**が導かれる。

\textbf{TEモード(横電場モード)}の場合、電場が$y$方向に垂直($\bm{E} \perp y$)で、固有方程式は、

\begin{equation}
\tan(h d) = \frac{2hq}{h^2 - q^2}
\label{eq:te_eigenvalue}
\end{equation}

\noindent
である。ここで、

\begin{align}
h &= \sqrt{n_1^2 k_0^2 - \beta^2} \label{eq:h_parameter} \\
q &= \sqrt{\beta^2 - n_2^2 k_0^2} \label{eq:q_parameter}
\end{align}

\noindent
であり、$k_0 = 2\pi/\lambda_0$は真空中の波数である。

$h$はコア内での横方向波数、$q$はクラッド内での減衰定数に対応する。

\subsubsection{規格化周波数とカットオフ}

導波路のパラメータを無次元化するため、**規格化周波数(normalized frequency)**または**V値(V-number)**を導入する:

\begin{equation}
V = k_0 d \sqrt{n_1^2 - n_2^2} = \frac{2\pi d}{\lambda_0}\text{NA}
\label{eq:v_number_planar}
\end{equation}

\noindent
ここで、**開口数(numerical aperture, NA)**は、

\begin{equation}
\text{NA} = \sqrt{n_1^2 - n_2^2} \approx n_1\sqrt{2\Delta}
\label{eq:na_definition}
\end{equation}

\noindent
であり、$\Delta = (n_1 - n_2)/n_1$は比屈折率差である。

各モード$m$($m = 0, 1, 2, \ldots$)には**カットオフV値**が存在し、

\begin{equation}
V_c^{(m)} = m\pi \quad (m = 1, 2, 3, \ldots)
\label{eq:cutoff_v}
\end{equation}

\noindent
である。$V < V_c^{(m)}$では、モード$m$は伝搬できない。

最低次モード($m=0$)はカットオフを持たない($V_c^{(0)} = 0$)。したがって、どんなに薄い導波路でも、必ず少なくとも1つのモードが存在する。

単一モード動作の条件は、

\begin{equation}
V < \pi
\label{eq:single_mode_condition}
\end{equation}

\noindent
である。

\subsubsection{電場分布}

TEモードの電場振幅$E_x(y)$は、

\begin{equation}
E_x(y) = \begin{cases}
A e^{qy} & y < -d/2 \text{ (下部クラッド)} \\
B \cos(hy - \phi) & -d/2 < y < d/2 \text{ (コア)} \\
C e^{-qy} & y > d/2 \text{ (上部クラッド)}
\end{cases}
\label{eq:te_field_distribution}
\end{equation}

\noindent
である。ここで、$A$、$B$、$C$は振幅定数、$\phi$は位相定数であり、境界条件から決定される。

クラッド内では電場が指数関数的に減衰する(エバネッセント波)。この減衰距離を**浸み出し深さ(penetration depth)**と呼び、

\begin{equation}
d_p = \frac{1}{q}
\label{eq:penetration_depth}
\end{equation}

\noindent
で与えられる。

\subsection{円筒型光ファイバー}

\subsubsection{光ファイバーの構造}

実用的な光ファイバーは、円筒対称の構造を持つ(図\ref{fig:optical_fiber}参照)。中心のコア(半径$a$、屈折率$n_1$)を、より低屈折率のクラッド(外半径$b$、屈折率$n_2$、$n_1 > n_2$)が囲んでいる。さらに外側には保護被覆がある。

\begin{figure}[h]
\centering
\begin{tikzpicture}[scale=1.5]
% クラッド
\fill[gray!20] (0,0) circle (1.5);
% コア
\fill[gray!60] (0,0) circle (0.5);
% 保護被覆
\draw[thick] (0,0) circle (2);
% ラベル
\node at (0,0) {\small コア};
\node at (1,0.7) {\small クラッド};
\node at (1.7,1.2) {\small 被覆};
% 半径
\draw[<->] (0,0) -- (0.5,0) node[midway,above] {\small $a$};
\draw[<->] (0,-0.6) -- (1.5,-0.6) node[midway,below] {\small $b$};
% 屈折率プロファイル
\begin{scope}[xshift=4cm]
\draw[->] (0,-1.5) -- (0,1.5) node[above] {$n(r)$};
\draw[->] (-0.5,0) -- (2.5,0) node[right] {$r$};
\draw[very thick] (-0.3,1) -- (0.5,1);
\draw[very thick] (0.5,0.5) -- (2,0.5);
\draw[dashed] (0.5,0) -- (0.5,1);
\node[below] at (0.5,0) {$a$};
\node[left] at (0,1) {$n_1$};
\node[left] at (0,0.5) {$n_2$};
\end{scope}
\end{tikzpicture}
\caption{円筒型光ファイバーの構造(左)と屈折率プロファイル(右)。ステップインデックス型ファイバーの場合。}
\label{fig:optical_fiber}
\end{figure}

\subsubsection{円筒座標系における波動方程式}

円筒座標系$(r, \phi, z)$において、電場の$z$成分$E_z$は、ヘルムホルツ方程式

\begin{equation}
\nabla^2 E_z + n^2(r)k_0^2 E_z = 0
\end{equation}

\noindent
を満たす。ここで、$n(r)$は半径方向の屈折率分布である。

伝搬する波を$E_z(r,\phi,z) = E_z(r)e^{il\phi}e^{i\beta z}$と表すと($l$は角度量子数)、

\begin{equation}
\frac{1}{r}\frac{d}{dr}\left(r\frac{dE_z}{dr}\right) + \left(n^2(r)k_0^2 - \beta^2 - \frac{l^2}{r^2}\right)E_z = 0
\label{eq:cylindrical_wave_eq}
\end{equation}

\noindent
が得られる。

\subsubsection{ステップインデックス型ファイバーのモード}

屈折率がコアとクラッドで急峻に変化する**ステップインデックス型ファイバー**の場合、

\begin{equation}
n(r) = \begin{cases}
n_1 & r < a \\
n_2 & r > a
\end{cases}
\end{equation}

式(\ref{eq:cylindrical_wave_eq})の解は、コア内ではベッセル関数$J_l$、クラッド内では変形ベッセル関数$K_l$で表される:

\begin{equation}
E_z(r) = \begin{cases}
A J_l(ur/a) & r < a \text{ (コア)} \\
B K_l(wr/a) & r > a \text{ (クラッド)}
\end{cases}
\label{eq:fiber_field_solution}
\end{equation}

\noindent
ここで、

\begin{align}
u &= a\sqrt{n_1^2 k_0^2 - \beta^2} \\
w &= a\sqrt{\beta^2 - n_2^2 k_0^2}
\end{align}

\noindent
であり、$A$、$B$は振幅定数である。

境界条件($r=a$で電場と磁場が連続)から、固有方程式が導かれる。

\subsubsection{規格化周波数とモード数}

円筒型ファイバーの規格化周波数は、

\begin{equation}
V = k_0 a \sqrt{n_1^2 - n_2^2} = \frac{2\pi a}{\lambda_0}\text{NA}
\label{eq:v_number_fiber}
\end{equation}

\noindent
で定義される。

$V$値は、ファイバーが伝搬できるモード数を決定する重要なパラメータである。ステップインデックス型ファイバーで伝搬可能なモード数$M$は、近似的に、

\begin{equation}
M \approx \frac{V^2}{2}
\label{eq:mode_number}
\end{equation}

\noindent
で与えられる。

\subsubsection{シングルモードファイバー}

単一モード動作の条件は、

\begin{equation}
V < 2.405
\label{eq:single_mode_fiber}
\end{equation}

\noindent
である。この値は、$\text{LP}_{11}$モードのカットオフV値に対応する。

シングルモードファイバー(SMF)では、最低次の$\text{LP}_{01}$モードのみが伝搬する。このモードの電場分布は、近似的にガウス分布で表される:

\begin{equation}
E(r) \approx E_0 \exp\left(-\frac{r^2}{w^2}\right)
\label{eq:smf_field}
\end{equation}

\noindent
ここで、$w$はモードフィールド径(mode field diameter, MFD)に関係するパラメータで、Petermann IIの定義では、

\begin{equation}
w^2 = 2\frac{\int_0^\infty E^2(r) r\,dr}{\int_0^\infty \left(\frac{dE}{dr}\right)^2 r\,dr}
\label{eq:mfd_definition}
\end{equation}

\noindent
で与えられる。

典型的なシングルモードファイバー(波長1.55 μm用)のパラメータ:
\begin{itemize}
\item コア直径:$2a \approx 8$--10 μm
\item クラッド直径:$2b = 125$ μm
\item 比屈折率差:$\Delta \approx 0.3$\%
\item モードフィールド径:MFD $\approx 10$ μm
\end{itemize}

\subsubsection{弱導波近似とLPモード}

実用的な光ファイバーでは、比屈折率差が小さい($\Delta \ll 1$)。この場合、**弱導波近似(weakly guiding approximation)**が適用でき、モードの解析が簡単になる。

弱導波近似では、電磁場の縦成分($E_z$、$H_z$)が横成分に比べて十分小さいと仮定する。この近似のもとで、モードは**直線偏光モード(linearly polarized modes, LP modes)**として扱える。

$\text{LP}_{lm}$モードは、角度量子数$l$と半径方向の量子数$m$で分類される。最低次モードは$\text{LP}_{01}$であり、シングルモードファイバーで伝搬するのはこのモードのみである。

\subsubsection{分散特性}

光ファイバーにおける光パルスの伝搬は、**分散(dispersion)**によって制限される。主な分散メカニズムは以下の通り:

\paragraph{材料分散}
媒質の屈折率$n$が波長に依存することによる分散。群速度$v_g = c/n_g$の波長依存性で特徴づけられる。ここで、群屈折率は、

\begin{equation}
n_g = n - \lambda\frac{dn}{d\lambda}
\end{equation}

\paragraph{導波路分散}
光ファイバーの構造に起因する分散。伝搬定数$\beta$の波長依存性による。

\paragraph{全分散}
これらを合わせた分散係数$D$は、

\begin{equation}
D = -\frac{\lambda}{c}\frac{d^2\beta}{d\lambda^2}
\end{equation}

\noindent
で定義され、単位は [ps/(nm·km)] である。

石英ガラスファイバーでは、波長約1.3 μmで材料分散と導波路分散が打ち消し合い、**ゼロ分散波長**となる。長距離通信では、分散が小さい波長帯(1.3 μmまたは1.55 μm)が用いられる。

\subsubsection{ファイバーからの出力ビーム}

シングルモードファイバーの出射端からは、ガウシアンビームに近い電場分布の光が放射される。モードフィールド径を$w_f$とすると、ファイバー端での電場は、

\begin{equation}
E(r,0) = E_0 \exp\left(-\frac{r^2}{w_f^2}\right)
\end{equation}

この光の遠方界での発散角は、式(\ref{eq:divergence_angle})と同様に、

\begin{equation}
\theta_f = \frac{\lambda}{\pi w_f}
\label{eq:fiber_divergence}
\end{equation}

\noindent
で与えられる。

典型的な値($\lambda = 1.55$ μm、$w_f = 5$ μm)では、$\theta_f \approx 6°$程度である。この発散を抑え、平行ビームを得るためには、ファイバー端に**コリメータレンズ**を配置する。

\subsubsection{コリメータによる平行光の生成}

コリメータは、ファイバーから出射した発散ビームを平行光に変換する光学系である。最も単純な構成は、単レンズコリメータである(図\ref{fig:collimator}参照)。

\begin{figure}[h]
\centering
\begin{tikzpicture}[scale=1.2]
% ファイバー
\draw[thick] (0,0) circle (0.1);
\fill[gray] (0,0) circle (0.05);
\node[below] at (0,-0.2) {\small ファイバー};
% 発散光
\draw[->] (0,0) -- (1.5,0.6);
\draw[->] (0,0) -- (1.5,-0.6);
\draw[->] (0,0) -- (1.5,0.3);
\draw[->] (0,0) -- (1.5,-0.3);
\draw[->] (0,0) -- (1.5,0);
% レンズ
\draw[thick] (2,-1) .. controls (1.8,0) .. (2,1);
\draw[thick] (2,-1) .. controls (2.2,0) .. (2,1);
\node[below] at (2,-1.2) {\small レンズ};
% 平行光
\draw[->] (2.1,0.5) -- (4.5,0.5);
\draw[->] (2.1,-0.5) -- (4.5,-0.5);
\draw[->] (2.1,0.25) -- (4.5,0.25);
\draw[->] (2.1,-0.25) -- (4.5,-0.25);
\draw[->] (2.1,0) -- (4.5,0);
\node[below] at (3.5,-0.7) {\small 平行ビーム};
% 焦点距離
\draw[<->] (0,-1.5) -- (2,-1.5) node[midway,below] {$f$};
\end{tikzpicture}
\caption{コリメータによるファイバー出力の平行光化。ファイバー端をレンズの焦点位置に配置する。}
\label{fig:collimator}
\end{figure}

ファイバー端面をレンズの前側焦点位置に配置すると、レンズを通過した光は平行ビームとなる。このとき、出力ビームのスポットサイズ$w_{\text{out}}$は、式(\ref{eq:focused_spot_size})の逆の関係から、

\begin{equation}
w_{\text{out}} = \frac{f\lambda}{\pi w_f} = \frac{f}{\theta_f}
\label{eq:collimated_spot_size}
\end{equation}

\noindent
で与えられる。

より高性能なコリメータでは、非球面レンズや多レンズ系が用いられ、収差を最小化している。実用的なコリメータの性能指標には以下がある:

\begin{itemize}
\item \textbf{コリメーション性能}:出力ビームの平行度(発散角)
\item \textbf{結合効率}:逆方向にファイバーに結合する際の効率
\item \textbf{作動距離}:ファイバー端からレンズまでの距離
\item \textbf{波長依存性}:色収差の影響
\end{itemize}

高精度なコリメータでは、発散角を0.1 mrad以下に抑えることができ、長距離の光学系での光学損失を最小化できる。

\subsection{まとめ}

本章では、ガウシアンビームの伝搬理論から始めて、レンズによる集束の物理、そして光導波路と光ファイバーにおける光の閉じ込めまでを詳述した。これらの理論は、レーザー光学系の設計、光ファイバー通信、集積光学など、現代の光工学の基礎をなす。

特に重要な結果を以下にまとめる:

\begin{itemize}
\item ガウシアンビームのスポットサイズは$w(z) = w_0\sqrt{1 + (z/z_R)^2}$で変化し、レイリー長$z_R = \pi w_0^2/\lambda$で特徴づけられる
\item レンズによる集束スポットサイズは回折限界$d_0 \approx 2\lambda F\#$で制限される
\item シングルモードファイバーでは、規格化周波数$V < 2.405$の条件下で単一モード動作が実現される
\item ファイバー出力をコリメートすることで、発散角を抑えた平行ビームが得られる
\end{itemize}

これらの知見は、本研究における光学系の設計と評価において不可欠である。